\chapter{Campaign and Analysis}
\label{campaign_and_analysis}
The Analysis of this campaign bases of the observation campaign of NGC4593 in 2016 by Edward M. Cackett \parencite{cackett2018accretion}. The observations took place between the 12th of July and the 6th of August with 26 successful observations out of 27 and was performed with the Hubble Space Telescope (HST) using the Space Telescope Imaging Spectrograph (STIS). The following section will cover important properties of NGC4593 and the 2016 campaign.

\section{NGC4593}
\label{NGC4593}

NGC 4593 is an active galactic nucleus (AGN), classified as a Seyfert 1 galaxy with a \mbox{(R)SB(rs)b} barred spiral morphology. 
It is located in the southern sky at RA = 12:39:39.44, DEC = $-05$°$ 20' 39.03''$ (J2000) and has a redshift of $z = 0.0083 \pm 0.0005$, corresponding to a distance of about $35.6$ Mpc \parencite{simbaNGC4593} based on the $\Lambda$CDM model. 
The galaxy exhibits a prominent large-scale bar and nuclear dust ring connected to dust lanes along the bar, which likely channel gas toward the central region \parencite{mulchaey1997structure}, as seen in figure \ref{fig:NGC4593}. 
The AGN shows strong broad emission lines in  H$\alpha$,  H$\beta$,  H$\gamma$, Ly$\alpha$, He\,\textsc{i}, and He\,\textsc{ii} \parencite{bentz2015agn}, indicating a well-developed broad-line region. 
Reverberation mapping of the broad H$\beta$ line yields a supermassive black hole mass of $M = \left(7.63 \pm 1.62\right) \times 10^6\,M_\odot$ \parencite{bentz2015agn}, with a corresponding broad-line region radius of only a few light-days \parencite{denney2006ngc4593}. 
Multi-wavelength monitoring has revealed strong variability from X-ray to optical bands, with interband time delays indicating a UV/optical-emitting accretion disk about three times larger than predicted by standard thin-disk theory and signatures of diffuse continuum emission from the BLR, particularly around the Balmer jump \parencite{cackett2018accretion}. 
ALMA observations further reveal a central molecular gas reservoir of $\sim 10^8 \ M_\odot$ arranged in a one armed spiral and a circumnuclear ring, as well as evidence for a mild molecular outflow on scales of a few hundred parsecs \parencite{garcia2019alma}, highlighting the interplay between bar driven inflow and AGN feedback in this galaxy. 

\begin{figure}[!ht]
	\centering
	\includegraphics[width=0.5\textwidth]{pictures/Chapter3/NGC4593.PNG}
	\caption{A DSS image of NGC4593.}
	\label{fig:NGC4593}
\end{figure}

\section{2016 Campaign by E. M. Cackett}
\label{Campaign_Cackett}

E. M. Cackett's campaign was designed to study wavelength dependent continuum lags. Therefore, the STIS instrument on the Hubble Space Telescope was used with low-resolution gratings to measure a broad range of wavelengths. In each observation, spectra were taken using three different gratings: G140L, G430L, and G750L. These were used together with the $52'' \times 0.2''$ slit.\\
The characteristics of the STIS gratings used in this analysis are summarized in Table \ref{tab:stis_gratings}. After a standard pipeline-processing, a package was used to do a Charge Transfer Inefficiency correction with an algorithm based on \parencite{anderson2010empirical}. The few left rest of hot pixels got manually removed by interpolating the flux of neighbor pixels.\\


\begin{table}[h!]
	\centering
	\small
	\caption{Overview of STIS Grating Characteristics \parencite{stisgratings}}
	\label{tab:stis_gratings}
	\begin{tabular}{lcccc}
		\hline
		\textbf{Grating} & \textbf{Range [\AA]} & \textbf{Exp. Time [s]} & \textbf{Res. Power} & \textbf{Dispersion [\AA/pixel]} \\
		\hline
		G140L  & 1119--1715  & 1234 & $\sim 1000$         & 0.6 \\
		G430L  & 2888--5697  & 298  & $\sim 500 - 1000$    & 2.73 \\
		G750L  & 5245--10233 & 288  & $\sim 500 - 1000$    & 4.92 \\
		\hline
	\end{tabular}
\end{table}

\newpage


\section{Intercalibration and Determination of AVG and RMS Spectra}

Reverberation mapping requires multiple epochs to capture variability. For the 2016 campaign of NGC 4593, we retrieved 27 spectra from the ... archive of which 26 are usable for further analysis. The top panel of Figure \ref{fig:comparison_spectra} shows a section between about $4000\,\text{\AA}$ and $9000\,\text{\AA}$ of the optical spectral range from these epochs.\\
For the subsequent analysis, the average spectrum (AVG) is obtained by averaging over all epochs. Ideally, this improves the signal-to-noise ratio (S/N) sufficiently to identify spectral features in NGC 4593.  Furthermore it is essential for the reverberation mapping analysis to identify variability between the epochs, which can be obtained with the root-mean-spuare (RMS) spectrum, defined as the standard deviation of the flux at each wavelength across epochs:
\begin{equation}
	F_{\mathrm{RMS}}(\lambda) = \sqrt{\frac{1}{N-1}\sum_{i=1}^{N}\left[F_i(\lambda)-\bar{F}(\lambda)\right]^2}\,,
\end{equation}
with the mean spectrum at wavelength $\lambda$ given by
\begin{equation}
	\bar{F}(\lambda) = \frac{1}{N}\sum_{i=1}^{N} F_i(\lambda)\,.
\end{equation}

Constant features, like narrow emission lines, vanishes in the RMS spectrum, whereas variable components, like broad emission lines stands out. The top panel of Figure \ref{fig:comparison_spectra_avg_rms} shows the AVG and RMS spectra from the original retrieved data. It showes, that residual variability remains in nominally non-varying lines, especially in the forbidden features near $5000\,\text{\AA}$. This indicates small wavelength misalignment between epochs. Therefore an intercalibration anchored to the narrow [O \textsc{iii}] $\lambda5007$ line was performed. The lower panels of Figures \ref{fig:comparison_spectra} and \ref{fig:comparison_spectra_avg_rms} present the intercalibrated epochs and the corresponding AVG and RMS spectra. The disappearance of narrow features in the calibrated RMS spectrum confirms that the apparent variability in the uncalibrated RMS was induced by the wavelength shifts between the epochs rather than intrinsic line variability. For the most part, the intercalibrated AVG and RMS spectra are used in the following analysis. The exception will be for the measurement of the emission lines in the UV part of spectrum and their analysis. During the analysis of the UV emission lines it showed, that the results of the uncalibrated spectrum showed higher correlation then the results of the intercalibrated spectrum, which will be explained further in Chapter \ref{cap: Results}. 
\newpage
\begin{figure}[!ht]
	\centering
	\includegraphics[width=0.7\textwidth]{pictures/Chapter3/comparison_spectra}
	\caption{Comparison of the optical spectral range of the original spectra and the [O \textsc{iii}] $\lambda5007$ intercalibrated spectra from the 2016 campaign of NGC 4593.}
	\label{fig:comparison_spectra}
\end{figure}




\begin{figure}[!ht]
	\centering
	\includegraphics[width=0.7\textwidth]{pictures/Chapter3/comparison_avg_rms}
	\caption{Comparison of the optical spectral range of the avg and rms spectra from the original data and the [O \textsc{iii}] $\lambda5007$ intercalibrated data from the 2016 campaign of NGC 4593.}
	\label{fig:comparison_spectra_avg_rms}
\end{figure}



