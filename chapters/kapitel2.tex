\chapter{Campaign and Data Preparation}
\label{campaign_and_analysis}
The analysis of this campaign is based on the HST dataset of the observation campaign of NGC\,4503 in 2016 (PI: Cackett, E.M, Prop ID 14121). This campaign took place between the 12th of July and the 6th of August with daily observations, which resulted in 26 successful out of 27 observations. It was performed with the Hubble Space Telescope (HST) using the Space Telescope Imaging Spectrograph (STIS) with the three different Gratings. The following section will cover an overview of the properties and specifications of NGC4593 and the campaign in 2016.

\section{NGC4593}
\label{NGC4593}

NGC 4593 is classified as a Seyfert 1 galaxy with a barred spiral morphology of type (R)SB(rs)b \parencite{denney2006ngc4593}. 
It is located at $\mathrm{RA}=12^\mathrm{h}39^\mathrm{m}39.44^\mathrm{s}$, $\mathrm{Dec}=-05^\circ20'39.03''$ (J2000) and has a redshift of $z = 0.0083 \pm 0.0005$, corresponding to a distance of $\sim 35.9$\,Mpc \parencite{NED} assuming a $\Lambda$CDM cosmology. The galaxy exhibits a prominent large-scale bar and nuclear dust ring connected to dust lanes along the bar, as seen in Figure \ref{fig:NGC4593}.\\
\begin{figure}[!ht]
	\centering
	\includegraphics[width=0.5\textwidth]{pictures/Chapter3/NGC4593_Bearbeitet.PNG}
	\caption{Screenshot of NGC 4593 visualized with Aladin Lite \parencite{AladinLite_CDS} using DSS2 survey imagery \parencite{DSS_IRSA441}. The image is oriented with north up and east to the left. Right ascension increases to the left and declination increases upward.}
	\label{fig:NGC4593}
\end{figure}\\
The AGN in NGC 4593 exhibits strong broad emission lines, including Balmer and Lyman lines as well as He, O, and Ca lines among others \parencite{kollatschny1997balmer,denney2006ngc4593, cackett2018accretion, Ochmann_2025}.
Several variability and reverberation-mapping campaigns have monitored different broad emission lines(e.g. \cite{dietrich1994monitoring, kollatschny1997balmer, denney2006ngc4593}. They reported FWHM values from the mean (AVG) and root-mean-square (RMS) spectra of their respective campaigns. The RMS spectrum is defined as the standard deviation of the flux at each wavelength across epochs:
\begin{equation}
	\label{eqn:RMS}
	F_{\mathrm{RMS}}(\lambda) = \sqrt{\frac{1}{N-1}\sum_{i=1}^{N}\left[F_i(\lambda)-\bar{F}(\lambda)\right]^2}\,,
\end{equation}
with the mean spectrum at wavelength $\lambda$ given by
\begin{equation}
	\bar{F}(\lambda) = \frac{1}{N}\sum_{i=1}^{N} F_i(\lambda)\,.
\end{equation}
\cite{kollatschny1997balmer} measured $\mathrm{FWHM_{AVG/RMS}}=(3400 \pm 200)\,\mathrm{km\,s^{-1}}$ for H$\alpha$, while \cite{denney2006ngc4593} reported for H$\beta$ $\mathrm{FWHM_{AVG}} = (5142 \pm 16)\,\mathrm{km\,s^{-1}}$ and $\mathrm{FWHM_{RMS}} = (4141 \pm 416)\,\mathrm{km\,s^{-1}}$. Based on these broad-line widths, they estimated the SMBH mass to be $M \approx 1.4 \times 10^7\,M_\odot$ \parencite{kollatschny1997balmer} and $M = (9.8 \pm 2.1) \times 10^6\,M_\odot$ \parencite{denney2006ngc4593}. Overall, these results suggest that the SMBH mass is of order $10^7\,M_\odot$.\\
Furthermore, NGC 4593 shows a rare double-peaked emission-line complex involving O\,\textsc{i}\,$\lambda8446$ and the Ca\,\textsc{ii} $\lambda8498$, $\lambda8542$, $\lambda8662$ triplet \parencite{Ochmann_2025}. \cite{Ochmann_2025} found that the Ca\,\textsc{ii} triplet has an intensity ratio of 1:1:1 and line profiles that closely resemble those of O\,\textsc{i}\,$\lambda8446$, exhibiting a red-to-blue peak ratio of 4:3 and a FWHM of $\approx 3700\,\mathrm{km\,s^{-1}}$, suggesting that these lines originate in a similar high-density emission region. A fit of the asymmetric line profile of Ca\,\textsc{ii} $\lambda8662$ with an elliptic accretion disk model, shows an eccentric of $e \approx 0.22$ and a low-inclination of $i \approx 11^\circ$.




\section{The 2016 HST Campaign}
\label{Campaign_Cackett}

The HST campaign (PI: Cackett, E.M, Prop ID 14121) was designed to study wavelength-dependent continuum lags. It took place between the 12th of July and the 6th of August with daily observations, which resulted in 26 successful out of 27 observations. Observations were carried out with the Hubble Space Telescope (HST) using the Space Telescope Imaging Spectrograph (STIS) and three different gratings. The low-resolution STIS gratings provided continuous spectral coverage over a broad wavelength range of approximately  $1100\,\AA$ -- $1700\,\AA$ and $3900\,\AA$ -- $9000\,\AA$. In each observation, spectra were obtained using the G140L, G430L, and G750L. All spectra were acquired with the $52'' \times 0.2''$ slit.\\
The characteristics of the STIS gratings used in this work are summarized in Table \ref{tab:stis_gratings}. After standard pipeline processing, charge-transfer inefficiency (CTI) corrections were applied using an algorithm based on \parencite{anderson2010empirical}. The remaining hot pixels were removed manually by M. Ochmann.\\
We retrieved 27 spectra from the \cite{HASP} archive  using the HASP search form in \cite{MAST}. 26 of these spectra are usable for further analysis. The top panel of Figure \ref{fig:comparison_combined} shows all spectra in the spectral range from $4000\AA$ to $9000\AA$.\\


\begin{table}[h!]
	\centering
	\small
	\caption{Overview of STIS grating characteristics \parencite{stisgratings}.}
	\label{tab:stis_gratings}
	\begin{tabular}{lcccc}
		\hline
		\textbf{Grating} & \textbf{Range [\AA]} & \textbf{Exp. Time [s]} & \textbf{Res. Power} & \textbf{Dispersion [\AA/pixel]} \\
		\hline
		G140L  & 1119--1715  & 1234 & $\sim 1000$         & 0.6 \\
		G430L  & 2888--5697  & 298  & $\sim 500 - 1000$    & 2.73 \\
		G750L  & 5245--10233 & 288  & $\sim 500 - 1000$    & 4.92 \\
		\hline
	\end{tabular}
\end{table}




\section{Intercalibration and Determination of AVG and RMS Spectra}
\label{sec:spectra}


For the subsequent analysis, the average spectrum (AVG) is obtained by averaging over all epochs, improving the signal-to-noise ratio (S/N).  Furthermore it is essential for the reverberation mapping analysis to identify variability between the epochs, which can be assessed with the root-mean-spuare (RMS) spectrum, defined as the standard deviation of the flux at each wavelength across epochs (see. Equation \ref{eqn:RMS}).
Constant features, such as narrow emission lines, vanish in the RMS spectrum, leaving only variable components such as broad emission lines. The top panel of Figure \ref{fig:comparison_combined} shows the AVG and RMS spectra from the original retrieved data. Residual variability is still noticeable in nominally non-varying lines, particularly in the forbidden line [O \textsc{iii}]\,$\lambda5007$. This indicates small wavelength misalignment between epochs. Therefore, an intercalibration anchored to the narrow [O \textsc{iii}]\,$\lambda5007$ line was performed. This was achieved by shifting the wavelengths of the individual spectra and scaling the [O \textsc{iii}]\,$\lambda5007$ line flux to a constant value. As a narrow line, the flux of [O \textsc{iii}] $\lambda5007$ can be assumed to remain constant over the timescale of the campaign. Based on this assumption, the [O \textsc{iii}] flux in each spectrum was scaled to $(106 \pm 5)\times10^{-15}\,\mathrm{erg\,s^{-1}\,cm^{-2}}$ and the wavelength was shifted by a maximum of 1$\AA$. \\
Figures \ref{fig:comparison_combined} shows a comparison of the original and the intercalibrated epochs and the corresponding AVG and RMS spectra. The disappearance of narrow features in the calibrated RMS spectrum, especially the [O \textsc{iii}] $\lambda5007$ line, confirms that the apparent variability in the RMS of the original epochs was induced by the wavelength shifts between them, rather than intrinsic line variability. However the intercalibration was only applied to the optical part of the spectra, because a narrow-line flux calibration is only valid over a limited wavelength regime. In the following analysis, the intercalibrated AVG and RMS spectra are used for the optical range obtained with the G430L and G750L gratings, while the AVG and RMS spectra from the original epochs are used for the UV emission-line analysis, as these were acquired with the G140L grating. 

\begin{figure}[!ht]
	\centering
	\includegraphics[width=\textwidth]{pictures/Chapter3/comparison_spectra}
	\vspace{0.5cm} % optionaler Abstand zwischen den Bildern
	\includegraphics[width=\textwidth]{pictures/Chapter3/comparison_avg_rms}
	\caption{Comparison of the  spectral range between $4000 \AA$ and $9000 \AA$ from the 2016 HST campaign of NGC 4593, showing the effects of [O \textsc{iii}] $\lambda5007$ intercalibration on both the individual spectra (top) and the derived average and rms spectra (bottom).}
	\label{fig:comparison_combined}
\end{figure}



