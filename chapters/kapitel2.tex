\chapter{Campaign and Analysis}
\label{campaign_and_analysis}
The Analysis of this campaign bases of the observation campaign of NGC4593 in 2016 by Edward M. Cackett \cite{cackett2018accretion}. The observations took place between the 12th of July and the 6th of August with 26 successful observations and was performed with the Hubble Space Telescope (HST) using the Space Telescope Imaging Spectrograph (STIS). The following section will cover important properties of NGC4593 and the 2016 campaign.

\section{NGC4593}
\label{NGC4593}

\subsection{2016 Campaign by E. M. Cackett}
\label{Campaign_Cackett}

E. M. Cackett's campaign was designed to study wavelength dependent continuum lags. Therefore, the STIS instrument on the Hubble Space Telescope was used with low-resolution gratings to measure a broad range of wavelengths.In each observation, spectra were taken using three different gratings: G140L, G430L, and G750L. These were used together with the $52'' \times 0.2''$ slit.
The G140L grating was centered at 1425~\text{\AA} and covered a wavelength range from 1119~\text{\AA} to 1715~\text{\AA}, with an exposure time of 1234 seconds. The G430L grating had a central wavelength of 4300~\text{\AA}, covered the range from 2888~\text{\AA} to 5697~\text{\AA}, and had a exposure time of 298 seconds. The G750L grating, centered at 7751~\text{\AA}, covered 5245~\text{\AA} to 10233~\text{\AA} and was exposed for 288 seconds.
