\chapter{Scientific Background}
\label{chap:scientific_background}

\section{Active Galactic Nuclei}
\label{sec:agn}

Active Galactic Nuclei (AGNs) refer to the central region of active galaxies. These objects are among the most luminous in the universe, with bolometric luminosities ranging from $10^{41}$ to $10^{48} \ \mathrm{erg \ s^{-1}}$, outshining entire galaxies by several orders of magnitude \parencite{peterson1997introduction}. 
Historically, several stellar-based models were proposed, such as dense star clusters or supermassive stars. However, these scenarios were discarded, as they are expected to collapse into black holes themselves, and they cannot provide the required energy output. Today, it is understood that the enormous luminosities of AGN are powered by accretion of matter onto a supermassive black hole (SMBH) at the centers of galaxies \parencite{rees1984oldAGN}. The most widely accepted model for this accretion is a hot, rotating accretion disk surrounding the SMBH, which produces most of the observed radiation \parencite{shakura1973black}.
The following sections will outline the key components of an AGN and its variability, which is central to reverberation mapping analysis.



\begin{figure}[!ht]
	\centering
	\includegraphics[width=0.75\textwidth]{pictures/Chapter2/AGN_standard_paradigm.png}
	\caption{Different components of an AGN. Adopted from \parencite{mo2010galaxy} Figure 14.3.}
	\label{fig:agn_structure_mo}
\end{figure}


\subsection{Structure and Spectral Features of an AGN}
\label{sec:agn_structure}

Figure \ref{fig:agn_structure_mo} shows a schematic illustration of an AGN, consisting of a central supermassive black hole (SMBH), a surrounding accretion disk, a dusty torus and ionized gas regions known as the broad-line region (BLR) and narrow-line region (NLR). In some cases, relativistic jets are launched perpendicular to the plane of the accretion disk \parencite{urry1995unified}. The following subsections describe the physical components of AGN and the spectral features associated with them.


\subsubsection{Supermassive Black Hole and Accretion Disk}

The center of an AGN is defined by a supermassive black hole (SMBH), with typical masses between $10^6\,M_\odot$ and $10^{9}\,M_\odot$ \parencite{peterson2004}. It does not contribute to the AGN spectrum directly, but acts as the central engine driving observed spectral features of the AGN. It dominates the gravitational potential and, unlike inactive galaxies such as the Milky Way, it is surrounded by an accretion disk. Through viscous processes within the disk, such as turbulent friction and magneto-rotational instability, the angular momentum of the matter is transported outward, which leads to a spiraling inflow of matter toward the SMBH \parencite{shakura1973black}. 
Several models have been proposed to describe the accretion process. The most widely used model is the geometrically thin and optically thick accretion disk, which consists of ionized gas in differential rotation around the SMBH \parencite{shakura1973black, netzer2013agn}. The disk is composed primarily of ionized hydrogen and helium, with trace amounts of heavier elements \parencite{netzer2013agn}. It extends from the innermost stable circular orbit (ISCO) near the event horizon out to distances of several light-days. The radial extent of the disk is relatively small compared to galactic scales and typically ranges from a few light-hours to a few light-days, corresponding to about $10^{-3}$ to $10^{-2}$\,pc \parencite{shakura1973black, netzer2013agn}.  \\
During the accretion process a substantial fraction of the gravitational energy of the matter is transformed into thermal radiation, which accounts for the enormous luminosity observed in AGNs and heats the accretion disk to high temperatures that depend on the mass of the SMBH \parencite{netzer2013agn}. For example, the maximum effective temperature for an accretion disk around a SMBH with $M = 10^8,M_\odot$ is on the order of several $\times 10^5$ K, leading to UV and optical emission \parencite{shakura1973black, netzer2013agn}. By contrast, disks around stellar-mass black holes reach much higher temperatures (up to a few $\times 10^6$ K), and therefore emit mostly in X-rays \parencite{shakura1973black, netzer2013agn}. Due to the radial temperature gradient, the emitted spectrum cannot be described as a single blackbody. Instead, it results from a superposition of many blackbody-like components at different temperatures, often referred to as a multicolour black-body \parencite{netzer2013agn}. This produces a broad optical–UV continuum of ionizing photons, which interact with gas clouds near the nucleus and play a crucial role in shaping the spectral features of the BLR and NLR. These photons cause photoionization followed by recombination, which gives rise to the strong emission lines that are characteristic of AGN spectra \parencite{netzer2013agn}.

\subsubsection{Broad-Line and Narrow-Line Region}
\label{sec:BLRNLR}

The ionized gas clouds near the nucleus can be divided into the broad-line region (BLR) and the narrow-line region (NLR). Both regions differ in density, distance from the SMBH, and the observed line widths \parencite{urry1995unified}. The BLR is located close to the nucleus, at distances ranging from a few light-days to a few light-years from the central SMBH \parencite{goadBroadLine}(see Figure \ref{fig:agn_structure_mo}). It consists of dense gas clouds with electron densities of $n_e \approx 10^{11}\,\mathrm{cm^{-3}}$, moving at velocities of several thousand $\mathrm{km\,s^{-1}}$ due to the strong gravitational influence of the SMBH. These velocities lead to significant Doppler broadening of permitted emission lines, with widths of $(500$–$10{,}000)\,\mathrm{km\,s^{-1}}$ \parencite{peterson1997introduction}.
As described earlier, the BLR is photoionized by the continuum radiation emitted from the accretion disk. Consequently, the line emission from this region responds to changes in the continuum, leading to a strong correlation between the two and strong variability \parencite{netzer2013agn}. This relationship is particularly relevant for reverberation mapping, which will be discussed later in Section \ref{sec:reverberation_mapping}.\\ Modelling the geometry of the BLR is challenging, because several emission lines have to be considered, whose intensities vary in response to changes in the continuum radiation \parencite{netzer2013agn}. A common model assumes a spherical distribution of clouds connected to the accretion disk and located between the accretion disk and the dusty torus  \parencite{goadBroadLine}. Broad emission lines arise from permitted transitions such as H$\alpha$, H$\beta$, and Ly$\alpha$ \parencite{netzer2013agn}.
\\\\
The narrow-line region (NLR) extends out to several hundred parsecs from the central region \parencite{peterson1997introduction}. The gas in this region moves at much lower velocities, resulting in emission lines with widths typically of order $(350$--$400)\,\mathrm{km\,s^{-1}}$ \parencite{peterson1997introduction}. In contrast to the BLR, the NLR exhibits both permitted and forbidden transitions. Forbidden lines, such as [O \textsc{iii}] $\lambda5007$, are prominent in the NLR because at its low densities ($n_e \sim 10^2$–$10^4\,\mathrm{cm^{-3}}$) collisional de-excitation is rare, allowing radiative decay from metastable levels \parencite{peterson1997introduction}. Due to its much larger extent compared to the BLR, the NLR responds only very slowly to variations in the ionizing continuum. Therefore, the flux of the narrow emission lines can be treated as constant over timescales of several years \parencite{peterson1993}. Because permitted emission lines can also originate in the NLR, multi-component emission-line profiles can be observed in AGN spectra \parencite{peterson1997introduction}.





\subsubsection{Dusty Torus}

Surrounding the accretion disk and the broad-line region is the dusty torus, a geometrically thick, optically dense structure composed of gas and dust. It extends from the radius at which dust can survive the intense radiation from the accretion disk out to scales of a few parsecs \parencite{netzer2013agn}. The torus likely has a clumpy distribution and plays a crucial role in the unified model of AGNs \parencite{urry1995unified, netzer2013agn}.
The dust in the torus absorbs a significant fraction of the UV and optical radiation emitted by the accretion disk and re-emits it thermally in the infrared. As a result, AGNs typically exhibit strong infrared emission, with the peak wavelength depending on the dust temperature in the torus \parencite{netzer2013agn}.
Even when the central region is hidden from direct view by the torus, this reprocessed infrared emission remains observable. It therefore provides a characteristic signature of obscured AGN activity and enables indirect constraints on the otherwise hidden central engine \parencite{netzer2013agn}.


\begin{figure}[!ht]
	\centering
	\includegraphics[width=0.75\textwidth]{pictures/Chapter2/Syefert1vsSeyfer2}
	\caption{An example of Seyfert I and Seyfert II spectra illustrating their differences. Broad lines, such as the highlighted $H\alpha$ and $H\beta$, are only present in the Seyfert I spectrum, whereas forbidden [O III] lines are visible in both cases. Adapted from \parencite{keel2002quasars}.}
	
	\label{fig:Seyfert1vsSeyfert2}
\end{figure}




\subsection{Classification}
\label{sec:classification}

AGNs get classified in subgroups based on their spectral features, which are strongly dependent to their intrinsic structure. The key parameters for this classification are luminosity, emission-line profiles and radio properties. Based on those parameters AGN get grouped into Seyfert galaxies, quasars and radio galaxies. They get further subdivided based on the appearance of broad and narrow emission lines. Some examples for these sub-classes are narrow-line Seyfert I galaxies (NLS1s), low-ionization nuclear emission-line regions (LINERs), and jet-dominated sources such as BL Lac objects or blazars \parencite{antonucci1993unified, urry1995unified}. 


\subsubsection{Seyfert Galaxies}

Seyfert galaxies are named after Carl K. Seyfert, who in 1943 observed spiral galaxies characterized by exceptionally bright nuclei and strong emission lines in their optical spectra \parencite{seyfert1943nuclear}. They are mainly classified into the sub-classes Seyfert I and Seyfert II based on the presence of broad emission lines. Figure \ref{fig:Seyfert1vsSeyfert2} highlights the differences of the spectra of Type I and Type II Seyfert galaxies.\\
Seyfert I galaxies, such as NGC 4593, show both broad and narrow emission lines in their optical spectra. The broad lines, such as $H\alpha$ and $H\beta$, typically have full widths at half maximum (FWHM) of several thousand kilometers per second and from the fast-moving, high-density gas in the BLR \parencite{peterson1997introduction}. In contrast, narrow lines, including prominent forbidden transitions like [O\,\textsc{iii}] $\lambda5007$ or [N\,\textsc{ii}] $\lambda6584$, originate from the slow-moving, low-density gas in the NLR \parencite{ peterson1997introduction}. The presence of both components in the spectrum allows for a clear classification as a Seyfert I galaxy, which is the case for NGC 4593. Further details on NGC 4593 are provided in Section \ref{NGC4593}. Between the two main Seyfert classes, several intermediate subclasses (1.2, 1.5, 1.8, 1.9) are defined based on the ratio of the broad towards the narrow components in the optical spectrum \parencite{osterbrock1977SeyfertSub, osterbrock1981SeyfertSub, peterson1997introduction}. Seyfert 1.8 and 1.9 galaxies show very weak broad components. In Seyfert 1.9 objects, the broad component is visible only in the H$\alpha$ line, whereas in Seyfert 1.8 objects it is also detectable in H$\beta$. Furthermore, if the broad and narrow components in H$\beta$ are of equal strength, the AGN is classified as a Seyfert 1.5 \parencite{peterson1997introduction}. If the narrow component is even weaker than the broad component, it is classified as a Seyfert 1.2 \parencite{osterbrock1977SeyfertSub}. The fact that the optical spectrum shows multi-component lines with both broad and narrow components, suggests that these emission lines originate in the BLR and the NLR in the respective ratio \parencite{peterson1997introduction}. \\
In comparison, Seyfert II galaxies completely lack these broad components in their optical spectra, likely due to orientation-dependent obscuration by the dusty torus. Following that the classification of a Seyfert galaxies strongly depends on the viewing angle of the observer, which is the key point for the Unified Model of AGN, which will deepened in section \ref{sec:unification_model} \parencite{peterson1997introduction}.\\  Another notable subclass is the group of so-called narrow-line Seyfert I galaxies (NLS1s). They show most of the features of Seyfert 1 or 1.5 galaxies, except that the usually broad lines, such as the H I or He I lines, exhibit FWHM values that are only slightly larger than those of the narrow lines.  They show a wide dispersion of spectral properties, with some objects have very strong Fe II emission, whereas others show almost none. This indicates that NLS1s do not form a homogeneous class \parencite{osterbrock1985nls1}. NLSls are thought to have low-mass black holes accreting at high Eddington rates, suggesting they may
represent a young evolutionary phase of AGN activity\parencite{peterson2011massesblackholesactive, netzer2013agn}. Another possible explanation is an orientation effect. Another possible explanation is an orientation effect. When an NLS1 is observed at a very low inclination, the projected velocities are reduced, which leads to smaller observed Doppler broadening and therefore to narrow lines \parencite{osterbrock1985nls1}.

\subsubsection{Additional AGN Classes}

In addition to Seyfert galaxies, there are several other classes of AGN. Quasars, which stands for quasi-stellar radio sources, are even more luminous than Seyfert galaxies and are typically found at higher redshifts. While the host galaxies of Seyfert galaxies are still observable, quasars completely outshine their host galaxies. Since quasars show similar emission characteristics to Seyfert galaxies, the modern distinction is based mainly on luminosity: quasars are classified as high-luminosity AGNs, while Seyfert galaxies represent the lower-luminosity end \parencite{netzer2013agn}.\\
Radio galaxies form another important AGN class, distinguished by their strong radio emission and prominent jets, typically found in elliptical host galaxies. When their jets are aligned close to our line of sight, they are observed as blazars or BL Lac objects, which exhibit rapid variability and featureless optical spectra due to relativistic beaming \parencite{netzer2013agn}.\\
Finally, LINERs are low-luminosity AGNs with spectra dominated by low-ionization emission lines. The physical origin of their ionization mechanism is still debated, and in some cases, they may not be powered by accretion at all \parencite{netzer2013agn}.\\\\
While these classifications are based primarily on spectral characteristics, many of the observed differences between AGN types can be attributed to orientation effects. The Unified Model of AGN provides a framework that explains this apparent diversity through a common internal structure, viewed from different angles.

\subsection{Unification Model}
\label{sec:unification_model}

Figure \ref{fig:agn_sed} shows an illustration of the Unification Model, which was postulated by Robert Antonucci in 1993. He proposed that the visible differences in AGN spectra are not due to fundamentally different structures. Instead, they arise mainly from the viewing angle toward the AGN center and from obscuration by the dusty torus \parencite{antonucci1993unified}.\\
The figure shows with what type the same AGN would get classified depending on the observers viewing angle. Like mentioned before, the dusty torus plays a key role here, as it surrounds the central region of the AGN, the accretion disk and the fast-moving BLR. If the observer's line of sight is blocked by the torus, only radio emission, the optical/UV continuum and narrow-line emission from the NLR outside the torus can be detected. In this case, the AGN is classified as a Seyfert 2 galaxy, as the broad emission lines originating from the BLR are obscured and the optical/UV continuum from the accretion disk is only partially visible. The observer essentially views the AGN from a flat angle, looking directly at the torus.If, on the other hand, the observer has a direct view into the central region of the AGN, not obscured by the torus, the fast moving gas clouds of the BLR as well as the optical/UV emission continuum from the accretion disk become visible. In this case, both broad and narrow emission lines are visible, meaning the AGN is classified as a Seyfert 1 galaxy. \parencite{urry1995unified}.\\
The same principle applies to other AGN classes. Quasars can be considered the high-luminosity counterparts of Seyfert galaxies, where orientation and torus obscuration likewise affect their observed properties. Blazars, on the other hand, are seen when the relativistic jet is aligned closely with the observer’s line of sight, leading to strong Doppler boosting, which makes the radiation appear significantly brighter and shifted to higher frequencies than it intrinsically is \parencite{urry1995unified}.\\
Although the classical Unification Model treats AGN classification as fixed and purely geometry-driven, some AGNs have been observed to change their spectral type over time \parencite{ricci2022changinglook}. These so-called "changing-look AGNs" demonstrate that a purely orientation-based interpretation, such as the Unification Model, cannot explain all observed phenomena. They suggest that intrinsic changes, such as variations in accretion rate or obscuring material, can also affect the classification \parencite{ricci2022changinglook}.




%\begin{figure}[!ht]
%	\centering
%	\includegraphics[width=0.9\textwidth]{pictures/Chapter2/AGN_unified_model.jpg}
%	\caption{Unification model of an AGN \parencite{fermi2025figure1}.}
	%\label{fig:agn_sed}
%\end{figure}

%\begin{figure}[!ht]
%	\centering
%	\includegraphics[width=0.9\textwidth]{pictures/Chapter2/AGN_unified_model_2.png}
%	\caption{This graphic shows a schematic of the unification model of an AGN. The figure was adopted from \parencite{collmar2001agn} and was originally adapted from \parencite{urry1995unified}.}
	%\label{fig:agn_sed}
%\end{figure}


\begin{figure}[!ht]
	\centering
	\includegraphics[width=0.9\textwidth]{pictures/Chapter2/AGN_unified_model_3.png}
	\caption{This graphic shows a schematic of the unification model of an AGN. The figure was adopted from \parencite{beckmann_unified}.}
	\label{fig:agn_sed}
\end{figure}


\subsection{Variability}
\label{sec:variability}


The variability of active galactic nuclei (AGNs) is one of the key aspects that enables the study of their central regions, which generally cannot be probed with conventional spatially resolved observations. Variability is observed on timescales ranging from hours to several years and is generally stochastic, resulting in flux variations of both emission lines and continuum emission of up to a few tens of percent in the UV and optical bands \parencite{Ulrich1997, ochmann2024transient}. Although the origin of this variability is not yet fully understood, the most widely accepted models attribute it to inhomogeneities and instabilities within the accretion disk \parencite{Ulrich1997, Dexter_2010}. \\
Depending on the underlying physical process, variations occur on different characteristic timescales. Processes such as thermal fluctuations or changes in the accretion flow happen on timescales of decades to centuries for typical SMBH masses and radii, and are therefore difficult to observe directly. In contrast, processes operating on shorter timescales are easier to study. Examples include gas motions and mechanical instabilities (e.g., sound waves) within the disk, which occur on timescales of weeks to months \parencite{ricci2022changinglook}. 
The shortest timescale is the light-crossing timescale, $t_\mathrm{lc} = R/c$, which specifies how long light takes to traverse the emitting region (e.g., the broad-line region, BLR) \parencite{ricci2022changinglook}. Here, $c$ denotes the speed of light, and $R$ denotes the characteristic size or radius of the variable emitting region. Following \cite{ricci2022changinglook}, assuming a SMBH of mass $\approx 10^8\,M_\odot$, the light-crossing timescale can be written as
\begin{equation}
	t_\mathrm{lc} = \frac{R}{c} \simeq 0.87 \left(\frac{R}{150\,r_g}\right)M_8\,\mathrm{days},
\end{equation}
where $r_g = GM/c^2$ is the gravitational radius of the black hole. Thus, the light-crossing timescale of the variable emitting region is of order days, and $t_\mathrm{lc}$ scales linearly with the size of the emitting region \parencite{ricci2022changinglook}.\\
Because variations in the ionizing continuum occur on such short timescales, it is possible to measure delayed responses from other regions within the AGN that are correlated with the continuum, using long-term monitoring campaigns \parencite{peterson1997introduction}. In particular, the BLR responds to changes in the photoionizing continuum radiation of the central source with a time delay (lag) that is longer than the light-crossing timescale of the emitting region \parencite{peterson1997introduction}. This lag forms the basis of classical reverberation mapping, which will be further elaborated in the next section.




\section{Reverberation Mapping}
\label{sec:reverberation_mapping}

The main focus of this work is a classical reverberation mapping analysis of the broad lines of NGC 4593. Reverberation mapping probes the structure of the BLR around the SMBH by measuring the time delay (lag) between continuum variations and the correlated response of the broad lines. This lag can be used to constrain the BLR geometry and to estimate the SMBH mass \parencite{cackett2018accretion}.  


\subsection{Principle of Reverberation Mapping}
\label{subsec:rm_principle}

The fundamental assumption in reverberation mapping is that variations in the observed continuum flux are echoed by variations in the emission-line flux, with a measurable time delay (lag). When the continuum luminosity varies, the emission-line response follows with a measurable time lag \parencite{Cackett_2021}.
\begin{figure}[!ht]
	\centering
	\includegraphics[width=0.7\textwidth]{pictures/Chapter2/iso-delay_surface}
	\caption{Spherical BLR model and an isodelay surface, adopted from \parencite{Peterson_2004}.}
	\label{fig:iso-delay}
\end{figure}
The time lag $\tau$ corresponds to the average light-travel time between the photoionizing continuum source and the line-emitting regions. Assuming an idealized BLR consisting of spherically distributed clouds \parencite{goadBroadLine}, then $\tau$ can be written as \parencite{peterson1997introduction}
\begin{equation}
	\tau = \left(1+\cos\theta\right)\cdot \frac{R_\mathrm{BLR}}{c},
\end{equation}
where $R_\mathrm{BLR}$ is the characteristic BLR radius, $c$ is the speed of light, and $\theta$ is the angle between the line of sight and the position vector of the emitting gas with respect to the central source (see Figure \ref{fig:iso-delay}) \parencite{peterson1997introduction}. The circle in Figure~\ref{fig:iso-delay} represents the BLR modeled as spherical distribution of the clouds. For a fixed lag $\tau$, the emitting regions that respond at that delay lie on a paraboloid aligned with the observer’s line of sight, known as an isodelay surface. Therefore, emission observed at a given lag $\tau$ originates from the intersection of the BLR distribution with the corresponding isodelay surface \parencite{peterson1997introduction}. Thus, reverberation mapping can be used to "map" the BLR by inferring a characteristic BLR radius from the measured time lag \parencite{peterson1997introduction}. However, the observer receives emission from a range of delays (i.e., effectively from many isodelay surfaces), so the so-called transfer equation is required, which integrates over all delays \parencite{peterson1997introduction}:
\begin{equation}
	\label{eqn:transfer-function}
	L\left(t\right) = \int \Psi\left(\tau\right)C\left(t-\tau\right) d\tau.
\end{equation}
Here, $\Psi(\tau)$ is the transfer function, which encodes the BLR geometry and kinematics, $C(t)$ is the continuum light curve, and $L(t)$ is the emission-line light curve \parencite{peterson1997introduction}.
Although the BLR response is, in principle, fully described by the transfer function $\Psi(\tau)$, the lag is commonly estimated using cross-correlation techniques. In practice, recovering the full transfer function $\Psi\left(\tau\right)$ requires densely sampled, high signal-to-noise light curves spanning a duration much longer than the expected lag. Since real monitoring campaigns are often affected by observational gaps and noise, such reconstructions are rarely possible \parencite{horne2004observational,peterson1993}. For this reason, this project focuses on measuring the mean time lag between continuum and emission-line variations using the interpolated cross-correlation function (ICCF) method \parencite{gaskell_peterson1986}.\\


\subsection{Lag Measurement}
\label{subsec:rm_ccf}

Using the notation of \cite{peterson1997introduction}, the cross-correlation function between the ionizing continuum and an emission line is defined as \begin{equation}
	F_\mathrm{CCF}(\tau) = \int_{-\infty}^{\infty}L(t)C(t-\tau)dt.
\end{equation}
The auto-correlation function of the ionizing continuum is \begin{equation}
	F_\mathrm{ACF}(\tau) = \int_{-\infty}^{\infty}C(t)C(t-\tau)dt.
\end{equation}
Together with the transfer equation (Equation \ref{eqn:transfer-function}), the cross-correlation function can be written as the convolution of the transfer function and the auto-correlation function of the ionizing continuum:
\begin{equation}
	F_\mathrm{CCF}(\tau) = \int_{-\infty}^{\infty}\Psi(\tau') F_\mathrm{ACF}(\tau-\tau')d\tau'.
\end{equation}
The lag is commonly defined as either the peak location ($\tau_{\mathrm{peak}}$) or the centroid ($\tau_{\mathrm{centroid}}$) of the CCF \parencite{peterson1997introduction}. 
While $\tau_{\mathrm{peak}}$ is defined as the location of the CCF maximum, $\tau_{\mathrm{centroid}}$ is calculated over all CCF points above a selected threshold, typically $80\%$ of the peak value. Because the CCF is closely related to the transfer function, it is possible to infer a characteristic BLR size associated with the emission-line response \parencite{peterson1997introduction}, which can be expressed as \begin{equation}
	\label{eqn:charc_radius}
	R_\mathrm{BLR} = c \cdot \tau_{\mathrm{centroid}}.
\end{equation}
This follows from the light-travel timescale \parencite{peterson2004}. Since the centroid lag is generally considered a more robust estimator of the mean BLR light-travel time of the BLR \parencite{peterson2004}, it is used in this project.\\
The uncertainty of the measured lag is estimated using a Monte Carlo approach combining flux randomization (FR) and random subset selection (RSS) \parencite{Peterson_1998b, peterson2004}. In the FR step, each flux value is randomly perturbed according to its measurement uncertainty. In the RSS step, $N$ data points are drawn randomly with replacement, while duplicate selections are discarded, resulting in a new light curve with $M \leq N$ points. For each realization, the ICCF analysis is repeated, yielding a distribution of centroid lags. The uncertainties are estimated from the distribution of centroid lags obtained from the simulations \parencite{peterson2004}. The 16th and 84th percentiles of this distribution are adopted as the bounds of the $1\sigma$ confidence interval \parencite{Peterson_1998b}.


\subsection{Black Hole Mass}
\label{subsec:BHM}

The reverberation mapping method can be used not only to measure the characteristic size of the BLR, but also to estimate the mass of the central SMBH. Under the assumption that the gas dynamics in the BLR are dominated by the gravitational potential of the central SMBH, the black hole mass can be estimated using the virial theorem \parencite{peterson2004}.\\  
The centroid time lag $\tau_{\mathrm{centroid}}$ provides an estimate of the characteristic BLR radius via Equation 
\ref{eqn:charc_radius}. 
Together with the velocity dispersion $\Delta V$ of the BLR gas, the virial mass is given by
\begin{equation}
	\label{eqn:M_vir}
	M_{\mathrm{vir}} = \frac{R_{\mathrm{BLR}}\,\Delta V^2}{G}.
\end{equation}

The black hole mass is then given by

\begin{equation}
	\label{eqn:M_BH}
	M_{\mathrm{BH}} = f \cdot M_{\mathrm{vir}}.
\end{equation}
Here, $G$ denotes the gravitational constant, and $f$ is a scale factor that accounts for the unknown geometry, kinematics, and inclination of the BLR \parencite{peterson2004}.The velocity dispersion $\Delta V$ can be estimated from the widths of the broad emission lines \parencite{peterson2004}.\\ 
The scale factor $f$ is calibrated by matching reverberation-based black hole masses to the empirical $M_{\mathrm{BH}}-\sigma_*$ relation observed in inactive galaxies, where $\sigma_*$ denotes the stellar velocity dispersion of the galactic bulge \parencite{onken2004}. Different studies have reported values of $f$ based on various AGN samples, for example, $f = 5.5$ \parencite{onken2004}, $f = 4.31$ \parencite{gier2013a}, and $f = 3.6$ \parencite{graham2011expanded}.\\
The calibration of the scale factor also depends on the measurement method used for the line width of the broad emission lines \parencite{peterson2004}. Two commonly used measures are the line dispersion $\sigma_{\mathrm{line}}$ (see \parencite{peterson2004}) and the FWHM. In this project, the FWHM is used as the line-width measure.\\
Following \cite{probst2025emissionlinecontinuumreverberationmapping}, we adopt a scale factor of $f = 1.8$. This value is obtained by applying the relation $\mathrm{FWHM}/\sigma_{\mathrm{line}} \approx 2$ from \cite{peterson2004} to the scale factor reported by \cite{graham2011expanded}.





\section{Bowen Fluorescence of O\textsc{I}$\lambda$8446}
\label{sec:bowen_fluorescence}
Previous studies of NGC\,4593 report strong emission in the low-ionization lines O\textsc{i}$\,\lambda8446$ and the Ca\,\textsc{ii}$\,\lambda8498\,\lambda8542\,\lambda8662$ triplet \parencite{Ochmann_2025}. In particular, O\,\textsc{i}\,$\lambda8446$ is of interest because it can be enhanced through a fluorescence mechanism referred to as Bowen fluorescence \parencite{grandi1980}.
The mechanism was first described by I.S. Bowen in 1934 to explain unexpected emission lines in nebular spectra. This mechanism describes a resonant line-pumping process in which photons emitted by one ion are absorbed by another ion of a different species via a permitted transition enabled by a near coincidence in wavelength between the pumping line and the absorbing transition. The resulting de-excitation leads to an enhancement of the emission lines \parencite{bowen1934}. \\
One transition that can be enhanced by Bowen fluorescence is O\,\textsc{i}\,$\lambda8446$, which can be pumped by Ly$\beta$ photons \parencite{netzer1976}.  
In this process, Ly$\beta$ photons at $\lambda1025.72$\,\AA{} are absorbed by neutral oxygen 
through the near-resonant transition $2p^4\,^3\!P_2 \rightarrow 3d\,^3\!D^0$ of O\,\textsc{i} 
at $\lambda1025.77$\,\AA. The excited $3d\,^3\!D^0$ level decays to $3p\,^3\!P$, 
which then decays to $ 3s\,^3\!S^\circ$, 
emitting the O\,\textsc{i}\,$\lambda8446$ emission line (see figure \ref{fig:bowen_cascate} ) \parencite{grandi1980}. This provides an additional excitation channel for O\,\textsc{i}\,$\lambda8446$, in addition to recombination.



\begin{figure}[!ht]
	\centering
	\includegraphics[width=0.6\textwidth]{pictures/Chapter2/OITriplet.jpg}
	\caption{Energy level diagram displaying the process of Bowen fluorescence pumping of OI, adopted from \parencite{grandi1980}.}
	\label{fig:bowen_cascate}
\end{figure}



