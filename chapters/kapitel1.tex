\chapter{Scientific Background}
\label{chap:scientific_background}

\section{Active Galactic Nuclei}
\label{sec:agn}

What are Active Galactic Nuclei (AGN)? An Active Galactic Nucleus (AGN) refers to the central region of a galaxy that is exceptionally bright and energetic. These regions are among the most luminous objects in the universe, with bolometric luminosities ranging from $10^{41}$ to $10^{48} \ \mathrm{erg \ s^{-1}}$, outshining entire galaxies by several orders of magnitude \parencite{peterson1997introduction}.
Its emission spans the entire electromagnetic spectrum and is powered by the accretion of matter onto the supermassive black hole (SMBH) in the center of the AGN. The most common model for this accretion is a hot, rotating accretion disk around the SMBH, which is responsible for most of the observed radiation \parencite{shakura1973black}.\\
To provide a basic understanding of the structure and physical processes within AGN, the following sections outline their key components, introduce the unification model that connects various AGN types, and summarize common classification schemes. Particular emphasis is placed on AGN variability, which plays a central role in the reverberation mapping analysis conducted in this thesis.


\subsection{Structure of an AGN}
\label{sec:agn_structure}

The structure of an AGN consists of several components, which are illustrated schematically in Figure \ref{fig:agn_structure_mo}. These include a central supermassive black hole (SMBH), an accretion disk that feeds the SMBH, a surrounding dusty torus, and ionized gas regions known as the broad-line region (BLR) and narrow-line region (NLR). In some AGNs, powerful relativistic jets are launched perpendicular to the accretion disk. However, these jets will not be discussed further in this section, as they are not relevant to the scope of this thesis. Each of the components contributes differently to the observed spectrum of the AGN, as will be discussed in more detail in Section \ref{sec:spectral_features}.


\begin{figure}[!ht]
	\centering
	\includegraphics[width=0.8\textwidth]{pictures/Chapter2/AGN_standard_paradigm.png}
	\caption{Different components of an AGN. Adapted from \textcite{mo2010galaxy} Figure 14.3.}
	\label{fig:agn_structure_mo}
\end{figure}

\subsubsection{Supermassive Black Hole and Accretion Disk}


The center of an AGN is occupied by a supermassive black hole (SMBH), with masses typically ranging from $10^6\,M_\odot$ to more than $10^{10}\,M_\odot$. While the SMBH itself does not emit radiation, it dominates the gravitational potential in the innermost regions and acts as the central engine for all observed AGN phenomena. The matter from the surrounding accretion disk slowly spirals inward toward the SMBH due to angular momentum transport driven by viscosity inside the disk. During this process, gravitational potential energy is converted into heat, causing the disk to reach very high temperatures. As a result, a significant fraction of the gravitational energy of the matter is transformed into thermal radiation, which accounts for the enormous luminosity observed in AGNs \parencite{netzer2013agn}.\\
The accretion disk itself is a geometrically thin and optically thick structure composed of ionized gas in differential rotation around the SMBH \parencite{shakura1973black}. The disk's composition is primarily ionized hydrogen and helium, with traces of heavier elements \parencite{netzer2013agn}. It extends from the innermost stable circular orbit (ISCO) near the event horizon out to distances of several light-days, where the temperature drops and dust can survive. The radial extent of the disk is relatively small on galactic scales, typically ranging from a few light-hours to a few light-days, corresponding to about $10^{-3}$ to $10^{-2}$\,pc \parencite{netzer2013agn,hickox2018obscured}.  The hottest regions are located in the innermost part of the disk, with typical temperatures ranging from $10^4$ to $10^5$\,K \parencite{hickox2018obscured}. 


\subsubsection{Broad-Line and Narrow-Line Region}

Outside the accretion disk lies a distribution of photo-ionized gas clouds that are the origin of the characteristic emission lines observed in AGN spectra. The innermost of these clouds is the broad-line region (BLR), located at distances of a few light-days to light-weeks from the central SMBH. The BLR consists of dense gas clouds (electron densities $n_e \sim 10^9$–$10^{11}\,\mathrm{cm}^{-3}$) moving at high velocities of several thousand kilometers per second due to the strong gravitational influence of the black hole \parencite{netzer2013agn}. These velocities lead to a significant Doppler broadening of permitted emission lines and line widths of several thousand km/s. The BLR is primarily photo-ionized by the continuum radiation emitted from the accretion disk.\\
Further out lies the narrow-line region (NLR), which extends over scales of hundreds to thousands of parsecs. The gas in this region is less dense ($n_e \sim 10^2$–$10^6\,\mathrm{cm}^{-3}$) and moves at much lower velocities (a few hundred km/s), resulting in narrow emission lines with widths typically below $1000\,\mathrm{km/s}$. In contrast to the BLR, the NLR emits both permitted and forbidden lines, which gets excited by collision and can only form in low-density environments. The NLR is often spatially resolved in nearby AGNs and is photo-ionized by the central source, although additional ionization from shocks may also contribute in some cases \parencite{hickox2018obscured,netzer2013agn}.\\
Both the BLR and NLR serve as important diagnostics of the AGN structure and provide key information about gas dynamics, ionization mechanisms, and the orientation-dependent appearance of the active nucleus.



\subsubsection{Dusty Torus}

Surrounding the accretion disk and broad-line region is the dusty torus, a geometrically thick and optically dense structure composed of gas and dust. It extends from the sublimation radius, where dust can survive the intense radiation of the accretion disk, out to scales of a few parsecs. The torus likely has a clumpy distribution and plays a crucial role in the unified model of AGNs which will be discussed in a later section \parencite{netzer2013agn,hickox2018obscured}.




\subsection{Classification}
\label{sec:classification}
As previously implied, AGN emit across the entire electromagnetic spectrum, with their emission characteristics strongly depending on their internal structure. Consequently, each AGN exhibits a unique spectral signature, based on which they have historically been classified.\\
These classifications are based on the differences in luminosity, emission-line profiles, and radio properties. Broadly, AGNs can be grouped into Seyfert galaxies, quasars, and radio galaxies. Seyfert galaxies are further subdivided based on the width of their optical emission lines and their radio characteristics. Seyfert 1 galaxies exhibit both broad and narrow emission lines, where Seyfert 2 galaxies show only narrow emission lines. Besides those main classes there are additional subclasses including narrow-line Seyfert 1 galaxies (NLS1s), low-ionization nuclear emission-line regions (LINERs), and jet-dominated sources such as BL Lac objects or blazars \parencite{antonucci1993unified, urry1995unified}.

\subsubsection{Seyfert Galaxies}

Seyfert galaxies are named after Carl K. Seyfert, who in 1943 observed spiral galaxies characterized by exceptionally bright nuclei and prominent broad emission lines in their optical spectra \parencite{seyfert1943nuclear}. For that they represent a class of AGN with bright nuclei and strong emission lines. Today, they are classified primarily based on the presence and width of permitted and forbidden low- and high-ionized emission lines.\\
Seyfert 1 galaxies, of which NGC 4593 is an example, show both broad and narrow emission lines in their optical spectra. The broad lines, such as $H_\alpha$ and $H_\beta$, have a full width at half maximum (FWHM) of typically several thousand kilometers per second and arise from the high-velocity BLR. In contrast, narrow lines, including prominent forbidden transitions like [O\,\textsc{iii}] $\lambda5007$ or [N\,\textsc{ii}] $\lambda6584$, originate from the lower-velocity NLR \parencite{osterbrock1989agn, peterson1997introduction}.\\
The presence of both components in the spectrum allows for a clear classification as a Seyfert 1 galaxy, which is the case for NGC4593. More to NGC4593 in section \ref{NGC4593}.\\
In comparison, Seyfert 2 galaxies lack these broad components in their optical spectra, likely due to orientation-dependent obscuration by circumnuclear material. This distinction is central to the Unified Model of AGN, which attributes observed differences between Seyfert types primarily to the viewing angle rather than intrinsic differences in structure adn will deepened in section \ref{sec:unification_model} \parencite{antonucci1993unified, urry1995unified}.\\ Another notable subclass are the so-called narrow-line Seyfert 1 galaxies (NLS1s). Despite their classification as Seyfert 1, the broad permitted lines in their spectra exhibit unusually small widths, with FWHM $<$ 1000 km\,s$^{-1}$. They often show strong Fe\,\textsc{ii} emission complexes and steep soft X-ray spectra. NLS1s are thought to have low-mass black holes accreting at high Eddington ratios, suggesting they may represent a young evolutionary phase of AGN activity \parencite{osterbrock1985nls1, netzer2013agn}.



\subsubsection{Others}

Next to Seyfert galaxies, there are several other classes of AGN. Quasars, short for quasi-stellar radio sources, are even more luminous than Seyfert galaxies and are typically found at higher redshifts. While the host galaxies of Seyfert galaxies are still observable, quasars completely outshine their host galaxies. Since quasars show similar emission characteristics to Seyfert galaxies, the modern distinction is based mainly on luminosity: quasars are classified as high-luminosity AGNs, while Seyferts represent the lower-luminosity end \parencite{netzer2013agn}.\\
Radio galaxies are defining another AGN class. There are characterized by their strong radio emission and prominent jets, often associated with elliptical host galaxies. When their jets are aligned close to our line of sight, they are observed as blazars or BL Lac objects, which exhibit rapid variability and featureless optical spectra due to relativistic beaming \parencite{netzer2013agn}.\\
Finally, LINERs are low-luminosity AGNs with spectra dominated by low-ionization emission lines. The physical origin of their ionization mechanism is still debated, and in some cases, they may not be powered by accretion at all\parencite{netzer2013agn}.

\subsection{Unification Model}
\label{sec:unification_model}
Figure \ref{fig:agn_sed} shows the unification model of an AGN. 
As illustrated an AGN is powered by a supermassive black hole surrounded by several distinct regions. Closest to the black hole is the accretion disc, whose hot, optically thick gas emits the thermal “Big Blue Bump” in the optical/UV bands \parencite{peterson1997introduction}. Encircling the disc is the Broad-Line Region (BLR), a compact area of dense clouds orbiting at thousands of kilometers per second, which produces the broad emission lines. Outside the BLR lies the dusty torus, a toroidal structure of cooler gas and dust that can obscure the inner regions when viewed edge-on \parencite{antonucci1993unified}. Beyond the torus, the more extended Narrow-Line Region (NLR) emits narrower lines from slower gas at distances of hundreds of parsecs. In radio-loud AGN, powerful relativistic jets emerge perpendicular to the disc plane, accelerating particles to near-light speeds and generating strong radio emission \parencite{urry1995unified}.


\begin{figure}[!ht]
	\centering
	\includegraphics[width=\textwidth]{pictures/Chapter2/AGN_unified_model.jpg}
	\caption{Unification model of an AGN \parencite{fermi2025figure1}.}
	\label{fig:agn_sed}
\end{figure}



\subsection{Spectral Features}
\label{sec:spectral_features}

AGNs emit radiation across the entire electromagnetic spectrum. A typical AGN exhibits strong X-ray and radio emission, as well as non-stellar ultraviolet and optical continua, along with both broad and narrow emission lines. However, not every AGN displays all of these features, as will be discussed in more detail later \parencite{peterson1997introduction}.
\subsubsection{Non-stellar Continuum Emission}
AGNs typically have a strong, non-stellar continuum emission. In the optical and ultraviolet range, the spectrum often follows a power-law shape instead of a blackbody curve, which implies a non-thermal origin. A noticeable feature in this part of the spectrum is the so-called \emph{big blue bump}, a broad increase in emission that peaks in the ultraviolet between about 1000 and 3000\,\AA. It is thought to be caused by thermal radiation from the hot inner part of the accretion disk around the supermassive black hole \parencite{peterson1997introduction, osterbrock1989agn}. The shape and strength of this bump can give information about how fast matter is falling into the black hole. \\
Additionally, the ionized photons, emitted by the continuum, are responsible for exciting the gas clouds around the nucleus, which leads to emission lines in AGN spectra \parencite{osterbrock1989agn}.

\subsubsection{Broad Emission Lines}


\subsubsection{Narrow Emission Lines}


\subsubsection{Infrared Emission}


\subsubsection{X-ray Emission}


\subsubsection{Radio Emission}




\subsection{Variability}
\label{sec:variability}

\section{Reverberation Mapping}
\label{sec:reverberation_mapping}


\subsection{Principle}
\label{subsec:rm_principle}



The main focus of this work was to perform a classic reverberation analysis of NGC 4593, with a focus on the broad line region (BLR) and its geometry around the central supermassive black hole (SMBH).\\\\
This type of analysis aims to measure the time lag $\tau$ between the variable continuum and the emission line response, in order to determine the spatial scale and structure of the BLR. By observing these variations over time and analyzing the delayed response of the broad lines, it is possible to learn more about the geometry and dynamics of the BLR and to estimate the mass of the SMBH.\\\\
Reverberation mapping (RM) is based on the strong correlation between a variable continuum emission $C(t)$ and the emission line flux $L(\nu, t)$ \parencite{horne2021space}. This correlation originates from the photoionization of gas clouds in the BLR by the central continuum source. As the continuum changes, the emission lines react in a similar way, but with a time delay $\tau$, because of the distance between the central source and the BLR. This delay corresponds to the time it takes for light to travel from the central source to the BLR.\\\\

\subsection{Transferfunction}
\label{subsec:rm_transferfunction}

\subsection{Cross-Correlation Function}
\label{subsec:rm_ccf}

\subsection{Black-Hole Mass}
\label{subsec:rm_bh_mass}

\section{Bowen Fluorescence}
\label{sec:bowen_fluorescence}