\chapter{Scientific Background}
\label{chap:scientific_background}

\section{Active Galactic Nuclei}
\label{sec:agn}

Active Galactic Nuclei (AGN) refer to the central region of active galaxies. Those objects are among the most luminous objects in the universe, with bolometric luminosities ranging from $10^{41}$ to $10^{48} \ \mathrm{erg \ s^{-1}}$, outshining entire galaxies by several orders of magnitude \parencite{peterson1997introduction}. 
Over the years, several stellar-powered models were proposed, such as dense star clusters or supermassive stars. However, these scenarios were discarded, as they are expected to collapse into black holes themselves, and they cannot provide the required energy output \parencite{rees1984oldAGN}. Today, it is understood that the enormous luminosities of AGN are powered by accretion of matter onto a supermassive black hole (SMBH) at their centers \parencite{rees1984oldAGN}. The most widely accepted model for this accretion is a hot, rotating accretion disk surrounding the SMBH, which produces most of the observed radiation \parencite{shakura1973black}.
The following sections will outline the key components of an AGN, introduce the unification model that connects various AGN types, and summarize common classification schemes. A special focus will be on AGN variability, which plays a central role in the reverberation mapping analysis conducted in this thesis.



\begin{figure}[!ht]
	\centering
	\includegraphics[width=0.75\textwidth]{pictures/Chapter2/AGN_standard_paradigm.png}
	\caption{Different components of an AGN. Adopted from \parencite{mo2010galaxy} Figure 14.3.}
	\label{fig:agn_structure_mo}
\end{figure}


\subsection{Structure and Spectral Features of an AGN}
\label{sec:agn_structure}

Figure \ref{fig:agn_structure_mo} shows a schematic structure of an AGN, consisting of a central supermassive black hole (SMBH), a surrounding accretion disk, a dusty torus, and ionized gas regions known as the broad-line region (BLR) and narrow-line region (NLR). In some cases, relativistic jets are launched perpendicular to the plane of the accretion disk \parencite{urry1995unified}. The following subsections describe the intrinsic structures of AGN and the spectral features associated with them.


\subsubsection{Supermassive Black Hole and Accretion Disk}

The center of an AGN is formed by a supermassive black hole (SMBH), with typical masses between $10^6\,M_\odot$ and $10^{9}\,M_\odot$ \parencite{peterson2004}. It does not contribute to the AGN spectrum by itself, but acts as the central engine for all observed spectral features of the AGN. It dominates the gravitational potential and other then inactive Galaxies, like the milky-way, it is surrounded by an accretion disk. Due to viscosity processes within the disk, like turbulent friction or magneto-rotational instability, the angular momentum of the matter is getting transported further out of the disk, which leads to a spiraling matter flow inwards the SMBH \parencite{shakura1973black}. 
Several models have been proposed to describe the accretion process. The most widely used one is the geometrically thin and optically thick accretion disk, consisting of ionized gas in differential rotation around the SMBH \parencite{netzer2013agn}. The disk is composed mainly of ionized hydrogen and helium, with traces of heavier elements \parencite{netzer2013agn}. It extends from the innermost stable circular orbit (ISCO) near the event horizon out to distances of several light-days. The radial extent of the disk is relatively small compared to  galactic scales and typical ranges from a few light-hours to a few light-days, corresponding to about $10^{-3}$ to $10^{-2}$\,pc \parencite{shakura1973black, netzer2013agn}.  \\
During the accretion process a significant fraction of the gravitational energy of the matter is transformed into thermal radiation, which accounts for the enormous luminosity observed in AGNs and heats the accretion disk up to very high temperatures depending on the size of the SMBH \parencite{netzer2013agn}. As an example, the maximum effective temperature for an accretion disk around a SMBH with $M = 10^8,M_\odot$ is on the order of several $\times 10^5$ K, leading to UV and optical emission \parencite{shakura1973black, netzer2013agn}. In comparison, disks around stellar-mass black holes reach much higher temperatures (up to a few $\times 10^6$ K), emitting mostly in X-rays \parencite{shakura1973black, netzer2013agn}. Due to the radial temperature gradient, the emitted spectrum cannot be described as a single black body. Instead, it results from a combination of many black-body-like components at different temperatures, often referred to as a multi-color black-body \parencite{netzer2013agn}. This produces a broad optical–UV continuum of ionizing photons, which interact with gas clouds near the nucleus and play a crucial role in shaping the spectral features of the BLR and NLR. These photons cause photoionization followed by recombination, which leads to to the strong emission lines that are characteristic of AGN spectra \parencite{netzer2013agn}.


\subsubsection{Broad-Line and Narrow-Line Region}
\label{sec:BLRNLR}

The ionized gas clouds near the nucleus can be divided into the broad-line region (BLR) and the narrow-line region (NLR). Both regions differ in density, distance from the SMBH, and observed line widths \parencite{urry1995unified}. The BLR is located close to the nucleus at distances of a few light-days to a few light-years from the central SMBH \parencite{goadBroadLine}(see Figure \ref{fig:agn_structure_mo}). It consists of dense gas clouds with electron densities of $n_e \approx 10^{11}\,\mathrm{cm^{-3}}$, moving at high velocities of several thousand $\mathrm{km\,s^{-1}}$ due to the strong gravitational influence of the SMBH. These velocities lead to a significant Doppler broadening of permitted emission lines with widths of $(500$–$10{,}000)\,\mathrm{km\,s^{-1}}$ \parencite{peterson1997introduction}.
As described earlier, the BLR is photo-ionized by the continuum radiation emitted from the accretion disk. Consequently, the line emission from this region responds to changes in the continuum, leading to a strong correlation between both and shows strong variation \parencite{netzer2013agn}. This relationship is particularly relevant for reverberation mapping, which will be discussed later in Section \ref{sec:reverberation_mapping}.\\ To model the geometry of the BLR is a challenging task, as several emission lines have to be considered, which vary in their intensities due to changes in the variable continuum radiation \parencite{netzer2013agn}. A common model assumes a spherical distribution of clouds that is connected to the accretion disk and located between the accretion disk and the dusty torus  \parencite{goadBroadLine}. Broad emission lines appear in permitted transitions such as H$\alpha$, H$\beta$ and Ly$\alpha$. \parencite{netzer2013agn}
\\\\
Up to several hundred parsecs from the central region lies the narrow-line region (NLR) \parencite{peterson1997introduction}. The gas in this region moves at much lower velocities, resulting in emission lines with widths typically in the order of $(350 - 400)\ \mathrm{km \ s^{-1}}$\parencite{peterson1997introduction}. In contrast to the BLR, the NLR allows both permitted and forbidden transitions. Forbidden lines, such as [O III] $\lambda5007$, arise because collisional de-excitation is inefficient at the relatively low densities of the NLR ($n_e \sim 10^2$–$10^4 \mathrm{cm^{-3}}$) \parencite{peterson1997introduction}. Due to its much larger extends, compared to the BLR, the NLR responds only very slowly to variations in the ionizing continuum. Following that, the flux of the narrow emission lines can be seen as constant over the time scales of several years \parencite{peterson1993}.
Therefore, narrow lines can be used to intercalibrate the spectra of the various observations employed in this thesis. In this work, the narrow [O III] $\lambda5007$ line was used for intercalibration (see Chapter \ref{campaign_and_analysis}).




\subsubsection{Dusty Torus}

Surrounding the accretion disk and broad-line region is the dusty torus, a geometrically thick and optically dense structure composed of gas and dust. It extends from a radius, where dust can survive the intense radiation of the accretion disk, out to scales of a few parsecs \parencite{netzer2013agn}. The torus likely has a clumpy distribution and plays a crucial role in the unified model of AGNs which will be discussed in a later section \parencite{urry1995unified, netzer2013agn}.
The dust in the torus absorbs a significant fraction of the UV and optical radiation emitted by the accretion disk and re-emits it thermally in the infrared. As a result, AGNs typically exhibit strong infrared emission, with the peak wavelength depending on the temperature of the dust in the torus \parencite{netzer2013agn}.
Even in cases where the central region is hidden from direct view by the torus, this reprocessed infrared emission remains observable. It therefore provides a characteristic signature of obscured AGN activity and enables indirect constraints on the otherwise hidden central region \parencite{netzer2013agn}.


\begin{figure}[!ht]
	\centering
	\includegraphics[width=0.75\textwidth]{pictures/Chapter2/Syefert1vsSeyfer2}
	\caption{An example of Seyfert I and Seyfert II spectra illustrating their differences. Broad lines, such as the highlighted $H\alpha$ and $H\beta$, are only present in the Seyfert I spectrum, whereas forbidden [O III] lines are visible in both cases. Adopted from \parencite{runco2015frequency}.  \textbf{Ursprüngliche quelle nicht nachvollziebar. BRauch etwas anderes}}
	
	\label{fig:Seyfert1vsSeyfert2}
\end{figure}




\subsection{Classification}
\label{sec:classification}

AGNs get classified in subgroups based on their spectral features, which are strongly dependent to their intrinsic structure. The key parameters for this classification are luminosity, emission-line profiles and radio properties. Based on those parameters AGN get grouped into Seyfert galaxies, quasars and radio galaxies. They get further subdivided based on the appearance of broad and narrow emission lines. Some examples for these sub-classes are narrow-line Seyfert I galaxies (NLS1s), low-ionization nuclear emission-line regions (LINERs), and jet-dominated sources such as BL Lac objects or blazars \parencite{antonucci1993unified, urry1995unified}. 


\subsubsection{Seyfert Galaxies}

Seyfert galaxies are named after Carl K. Seyfert, who in 1943 observed spiral galaxies characterized by exceptionally bright nuclei and strong emission lines in their optical spectra \parencite{seyfert1943nuclear}. They are mainly classified into the sub-classes Seyfert I and Seyfert II based on the presence of broad emission lines. Figure \ref{fig:Seyfert1vsSeyfert2} highlights the differences of the spectra of Type I and Type II Seyfert galaxies.\\
Seyfert I galaxies, such as NGC 4593, show both broad and narrow emission lines in their optical spectra. The broad lines, such as $H\alpha$ and $H\beta$, typically have full widths at half maximum (FWHM) of several thousand kilometers per second and from the fast-moving, high-density gas in the BLR \parencite{peterson1997introduction}. In contrast, narrow lines, including prominent forbidden transitions like [O\,\textsc{iii}] $\lambda5007$ or [N\,\textsc{ii}] $\lambda6584$, originate from the slow-moving, low-density gas in the NLR \parencite{ peterson1997introduction}. The presence of both components in the spectrum allows for a clear classification as a Seyfert I galaxy, which is the case for NGC 4593. Further details on NGC 4593 are provided in Section \ref{NGC4593}. Between the two main Seyfert classes, several intermediate subclasses (1.2, 1.5, 1.8, 1.9) are defined based on the ratio of the broad towards the narrow components in the optical spectrum \parencite{osterbrock1977SeyfertSub, osterbrock1981SeyfertSub, peterson1997introduction}. Seyfert 1.8 and 1.9 galaxies show very weak broad components. In Seyfert 1.9 objects, the broad component is visible only in the H$\alpha$ line, whereas in Seyfert 1.8 objects it is also detectable in H$\beta$. Furthermore, if the broad and narrow components in H$\beta$ are of equal strength, the AGN is classified as a Seyfert 1.5 \parencite{peterson1997introduction}. If the narrow component is even weaker than the broad component, it is classified as a Seyfert 1.2 \parencite{osterbrock1977SeyfertSub}. The fact that the optical spectrum shows multi-component lines with both broad and narrow components, suggests that these emission lines originate in the BLR and the NLR in the respective ratio \parencite{peterson1997introduction}. \\
In comparison, Seyfert II galaxies completely lack these broad components in their optical spectra, likely due to orientation-dependent obscuration by the dusty torus. Following that the classification of a Seyfert galaxies strongly depends on the viewing angle of the observer, which is the key point for the Unified Model of AGN, which will deepened in section \ref{sec:unification_model} \parencite{peterson1997introduction}.\\  Another notable subclass is the group of so-called narrow-line Seyfert I galaxies (NLS1s). They show most of the features of Seyfert 1 or 1.5 galaxies, except that the usually broad lines, such as the H I or He I lines, exhibit FWHM values that are only slightly larger than those of the narrow lines.  They show a wide dispersion of spectral properties, with some objects have very strong Fe II emission, whereas others show almost none. This indicates that NLS1s do not form a homogeneous class \parencite{osterbrock1985nls1}. NLSls are thought to have low-mass black holes accreting at high Eddington rates, suggesting they may
represent a young evolutionary phase of AGN activity\parencite{peterson2011massesblackholesactive, netzer2013agn}. Another possible explanation is an orientation effect. Another possible explanation is an orientation effect. When an NLS1 is observed at a very low inclination, the projected velocities are reduced, which leads to smaller observed Doppler broadening and therefore to narrow lines \parencite{osterbrock1985nls1}.

\subsubsection{Additional AGN Classes}

In addition to Seyfert galaxies, there are several other classes of AGN. Quasars, which stands for quasi-stellar radio sources, are even more luminous than Seyfert galaxies and are typically found at higher redshifts. While the host galaxies of Seyfert galaxies are still observable, quasars completely outshine their host galaxies. Since quasars show similar emission characteristics to Seyfert galaxies, the modern distinction is based mainly on luminosity: quasars are classified as high-luminosity AGNs, while Seyfert galaxies represent the lower-luminosity end \parencite{netzer2013agn}.\\
Radio galaxies form another important AGN class, distinguished by their strong radio emission and prominent jets, typically found in elliptical host galaxies. When their jets are aligned close to our line of sight, they are observed as blazars or BL Lac objects, which exhibit rapid variability and featureless optical spectra due to relativistic beaming \parencite{netzer2013agn}.\\
Finally, LINERs are low-luminosity AGNs with spectra dominated by low-ionization emission lines. The physical origin of their ionization mechanism is still debated, and in some cases, they may not be powered by accretion at all \parencite{netzer2013agn}.\\\\
While these classifications are based primarily on spectral characteristics, many of the observed differences between AGN types can be attributed to orientation effects. The Unified Model of AGN provides a framework that explains this apparent diversity through a common internal structure, viewed from different angles.

\subsection{Unification Model}
\label{sec:unification_model}

Figure \ref{fig:agn_sed} shows an illustration of the Unification Model, which was postulated by Robert Antonucci in 1993. He proposed that the visible differences in AGN spectra are not due to fundamentally different structures. Instead, they arise mainly from the viewing angle toward the AGN center and from obscuration by the dusty torus \parencite{antonucci1993unified}.\\
The figure shows with what type the same AGN would get classified depending on the observers viewing angle. Like mentioned before, the dusty torus plays a key role here, as it surrounds the central region of the AGN, the accretion disk and the fast-moving BLR. If the observer's line of sight is blocked by the torus, only radio emission, the optical/UV continuum and narrow-line emission from the NLR outside the torus can be detected. In this case, the AGN is classified as a Seyfert 2 galaxy, as the broad emission lines originating from the BLR are obscured and the optical/UV continuum from the accretion disk is only partially visible. The observer essentially views the AGN from a flat angle, looking directly at the torus.\\
If, on the other hand, the observer has a direct view into the central region of the AGN, not obscured by the torus, the fast moving gas clouds of the BLR as well as the optical/UV emission continuum from the accretion disk become visible. In this case, both broad and narrow emission lines are visible, meaning the AGN is classified as a Seyfert 1 galaxy. \parencite{urry1995unified}.\\
The same principle applies to other AGN classes. Quasars can be considered the high-luminosity counterparts of Seyfert galaxies, where orientation and torus obscuration likewise affect their observed properties. Blazars, on the other hand, are seen when the relativistic jet is aligned closely with the observer’s line of sight, leading to strong Doppler boosting, which makes the radiation appear significantly brighter and shifted to higher frequencies than it intrinsically is \parencite{urry1995unified}.\\
Although the classical Unification Model treats AGN classification as fixed and purely geometry-driven, some AGNs have been observed to change their spectral type over time \parencite{ricci2022changinglook}. These so-called "changing-look AGNs" demonstrate that a purely orientation-based interpretation, such as the Unification Model, cannot explain all observed phenomena. They suggest that intrinsic changes, such as variations in accretion rate or obscuring material, can also affect the classification \parencite{ricci2022changinglook}.




%\begin{figure}[!ht]
%	\centering
%	\includegraphics[width=0.9\textwidth]{pictures/Chapter2/AGN_unified_model.jpg}
%	\caption{Unification model of an AGN \parencite{fermi2025figure1}.}
	%\label{fig:agn_sed}
%\end{figure}

%\begin{figure}[!ht]
%	\centering
%	\includegraphics[width=0.9\textwidth]{pictures/Chapter2/AGN_unified_model_2.png}
%	\caption{This graphic shows a schematic of the unification model of an AGN. The figure was adopted from \parencite{collmar2001agn} and was originally adapted from \parencite{urry1995unified}.}
	%\label{fig:agn_sed}
%\end{figure}


\begin{figure}[!ht]
	\centering
	\includegraphics[width=0.9\textwidth]{pictures/Chapter2/AGN_unified_model_3.png}
	\caption{This graphic shows a schematic of the unification model of an AGN. The figure was adopted from \parencite{beckmann_unified}.}
	\label{fig:agn_sed}
\end{figure}


\subsection{Variability}
\label{sec:variability}


The variability of active galactic nuclei (AGN) is one of the key aspects that enables the study of their central regions, which generally cannot be probed with classical spatially resolved methods. Variability is observed on timescales from hours to several years and generally shows stochastic behavior, resulting in flux variations of both emission lines and continuum emission of up to a few tens of percent in the UV and optical regimes \parencite{Ulrich1997, ochmann2024transient}. While the origin of this variability is not yet fully understood, the most widely accepted models attribute it to inhomogeneities and instabilities within the accretion disk \parencite{Ulrich1997, Dexter_2010}. \\
Depending on the physical process, the variations occur on different characteristic timescales. Processes such as thermal fluctuations or changes in the accretion flow act on timescales of decades to centuries for typical SMBH masses and radii, and are therefore difficult to observe directly. In contrast, processes operating on shorter timescales are much easier to study. Examples of such processes are gas motions and mechanical instabilities, such as sound waves, within the disk, which occur on timescales of weeks to months \parencite{ricci2022changinglook}. 
The shortest timescale is the light-crossing timescale, $t_\mathrm{lc} = R/c$, which specifies how long light takes to traverse the variable emitting region, such as the broad-line region (BLR) \parencite{ricci2022changinglook}. Here, $c$ denotes the speed of light and $R$ describes the characteristic size or radius of the variable emitting region. Following \cite{ricci2022changinglook}, by assuming a SMBH of mass $\approx 10^8\,M_\odot$, the light-crossing timescale can be written as
\begin{equation}
	t_\mathrm{lc} = \frac{R}{c} \simeq 0.87 \left(\frac{R}{150\,r_g}\right)M_8\,\mathrm{days},
\end{equation}
where $r_g = GM/c^2$ denotes the gravitational radius of the black hole. It follows that the light-crossing timescale of the variable emitting regions is on the order of days, and that $t_\mathrm{lc}$ scales linearly with the size of the emitting region \parencite{ricci2022changinglook}.\\
Because variations in the ionizing continuum occur on such short timescales, it is possible to measure delayed responses from other correlated regions within the AGN using long-term monitoring campaigns \parencite{peterson1997introduction}. In particular, the BLR responds to changes in the photoionizing continuum radiation of the central source with a time delay or lag that is bigger than the light-crossing timescale of the emitting region \parencite{peterson1997introduction}. This time lag forms the basis of a classical reverberation mapping analysis, which will be further elaborated in the next section.




\section{Reverberation Mapping}
\label{sec:reverberation_mapping}

The main focus of this work is a classic reverberation mapping analysis of the broad lines of NGC 4593. This observational technique allows to probe the structure of the BLR around the SMBH inside the AGN. This technique bases on the time delay or time lag between the continuum's variation and the correlated response of the broad lines. With this calculable time lag it is possible to infer the geometry of the BLR and to calculate the mass of the SMBH \parencite{cackett2018accretion}.  


\subsection{Principle of Reverberation Mapping}
\label{subsec:rm_principle}

The fundamental assumption in reverberation mapping is that variation in the observed continuum's flux corresponds to the variation in the emission-line flux, with a measurable time delay. With variation in the continuum luminosity, the emission-line response follows these continuum variations with a measurable time lag \parencite{Cackett2021}.
\begin{figure}[!ht]
	\centering
	\includegraphics[width=0.7\textwidth]{pictures/Chapter2/iso-delay_surface}
	\caption{Spherical distribution model of the broad line region and isodelay surface, adopted from \parencite{peterson1997introduction}.}
	\label{fig:iso-delay}
\end{figure}

This time lag $\tau$ is corresponding to the average light-travel time from the photoionization continuum to the line emitting regions. Taking the idealized model for the BLR as spherical distributed clouds \parencite{goadBroadLine}, then $\tau$ can be assumed as \parencite{peterson1997introduction}
\begin{equation}
	\tau = \left(1+\cos\theta\right)\cdot \frac{R_\mathrm{BLR}}{c},
\end{equation}
where $R_\mathrm{BLR}$ is the characteristic size or radius of the BLR, c the speed of light and $\theta$ the angle of the BLR from the line of sight of the observer to the central source  (\ref{fig:iso-delay}) \parencite{peterson1997introduction}. The circle in Figure \ref{fig:iso-delay} describes the spherical distribution of the BLR. For a fixed lag $\tau$ the location for the responding emitting region must lay on a paraboloid, that fits to the observers line of sight to the continuum, which is called the isodelay surface. Following that the position of the BLR, that is emitting with the fixed lag $\tau$ has to be located on the intersection of the BLR distribution and the isodelay surface \parencite{peterson1997introduction}. Thus, reverberation mapping can be used to 'map' the BLR’s structure by infer the characteristic radius of the BLR from the time lag \parencite{peterson1997introduction}. But because the observer sees not one, but all possible isodelay surfaces that a possible for a fixed lag $\tau$, the so called 'transfer equation' is needed which integrates over all isodelay surfaces \parencite{peterson1997introduction}:
\begin{equation}
	\label{eqn:transfer-function}
	L\left(t\right) = \int \Psi\left(\tau\right)C\left(t-\tau\right) d\tau
\end{equation}
$\Psi\left(\tau\right)$ refers to the transfer function, delivering the BLR's geometric and kinematic information, $C(t)$ refers to the contiuuum light curve and $L(t)$ the emission light curve \parencite{peterson1997introduction}.
While in theory the response of the BLR can be fully described by the transfer function $\Psi\left(\tau\right)$, this lag can be estimated using cross-correlation techniques. In practice, recovering the full transfer function $\Psi\left(\tau\right)$ would require very well-sampled and high signal-to-noise light curves over a duration much longer than the expected lag. Since real monitoring campaigns are often affected by observational gaps and noise, such reconstructions are rarely possible \parencite{horne2004observational,peterson1993}. For this reason, this thesis concentrates on measuring the mean time lag between continuum and emission-line variations using the interpolated cross-correlation function (ICCF) method \parencite{gaskell_peterson1986}.\\


\subsection{Lag Measurement}
\label{subsec:rm_ccf}

Using the notation from \cite{peterson1997introduction} the cross-correlation function between the ionizing continuum and an emission line is defined as \begin{equation}
	F_\mathrm{CCF}(\tau) = \int_{-\infty}^{\infty}L(t)C(t-\tau)dt,
\end{equation}
With the auto-correlation function of the ionizing continuum\begin{equation}
	F_\mathrm{ACF}(\tau) = \int_{-\infty}^{\infty}C(t)C(t-\tau)dt
\end{equation} 
and the transfer equation \ref{eqn:transfer-function}, the cross-correlation function can be written as the convolution of the transferfunction and the auto-correlation function of the ionizing continuum:
\begin{equation}
	F_\mathrm{CCF}(\tau) = \int_{-\infty}^{\infty}\Psi(\tau') F_\mathrm{ACF}(\tau-\tau')d\tau'
\end{equation}
The lag is defined to be at the location of the peak ($\tau_{\mathrm{peak}}$) or the centroid ($\tau_{\mathrm{centroid}}$) of the CCF \parencite{peterson1997introduction}. 
While $\tau_{\mathrm{peak}}$ is defined as the peak of the CCF, $\tau_{\mathrm{centroid}}$ gets calculated over all points of the CCF, which are above a selected threshold which is typically $80\%$. Because of the strong relation of the CCF and the transferfunction, it is possible to infer a scale length for the BLR, where the emission line originated from \parencite{peterson1997introduction}, which can be describe with \begin{equation}
	R_\mathrm{BLR} = c \cdot \tau_{\mathrm{centroid}},
\end{equation}  using the light-travel time scale \parencite{peterson2004}. Since the centroid lag is generally considered a more robust representation of the mean light-travel time of the BLR \parencite{peterson2004}, it is used in this thesis.\\
The uncertainty of the measured lag is estimated using a Monte Carlo approach combining flux randomization (FR) and random subset selection (RSS) \parencite{Peterson_1998b, peterson2004}. In the FR step, each flux value is randomly perturbed according to its measurement uncertainty. In the RSS step, a set of $N$ data points is randomly drawn with replacement, but redundant selections are discarded, resulting in a new light curve with $M \leq N$ points. For each realization, the ICCF analysis is repeated, yielding a distribution of centroid lags. The uncertainties are estimated from the distribution of centroid lags obtained through the simulations\parencite{peterson2004}.The 16th and 84th percentiles of this distribution are adopted as the bounds of the $1\sigma$ confidence interval \parencite{Peterson_1998b}.
\subsection{Black-Hole Mass}
\label{subsec:BHM}

The reverberation mapping methode can further be used to calculate the mass of the central SMBH. Having the centroid lag of broad emission lines, it can be determined together with velocity dispersion $\Delta V$ of the BLR:

\begin{equation}
	M_\mathrm{BH}= \frac{fc\tau_\mathrm{centroid}\Delta V^2}{G}
\end{equation}
Here, $G$ describes the gravitational constant and $f$ a scale factor which accounts for the unkonw geometry, kinematics and inclination of the BLR \parencite{peterson2004}.
\\\\

With the measured centroid lag $\tau_{\mathrm{centroid}}$ from the ICCF analysis and the velocity width of the broad emission line, the mass of the central supermassive black hole (SMBH) can be estimated under the assumption that the BLR gas is gravitationally bound and its motions are dominated by the SMBH potential \parencite{peterson2004}. The distance to the BLR is given by

\begin{equation}
	R_{\mathrm{BLR}} = c \cdot \tau_{\mathrm{centroid}}
\end{equation}
and, combined with the line-of-sight velocity dispersion $\Delta V$ of the BLR, the virial product is defined as:

\begin{equation}
	M_{\mathrm{vir}} = \frac{R_{\mathrm{BLR}}\,\Delta V^2}{G}.
\end{equation}

The black hole mass is then obtained by applying a scale factor $f$, which accounts for the unknown geometry, kinematics, and inclination of the BLR:
\begin{equation}
	M_{\mathrm{BH}} = f \cdot M_{\mathrm{vir}}.
\end{equation}
Following \textcite{onken2004}, a mean value of $f$ is adopted, calibrated by aligning reverberation-based masses with the $M_{\mathrm{BH}}-\sigma_*$ relation observed in not-active galaxies. Here, $\sigma_*$ describes the stellar velocity dispersion of the galactic bulge. The actual scale factor can still vary between individual AGN because of differences in their geometry and orientation, which results in systematic uncertainties in the SMBH mass estimation. Still, the calibration of $f$ against $M_{\mathrm{BH}}-\sigma_*$ delivers a well-established method for estimation the SMBH mass. With that and the mass of the SMBH is given by
\begin{equation}
	M_\mathrm{BH}= \frac{fc\tau_\mathrm{centroid}\Delta V^2}{G}
\end{equation}

$\Delta V$ can be obtained, by measuring the line widths of the broad emission-lines. Following \cite{peterson2004} 






\section{Bowen Fluorescence}
\label{sec:bowen_fluorescence}

The mechanism known as Bowen Fluorescence was first described by I.S. Bowen in 1934 to explain unexpected emission lines in nebular spectra. This mechanism describes a resonant line pumping process, where Photons emitted by an ion, hits randomly another ion of a different species with a matching permitted transition, and excites it via absorption due to a near-wavelength coincidence. The resulting de-excitation leads to an enhancement of the emission lines \parencite{bowen1934}. \\
One transition that can be enhanced by Bowen fluorescence is the O\textsc{i}$\lambda$8446 emission line, which is pumped by Ly$\beta$ fluorescence \parencite{netzer1976}.  
In this process, Ly$\beta$ photons at $\lambda1025.72$\,\AA{} are absorbed by neutral oxygen 
through the near-resonant transition $2p^4\,^3\!P_2 \rightarrow 3d\,^3\!D^0$ of O\,\textsc{i} 
at $\lambda1025.77$\,\AA. The excited $3d\,^3\!D^0$ level decays to $3p\,^3\!P$, 
which then decays to $ 3s\,^3\!S^\circ$, 
emitting the O\,\textsc{i}\,$\lambda8446$ emission line (see figure \ref{fig:bowen_cascate} ) \parencite{grandi1980}. This adds another possible contributing excitation mechanism to the O\textsc{i}$\lambda$8446 emission line, in addition to recombination.
\textbf{evtl noch erweitern}


\begin{figure}[!ht]
	\centering
	\includegraphics[width=0.7\textwidth]{pictures/Chapter2/OITriplet.jpg}
	\caption{Energy level diagram of the triplet states of OI, adopted from \parencite{grandi1980}.}
	\label{fig:bowen_cascate}
\end{figure}



