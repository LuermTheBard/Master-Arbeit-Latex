\chapter{Reverberation Mapping Analysis of NGC4593}
\label{cap: Results}

\section{Line Identification}

The first step of the reverberation mapping analysis is the identification of emission and absorption lines within the AVG spectrum. Figures  \ref{fig:AVG_RMS_SPECTRUM} and \ref{fig:UV_uncalibrated_AVG_RMS} show the optical to near-infrared range between $3900 \AA$ and $9000 \AA$ and the UV range between $1100 \AA$ and $1700 \AA$, respectively. 
Looking at Figure \ref{fig:AVG_RMS_SPECTRUM}, it is possible to identify prominent broad emission lines in the AVG spectrum are the Balmer-Lines up to H$\epsilon$, with H$\alpha$ being the strongest followed by H$\beta$ and H$\gamma$, and several He-emission lines such as He\,I$\lambda 4471$, He\,II $\lambda 4685$, the He\,I $\lambda 5015$, He\,I $\lambda 5875$, He\,I $\lambda 7065$. Another significant broad emission line complex appears in the NIR part of the spectrum, including the O\textsc{i}$\,\lambda8446$ emission line and the Ca\,\textsc{ii}$\,\lambda8498\,\lambda8542\,\lambda8662$ triplet. All of those lines show significant variation in the RMS, especially the Balmer-lines and the He\,II $\lambda 4685$ and He\,I $\lambda 5875$ with very good distinguishable profiles. The He\,I$\lambda 4471$, the He\,I $\lambda 5015$, the He\,I $\lambda 7065$, the O\textsc{i}$\,\lambda8446$ emission line and the Ca\,\textsc{ii}$\,\lambda8498\,\lambda8542\,\lambda8662$ triplet show smaller but still noticeable variation. The He\,I$\lambda 4471$ and He\,I $\lambda 5875$ lines are blended with a Fe II emission line group in the AVG, which is vanishing in the RMS, as it shows no variation. The same aplies to the Fe II emission line group around 5200 \AA.\\
Apart from the broad emission lines, the AVG spectrum shows strong forbidden [O\,III] narrow emission lines. [O\,III] $\lambda 5007$ and [O\,III] $\lambda 4958$ are strong enough to be distinguishable, while the [O\,III] $\lambda 4363$ blends with H$\gamma$. As narrow lines they and several more weaker narrow lines, show no variability in the RMS spectrum over the period of the campaign. 
\begin{figure}[!htbp]
 	\centering
 	\makebox[\textwidth][c]{%
 		\includegraphics[width=1\textwidth]{pictures/Chapter4/avg_rms_spec/avg_rms_spec.pdf}}
 	\caption{Optical-to-NIR AVG and RMS spectrum with identified emission lines.}
 	\label{fig:AVG_RMS_SPECTRUM}
\end{figure}
\begin{figure}[!htbp]
	\centering
	\makebox[\textwidth][c]{%
		\includegraphics[width=\textwidth]{pictures/Chapter4/avg_rms_spec/UV_uncalibrated_AVG_RMS.pdf}}
	\caption{UV spectrum AVG and RMS spectrum with identified emission lines}
	\label{fig:UV_uncalibrated_AVG_RMS}
\end{figure}\\
The campaign also included an UV spectrum between $1000$ \AA and $1700$ \AA, which is shown in Figure \ref{fig:UV_uncalibrated_AVG_RMS}. There are several broad lines in the UV spectrum, naming the Ly$\alpha$ line, which overlaps with the N\,\textsc{v}$\lambda\lambda 1238,1242$ doublet, the Si\,\textsc{iv}$\lambda\lambda 1393,1402$ doublet, the C\,\textsc{iv}$\lambda\lambda 1548,1550$ doublet and the He\textsc{ii}$\,\lambda1640$ line. Ly$\alpha$ and C\,\textsc{iv}$\lambda\lambda 1548,1550$ show strong variation, while the variation of N\,\textsc{v}$\lambda\lambda 1238,1242$ and He\textsc{ii}$\,\lambda1640$ is weaker but still noticeable. Like in the AVG spectrum, the variation of N\,\textsc{v}$\lambda\lambda 1238,1242$ is overlapping with Ly$\alpha$ emission in the RMS as well, but is still distinguishable. The same applies to the variation of He\textsc{ii}$\,\lambda1640$, which is overlapping with the semi-forbidden line doublet O\textsc{III}]$\,\lambda\lambda\,1660,\,1666$.



\section{Emission Line and Continua Measurement}

After identifying the emission lines, the next step of the analysis is to measure the fluxes of those lines. This is done with the help of a python-based tool called GECHO created by M. Probst. It enables to import full campaigns, determine the AVG and RMS spectra, extracting lightcurves and do further measurements, naming here line-width measurements and lags measurement using the ICCF methode \parencite{gaskell_peterson1986, peterson2004}, which will be further discussed in Section \ref{sec:line_profiles} and \ref{sec:time_lag_bh_mass}.\\
With help of GECHO the flux density gets integrated over the extent of each emission line for every observed spectrum of the campaign. Here it is important to define the integration limits so that all of the emission line flux is measured in this way. To ensure this, a parallel view of the AVG- and RMS-spectrum was used to define those integration limits, and to ensured, that no component of any other line is contributing to the line flux of the measured line. Additionally, the integration boundaries get selected, making sure they span a similar area in velocity space for every emission line. To accounting the surrounding continuum a linear underlying continuum gets interpolated and subtracted first. For this interpolation, sections on the blue and red side of the emission line with no line contribution gets selected. The chosen integration limits and pseudo-continua are listed in Table \ref{tab:emission_lines}. \\
To measure the delay between the ionizing continuum and the emission lines it is necessary to extract lightcurves of selected continua in the AVG spectrum. Continua lightcurves are getting extracted by measuring the mean flux density of a sufficiently large region without line emission or absorption. The chosen integration limits for the continua can be found in Table \ref{tab:continua}.\\

\newpage

\begin{table}[h!]
	\centering
	\small
	\caption{Integration Limits and Pseudo-Continua of the measured emission lines}
	\label{tab:emission_lines}
	\begin{tabular}{lcc}
		\hline
		\hline
		\textbf{Line} & \textbf{Integration Limits $[\AA]$} & \textbf{Pseudo-Continua $[\AA]$}  \\
		\hline
		\hline
		Ly$\alpha$ & $1207-1238$ & $1151-1161, 1461-1469$\\
		NV$\,\lambda\lambda 1238,\,1242$ & $1207-1238$ & $1151-1161, 1461-1468$\\
		SiIV$\,\lambda\lambda 1393,\,1402$ & $1358-1423$ & $1151-1161, 1461-1469$\\
		CIV$\,\lambda\lambda 1548,\,1550$ & $1511-1578$ & $1461-1469, 1680-1685$\\
		HeII$\,\lambda1640$ & $1599-1645$ & $1461-1468, 1680-1685$\\
		\hline
		H$\alpha$ & $6453-6695$ & $6107-6129, 6861-6900$ \\
		H$\beta$ & $4779-4944$ & $4762-4774, 5085-5112$ \\
		H$\gamma$ & $4230-4427$ & $4197-4220, 4435-4450$ \\
		
		HeI$\,\lambda5875$ & $5742-6039$ & $5645-5653, 6044-6057$ \\
		HeII$\,\lambda4685$ & $4545-4758$ & $4435-4450, 4762-4774$ \\
		OI$\,\lambda 8446$ & $8380-8498$ & $8005-8031, 8850-8955$ \\
		\hline
		OIII$\,\lambda 5007$ & $4982-5033$ & $4762-4774, 5085-5112$ \\
		\hline
		\hline
	\end{tabular}
\end{table}
\begin{table}[h!]
	\centering
	\small
	\caption{Integration Limits of the measured continua}
	\label{tab:continua}
	\begin{tabular}{lc}
		\hline
		\hline
		\textbf{Line} & \textbf{Integration Limits $[\AA]$}  \\
		\hline
		\hline
		Cont. 1150 & $1151-1161$\\
		\hline
		Cont. 4010 & $4026-4033$\\
		Cont. 4440 & $4435-4450$\\
		Cont. 5100 & $5085-5112$\\
		Cont. 6110 & $6107-6129$\\
		Cont. 6880 & $6861-6900$\\
		Cont. 8015 & $8005-8031$\\
		Cont. 8900 & $8864-8955$\\
		\hline
		\hline
	\end{tabular}
\end{table}

\subsection{Variability Statistics}

To quantify the variability of the emission lines and continua variability statistics the definition by \cite{rodriguez1997steps} are adopted. They name the  maximum-to-minimum flux ratio $R_\mathrm{max}$ and the fractional variability	$F_\mathrm{var}$ as two common measures of variability. $R_\mathrm{max}$ is defined as the ratio of the the extreme of the integrated fluxes, $F_\mathrm{min}$ and $F_\mathrm{max}$, and  $F_\mathrm{var}$ as: 
\begin{equation}
	F_\mathrm{var}=\frac{\sqrt{\sigma_F^2-\Delta^2}}{\left<F\right>}
\end{equation}
Here, $\sigma_F^2$ denotes the standard deviation, $\left<F\right>$ the mean flux and $\Delta^2$ the mean square value of the flux uncertainties, which is defined by:
\begin{equation}
	\Delta^2 = \frac{1}{N} \sum_{i=1}^{N}\Delta_i^2
\end{equation}
The results for all parameters can be found in the Tables \ref{tab:varstatistics}.\\
First, considering $R_\mathrm{max}$ and $F_\mathrm{var}$ of the continuum lightcurves, all measured continua show similar variability, except for the UV continuum around $1150$ \AA, which shows significantly higher variability with $R_\mathrm{max} = 2.58$ and $F_\mathrm{var} = 0.28$ than the optical and NIR continua. These continua show a fairly uniform variation between $R_\mathrm{max} \simeq 1.33-1.59$ and $F_\mathrm{var} \simeq 0.08-0.14$. \\
The lightcurves of the Balmer emission lines exhibit lower variability, which increases towards the higher-order Balmer lines, with values between $R_\mathrm{max} \simeq 1.15-1.52$ and $F_\mathrm{var} \simeq 0.03-0.1$. The helium lightcurves show a similar variability to H$\delta$ with values  between $R_\mathrm{max} \simeq 1.48-1.62$ and $F_\mathrm{var} \simeq 0.09-0.11$ and the OI$\,\lambda 8446$ a similar variability to the lower-order balmer lines with $R_\mathrm{max} \simeq 1.22$ and $F_\mathrm{var} \simeq 0.04$.\\
The variability of the emission line lightcurves in the UV region is on a similar level as that of the helium light curves in the optical region, with $R_\mathrm{max} \simeq 1.42-1.62$ and $F_\mathrm{var} \simeq 0.14$, with the exception of the HeII$,\lambda1640$ lightcurve, which shows significant higher variability with $R_\mathrm{max} \simeq 3.12$ and $F_\mathrm{var} \simeq 0.31$.

\begin{table}[!htb] 
\centering 
\caption{Variability statistics of the measured continua and emission lines with minimum flux $F_{\text{min}}$ and maximum flux density or integrated flux $F_{\text{max}}$, peak-to-peak ratio $R_{\text{max}}$, mean $\langle F \rangle$, standard deviation $\sigma_F$ and fractional variation $F_{\text{var}}$.} 
\begin{tabular}{lrrrrrr} 
	\hline 
	\hline 
	\textbf{Continuum/Line} &  {$F_{\text{min}}$} &  {$F_{\text{max}}$} &  {$R_{\text{max}}$} &  {$\langle F \rangle$} &  {$\sigma_F$} &  {$F_{\text{var}}$} \\ 
	\hline
	\hline
	Cont. 1150  & $0.52$ & $1.35$ & $2.58$ & $0.86$ & $0.25$ & $0.28$ \\
	Cont. 4010  & $2.68$ & $4.21$ & $1.57$ & $3.49$ & $0.47$ & $0.14$ \\
	Cont. 4440  & $2.42$ & $3.73$ & $1.54$ & $3.14$ & $0.39$ & $0.12$ \\
	Cont. 5600  & $1.36$ & $2.15$ & $1.59$ & $1.82$ & $0.25$ & $0.14$ \\
	Cont. 6110  & $1.49$ & $2.27$ & $1.53$ & $1.9$ & $0.23$ & $0.12$ \\
	Cont. 6880  & $1.33$ & $2.01$ & $1.5$ & $1.72$ & $0.2$ & $0.11$ \\
	Cont. 8015  & $1.18$ & $1.69$ & $1.43$ & $1.48$ & $0.15$ & $0.1$ \\
	Cont. 8900  & $1.14$ & $1.52$ & $1.33$ & $1.38$ & $0.11$ & $0.08$ \\
	\hline 
	H$\alpha$  & $130.83$ & $149.85$ & $1.15$ & $141.35$ & $4.83$ & $0.03$ \\
	H$\beta$  & $38.29$ & $45.4$ & $1.19$ & $42.32$ & $1.94$ & $0.05$ \\
	H$\gamma$  & $18.82$ & $24.38$ & $1.3$ & $21.91$ & $1.29$ & $0.06$ \\
	H$\delta$  & $7.17$ & $10.92$ & $1.52$ & $9.13$ & $0.94$ & $0.1$ \\
	HeII$\,\lambda4685$  & $11.93$ & $17.7$ & $1.48$ & $14.82$ & $1.6$ & $0.11$ \\
	HeI$\,\lambda5875$   & $8.53$ & $13.82$ & $1.62$ & $11.58$ & $1.01$ & $0.09$ \\
	OI$\,\lambda 8446$ & $7.47$ & $9.13$ & $1.22$ & $8.32$ & $0.37$ & $0.04$ \\
	\hline 
	Ly$\alpha$  & $66.87$ & $94.88$ & $1.42$ & $82.21$ & $8.03$ & $0.1$ \\
	SiIV$\,\lambda\lambda 1393,\,1402$  & $21.93$ & $35.5$ & $1.62$ & $27.92$ & $3.88$ & $0.14$ \\
	CIV$\,\lambda\lambda 1548,\,1550$  & $115.31$ & $165.57$ & $1.44$ & $138.77$ & $12.12$ & $0.09$ \\
	HeII$\,\lambda1640$  & $6.83$ & $21.29$ & $3.12$ & $13.29$ & $4.13$ & $0.31$ \\
	\hline 
	\hline 
	\label{tab:varstatistics} 
\end{tabular} 

\end{table}


\subsection{Uncertainties Estimation}
The uncertainties of the continuum and emission-line lightcurves are estimated from two components. The first component originates from the noise in the flux densities over the pixels within the integration bounds, $N_\mathrm{pixel}$. For the continuum lightcurves, the noise uncertainty is estimated as 
\begin{equation}
	\sigma_\mathrm{noise}^{\mathrm{cont}} = \frac{\sigma_F}{\sqrt{N_\mathrm{pixel}^{\mathrm{cont}}}}, 
\end{equation} 
where $\sigma_F$ denotes the standard deviation of the flux density within the selected continuum window. Since the emission-line lightcurves are extracted by integrating the continuum-subtracted flux density, the corresponding noise uncertainty is estimated using the noise measured from the used  pseudo-continua, $\sigma_\mathrm{noise}^{\mathrm{p.cont.}}$:
\begin{equation}
	\sigma_\mathrm{noise}^{\mathrm{lines}} = \frac{\sigma_\mathrm{noise}^{\mathrm{p.cont.}} \cdot \Delta\lambda}{\sqrt{N_\mathrm{pixel}^{\mathrm{line}}}}.
\end{equation}
Here, $\Delta\lambda$ denotes the integration window used for the emission line, and $N_\mathrm{pixel}^{\mathrm{line}}$ denotes the number of pixels within this window.
\\
The second component originates from the intercalibration of the epochs of the campaign. The flux of each epoch is scaled based on the narrow emission line [O\,\textsc{iii}] $\lambda5007$, which is assumed to be constant over the campaign. To account for the systematic uncertainty introduced by the intercalibration, the fractional variability of [O\,\textsc{iii}] $\lambda5007$ is multiplied by the integrated flux values $f_i$  and used as a proxy for the intercalibration uncertainty:
\begin{equation}
	\sigma_i^\mathrm{cal.} = F_\mathrm{var}^{\mathrm{[O\,\textsc{iii}]}\,\lambda5007} \cdot f_i .
\end{equation}
This yields the total light-curve uncertainty:
\begin{equation}
	\sigma_i = \sqrt{\left(\sigma_i^\mathrm{cal.}\right)^2 + \left(\sigma_{\mathrm{noise}}\right)^2} ,
\end{equation}
where $\sigma_{\mathrm{noise}}$ corresponds to $\sigma_\mathrm{noise}^{\mathrm{cont}}$ for continuum light curves and to $\sigma_\mathrm{noise}^{\mathrm{lines}}$ for emission-line light curves.




\begin{figure}[!ht]
	\centering
	\includegraphics[width=1\textwidth]{pictures/Chapter4/lightcurves/Continua.pdf}
	\caption{Comparison of the continua lightcurves. The first panel shows the UVW2 continuum lightcurve obtained from \cite{mchardy2018x}, while the other panels show the measured continua defined in Table \ref{tab:continua} }
	\label{fig:continua_lightcurves}
\end{figure}
\clearpage
\section{Lightcurves}
\label{sec:lightcurves}
The extracted lightcurves of the continua and emission lines are shown in Figures \ref{fig:continua_lightcurves}, \ref{fig:emission_line_lightcurves}, and \ref{fig:UV_emission_line_lightcurves}. In addition to the measured lightcurves of the HST campaign, the UVOT UVW2 lightcurve taken with \textit{Swift} by \cite{mchardy2018x}, which was also used in \cite{cackett2018accretion}, is included in this analysis, as it has a higher sampling rate than the light curves of the HST campaign. The spectra were taken during nearly every orbit for $6.4\,\mathrm{d}$ between July 13 and July 18, 2016 \parencite{mchardy2018x}. The central wavelength of the UVW2 lightcurve is located at about $1930\,\AA$ \parencite{mchardy2018x}. \\
When compared, all measured continuum lightcurves show a roughly similar shape (see Figure \ref{fig:continua_lightcurves}). The lightcurves start at a high flux level, followed by a significant decrease of about $50$--$80\%$ relative to the maximum-to-minimum range of each curve approximately between the 1st and 6th day, with the minimum at around the 4th day. A central region follows, with again high flux with variation depending on the wavelength range of the continua. For the UV continua, this plateau ranges from the 6th to the 9th day and is followed by another smaller decrease in flux between the 9th and 13th day, before it increases again with peak at around the 13th day. Subsequently, the flux decreases, with small interim variation, towards a minimum at the end of the campaign. The continua between $4010\,\AA$ and $6110\,\AA$ also show a plateau-like shape between the 6th and 14th day, except for a small increase in flux between the 7th and 10th day, before decreasing again. The red and NIR continua show a roughly constant plateau between the 8th and 14th day, before decreasing in the same way as the other light curves, with another small, more pronounced flux plateau between the 15th and 17th day. \\
One goal of this reverberation mapping analysis is to estimate the physical distance between the SMBH and the region from which the emission lines originate. Assuming that the continuum radiation originates from the accretion disk, it is common to use the bluest available continuum as reference for the time lag estimation, which is expected to originate closest to the SMBH \parencite{ochmann2026first}. In this case, this would be the continuum around $1150\,\AA$. Nevertheless, the UVW2 continuum was selected as the main reference lightcurve for the subsequent analysis due to its higher sampling rate. To accommodate this, the delay between the UV continuum lightcurve around $1150\,\AA$ and the UVW2 lightcurve has to be taken into account.
\begin{figure}[!ht]
	\centering
	\includegraphics[width=\textwidth]{pictures/Chapter4/lightcurves/Balmer_Lyman_and_O_lines.pdf}
	\caption{Comparison of the emission-line lightcurves to the UVW2 reference lightcurve in the first panel. The UVW2 lightcurve was obtained from \cite{mchardy2018x}}
	\label{fig:emission_line_lightcurves}
\end{figure}
\clearpage
\begin{figure}[!ht]
\centering
\includegraphics[width=\textwidth]{pictures/Chapter4/lightcurves/He_and_UV_lines.pdf}
\caption{Comparison of the emission-line lightcurves to the UVW2 reference lightcurve in the first panel. The UVW2 lightcurve was obtained from \cite{mchardy2018x}}
\label{fig:UV_emission_line_lightcurves}
\end{figure}
\clearpage
Figures \ref{fig:emission_line_lightcurves} and \ref{fig:UV_emission_line_lightcurves} show the lightcurves of the measured broad emission lines in the optical-to-NIR range and in the UV range. Overall, the light curves show a similar shape to the UVW2 lightcurve: a high or increasing flux in the first few days, a pronounced minimum, a plateau-like central part that is fluctuating in some cases, and a decrease towards the end of the campaign. The main exceptions are the OI$\,\lambda 8446$ and He\,\textsc{ii}$\,\lambda1640$ lightcurves, which are much more scattered than the others. Here, it has to be noted, that the integration boundaries of He\,\textsc{ii}$\,\lambda1640$ did not include parts of its red flank, as it is blended with the semi-forbidden emission line doublet O\,\textsc{iii}]$\,\lambda\lambda 1660,\,1666$. Nevertheless, it is possible to notice a pronounced minimum similar to that in the UVW2 light curve in these curves. By comparing these features, a shift of a few days between the light curves is already apparent. The flux minima of the Balmer lines, OI$\,\lambda 8446$, and HeI$\,\lambda5875$ occur between days 6 and 7, whereas the minima of the HeII light curves and the UV emission lines Ly$\alpha$, SiIV$\,\lambda\lambda 1393,\,1402$, and CIV$\,\lambda\lambda 1548,\,1550$ occur between days 4 and 5. This provides a first estimate of a time lag of about $2$--$3$ days for the first group of emission lines and about $0$--$1$ days for the second group. A more detailed investigation of the time lags is presented in Section \ref{sec:time_lag_bh_mass}. 





\section{Line Profiles}
\label{sec:line_profiles}

The next step of the analysis is the extraction of line profiles of the broad emission lines from both, the AVG and RMS spectrum. By investigation their shape and line width it allows conclusions about the kinematics of the region their originated from. Following the same procedure as for the lightcurve extraction, an underlying linear continuum is interpolated using a chosen blue and red pseudo-continuum for each line in both the AVG and RMS spectra. The selected boundaries of the pseudo-continua are listed in Table \ref{tab:line_profiles_pseudo}. By subtracting this interpolated continuum, a new zero-flux baseline for each line profile gets defined, which allows a better comparison between the line profiles. Subsequently, the line profile is converted to velocity space using\\ 
\begin{equation}
	\label{eqn:velocity_space}
	v_i = c \cdot \frac{\lambda_i^2 - \lambda_\mathrm{central}^2}{\lambda_i^2 + \lambda_\mathrm{central}^2}.
\end{equation}\\
Here, $\lambda_i$ denotes the wavelength values, $\lambda_\mathrm{central}$ the central wavelength of the emission line in rest-frame and $c$ the speed of light. Finally, the flux is getting normalized to the maximum of each line profile. Figure \ref{fig:line_profiles} shows an overview of the extracted and normalized line profiles. For the doublet the rest-frame wavelength of the second line gets used as the central wavelength of the emission line.

\begin{table}[!ht]
	\centering
	\small
	\caption{Boundaries of the blue and red pseudo-continua used for the interpolation of underlying continua for line profile extraction.}
	\label{tab:line_profiles_pseudo}
	\begin{tabular}{lcc}
		\hline
		\hline
		\textbf{Line} & \textbf{\textbf{Pseudo-Continua (avg) $[\AA]$}} & \textbf{Pseudo-Continua (rms) $[\AA]$}  \\
		\hline
		\hline
		H$\alpha$ & $6194-6216, 6861-6900$ & $6279-6301, 6742-6781$ \\
		H$\beta$ & $4762-4774, 5085-5112$ & $4762-4774, 4967-4984$ \\
		H$\gamma$ & $4197-4220, 4435-4450$ & $4197-4220, 4417-4429$ \\
		H$\delta$ & $4026-4033, 4197-4221$ &$4006-4016, 4197-4211$  \\
		\hline
		O\,\textsc{i}$\,\lambda 8446$ & $7999-8025, 8775-8798$ & $8222-8238, 8748-8767$ \\
		\hline
		He\,\textsc{i}$\,\lambda5875$ & $5679-5697, 6044-6057$ & $5736-5753, 6027-6045$ \\
		He\,\textsc{ii}$\,\lambda1640$  & $1461-1468, 1679-1685$ & $1461-1468, 1679-1685$\\
		He\,\textsc{ii}$\,\lambda4685$ & $4198-4225, 4762-4774$ & $4543-4554, 4766-4778$ \\
		\hline
		Ly$\alpha$ & $1151-1161, 1270-1285$  &  $1151-1161, 1340-1355$\\
		N\,\textsc{v}$\,\lambda\lambda 1283,\,1242 $ & $1151-1161, 1270-1285$  &  $1151-1161, 1340-1355$\\
		Si\,\textsc{iv}$\,\lambda\lambda 1393,\,1402 $ & $1350-1360, 1430-1440$ &  $1340-1355, 1430-1440$\\
		C\,\textsc{iv}$\,\lambda\lambda 1548,\,1550$ & $1461-1468, 1679-1685$  &  $ 1461-1468, 1679-1685$\\
		
		\hline
		\hline
	\end{tabular}
\end{table}

\subsection{FWHM}
By defining the height of the line profile between its maximum and the zero-flux baseline, the FWHM of each line profile in the AVG and RMS spectra is measured. This is done by linear interpolation between the two intersections of the line profile at the height 0.5. If no data point is found at exactly 0.5, the two neighboring points are used to linearly interpolate the velocity at 0.5. \\
With this method, the FWHM of the Balmer line profiles, the Ly$\alpha$ line profile, the He\,\textsc{i}$\,\lambda5875$ line profile, the C\,\textsc{iv}$\,\lambda\lambda 1548,\,1550$ doublet profile ,the Si\,\textsc{iv}$\,\lambda\lambda 1238,\,1242$ doublet profile and the He\,\textsc{ii}$\,\lambda1640$ line profile are measured in both the AVG and RMS spectra. In addition, the FWHM is measured only from the RMS profile of He\,\textsc{ii}$\,\lambda4685$ as its blue flank of its AVG profile is blended by the Fe\,\textsc{ii} band between $4489\,\AA$ and $4629\,\AA$. All measurements of the FWHM for the previous named emission lines are performed with GECHO. The FWHM of the AVG profile of He\,\textsc{ii}$\,\lambda4685$ is estimated by doubling the width of its red flank. The same approach is used for the N\,\textsc{V}$\,\lambda\lambda 1238,\,1242$ doublet in the AVG and RMS spectra, as it is blended with the Ly$\alpha$ profile. Finally, the FWHM measurements for O\,\textsc{i}$\,\lambda 8446$ is excluded from the subsequent analysis as its RMS profile is too noisy for a line width estimation. The resulting FWHM values are listed in Table \ref{tab:line_width_FWHM}.
\begin{figure}[!ht]
	\centering
	\includegraphics[width=\textwidth]{pictures/Chapter4/line_profiles/Normalized_Line_Profiles.pdf}
	\caption{A plot of the AVG and RMS line profiles.}
	\label{fig:line_profiles}
\end{figure}
\clearpage
The uncertainties are estimated based on the instrumental dispersion of the gratings (see Table \ref{tab:stis_gratings}), the shape of the line profiles (see Figure \ref{fig:line_profiles}), and the measurement method. H$\alpha$ and He\,\textsc{i}$\,\lambda5875$ are measured with the G750L grating, which has a dispersion of $4.97\,(\text{\AA}/\mathrm{pixel})$; H$\beta$, H$\gamma$, H$\delta$, and He\,\textsc{ii}$\,\lambda4685$ are measured with the G430L grating, which has a dispersion of $2.73\,(\text{\AA}/\mathrm{pixel})$; and the UV emission lines are measured with the G140L grating, which has a dispersion of $0.6\,(\text{\AA}/\mathrm{pixel})$. By using Equation \ref{eqn:velocity_space} together with the rest-frame central wavelengths of the emission lines, this corresponds to an equivalent dispersion per pixel in velocity space, listed in Table \ref{tab:grating_dispersion}. This dispersion is used as a minimum estimate for the uncertainty estimation. The FWHM uncertainties of the RMS profiles of H$\gamma$, He\,\textsc{i}$\,\lambda5875$, and the He\,\textsc{ii} lines are scaled up due to their profile shape and noise. The RMS profile of H$\delta$ it is much more scattered and extended at his blue flank at half height, which is why its uncertainty is estimated significantly higher. Ly$\alpha$ it is blended with absorption in its blue flank and with the N\,\textsc{iv}$\,\lambda\lambda 1238,\,1242$ doublet in its red flank, while the the profile of C\textsc{iv}$\,\lambda\lambda 1548,\,1550$ is a superposition of two lines, which gets accounted for with higher uncertainties as well. Finally, the uncertainties for the AVG and RMS profiles of N\,\textsc{v}$\,\lambda\lambda 1238,\,1242$ and for the AVG profile of He\,\textsc{ii}$\,\lambda4685$ are set significantly higher, since they are estimated from the width of only one flank.\\
It has to be noted, that the line profiles of the AVG spectrum includes narrow components of the respective emission line, as well in some cases a overlapping profile of a close narrow emission line (e.g. the [O\,\textsc{iii}] $\lambda4363$ next to H$\gamma$). The narrow components are hard to isolate in the profiles, which is why no attempt was made to decompose the AVG profiles. Having no narrow components in the RMS profiles, their FWHM will be used in subsequent analysis to describe the velocity dispersion of their emitting region.
\begin{table}[!ht]
	\centering
	\small
	\caption{Dispersion of grating in velocity space.}
	\label{tab:grating_dispersion}
	\begin{tabular}{lc}
		\hline
		\hline
		\textbf{Line} & \textbf{Dispersion} [$\frac{\mathrm{km}}{\mathrm{s}}/\mathrm{pixel}$]  \\
		\hline
		\hline
		Ly$\alpha$ & $144.0$\\
		\hline
		H$\alpha$  & $222.9$\\
		H$\beta$ & $169.3$ \\
		H$\gamma$ & $189.6$ \\
		H$\delta$ & $200.6$ \\
		\hline
		He\,\textsc{i}$\,\lambda5875$ & $249.0$  \\
		He\,\textsc{ii}$\,\lambda1640$  & $106.7$\\
		He\,\textsc{ii}$\,\lambda4685$& $175.6$ \\
		\hline
		N\,\textsc{v}$\,\lambda\lambda 1242$ &  $141.3$\\
		C\,\textsc{iv}$\,\lambda\lambda 1548$ & $113.1$\\
		
		\hline
		\hline
	\end{tabular}
\end{table}
\begin{table}[!ht]
	\centering
	\small
	\caption{Measured FWHM and of the AVG and RMS line profiles.}
	\label{tab:line_width_FWHM}
	\begin{tabular}{lcc}
		\hline
		\hline
		\textbf{Line} & FWHM (avg)[km/s] & FWHM (rms)[km/s]   \\
		\hline
		\hline
		Ly$\alpha$ & $3819 \pm 350$  &  $4566\pm 350$\\
		\hline
		H$\alpha$  & $2974 \pm 250$ &  $3111 \pm 250$\\
		H$\beta$ & $3439 \pm 200$&  $3437 \pm 200$ \\
		H$\gamma$ & $4067 \pm 200$&  $3852 \pm 400$ \\
		H$\delta$ & $3398 \pm 250$&  $5905 \pm 1000$ \\
		\hline
		He\,\textsc{i}$\,\lambda5875$ & $3477 \pm 300$&  $3952 \pm 400$  \\
		He\,\textsc{ii}$\,\lambda1640$  & $2847 \pm 500$ & $6891 \pm 600$\\
		He\,\textsc{ii}$\,\lambda4685$&  $1839 \pm 600$&  $5971 \pm 400$ \\
		
		\hline
		N\,\textsc{v}$\,\lambda\lambda 1238,\,1242$ &  $3216 \pm 800$&  $3383 \pm 1000$\\
		Si\,\textsc{iv}$\,\lambda\lambda 1238,\,1242$ &  $5184 \pm 500$&  $10005 \pm 1000$\\
		C\textsc{iv}$\,\lambda\lambda 1548,\,1550$ & $ 5413\pm 500$  &  $ 8428 \pm 500$\\
		
		\hline
		\hline
	\end{tabular}
\end{table}



\subsection{Balmer Line-Profiles}

Figure \ref{fig:RMS_Balmer} shows a comparison of the RMS line profiles of the Balmer lines. They all show a overall similar double-peaked shape, which is red-shifted compared to their AVG profiles. Most parts of the line lies at positive velocities, with the peaks located between $\sim 0$--$500\,\mathrm{km\,s^{-1}}$ and between $\sim 1500$--$2000\,\mathrm{km\,s^{-1}}$. Following that, the Balmer lines show the most variation within the red wing of the emission lines. The blue peak shows higher flux values that the red peak, giving the profile a asymmetric shape. The profile of H$\delta$ shows an additional peak at about $1000\,\mathrm{km\,s^{-1}}$ with a nearly similar flux than the blue peak of its profile.\\
Especially the profiles of H$\alpha$ and H$\beta$ show nearly similar blue and red flank. The blue flank of H$\gamma$ and H$\delta$ is more extended and scattered, while their red flank also show a similar steep slope with another small bump at about $\sim 3000 \mathrm{km\,s^{-1}}$. This leads to FWHM values between $3100$ -- $3900\mathrm{km\,s^{-1}}$ for H$\alpha$, H$\beta$ and H$\gamma$ and a much higher value of about $\sim 5900 \mathrm{km\,s^{-1}}$ for H$\delta$, which results from the much more extended and scattered blue flank at half height of the profile. Taking the similarities of the RMS profile shapes, it can be assumed that they would emerge under similar kinematic properties, which is why the FWHM of H$\delta$ will not taken into account for the subsequent analysis. 
\begin{figure}[!ht]
	\centering
	\includegraphics[width=1\textwidth]{pictures/Chapter4/line_profiles/RMS_overlay_Balmer.pdf}
	\caption{Comparison of the normalized AVG line profiles of the Balmer lines H$\alpha$, H$\beta$, H$\gamma$ and H$\delta$ in velocity space.}
	\label{fig:RMS_Balmer}
\end{figure}\\
\newpage
\subsection{Helium Line-Profiles}
The RMS profiles, as well as a comparison of the He\,\textsc{ii}$\,\lambda1640$, He\,\textsc{ii}$\,\lambda4685$ and He\,\textsc{i}$\,\lambda5875$ is shown in Figure \ref{fig:RMS_Helium}. 

\begin{figure}[!ht]
	\centering
	\includegraphics[width=1\textwidth]{pictures/Chapter4/line_profiles/RMS_overlay_Helium.pdf}
	\caption{Comparison of the AVG and RMS line profiles of the Helium lines HeI$\,\lambda5875$ vs HeII$\,\lambda$4685.}
	\label{fig:RMS_Helium}
\end{figure}


\subsection{UV Line Profiles}

Figure \ref{fig:RMS_UV} shows the RMS profiles of the Ly$\alpha$ line, the N\,\textsc{v}$\,\lambda\lambda 1238,\,1242$ doublet, the  Si\,\textsc{iv}$\,\lambda\lambda 1393,\,1402$ doublet and the 	C\textsc{iv}$\,\lambda\lambda 1548,\,1550$ doublet. Like mentioned before, the line profile of Ly$\alpha$ is blended with line absorption in its blue flank, as well as with the N\,\textsc{v}$\,\lambda\lambda 1238,\,1242$ doublet in his red flank. Still the central part of the line profile is good distinguishable, with a FWHM of about $\sim 4500 \pm 350 \mathrm{km\,s^{-1}}$. Similar to the Balmer lines, its profile is red-shifted, showing more variation in the red wing of the emission line. \\
The RMS profile of the neighboring N\,\textsc{v}$\,\lambda\lambda 1238,\,1242$ doublet is more difficult to distinguish, as its blue flank is blended with absorption and Ly$\alpha$ variation. Due to this uncertainty, its FWHM will not taken into account for the subsequent analysis. The RMS profile Si\,\textsc{iv}$\,\lambda\lambda 1393,\,1402$ doublet shows a very broad and scattered profile and is blended with the semi-forbidden line doublet O\textsc{IV}]$\,\lambda\lambda\,1397,\,1400$. This leads to a very broad FWHM of about $\sim 10000 \pm 1000 \mathrm{km\,s^{-1}}$, which also will not taken into account for the subsequent analysis, due to its noise and a possible interference with the O\textsc{IV}]$\,\lambda\lambda\,1397,\,1400$ doublet.\\
Finally, the RMS profile of the C\textsc{iv}$\,\lambda\lambda 1548,\,1550$ doublet shows a very asymmetric profile with a broad plateau like blue wing, with its flux staying below half height of the maximum until about $-3000\mathrm{km\,s^{-1}}$. The center of the RMS profile is shows a double-peak-like shape with a narrower blue peak and a broader red peak. They are located at about $\sim -1500\mathrm{km\,s^{-1}}$ and $250\mathrm{km\,s^{-1}}$, with higher flux in the red peak. Between the peaks the flux drops below half of the maximum which is why the left flank of the blue peak and the right flank of the red peak has been used to measure the FWHM. 


\begin{figure}[!ht]
	\centering
	\includegraphics[width=0.7\textwidth]{pictures/Chapter4/line_profiles/UV_Lines.pdf}
	\caption{UV}
	\label{fig:RMS_UV}
\end{figure}


\clearpage






\section{Time Lag}
\label{sec:time_lag_bh_mass}







\begin{figure}[!ht]
	\centering
	\includegraphics[width=0.9\textwidth]{pictures/Chapter4/lighcurves_and_ccfs_and_time_lag_tables/UVW2_ccfs_and_reference_lightcurves_optical.pdf}
	\caption{Compared lightcurves and CCFs H$\alpha$, H$\beta$, H$\gamma$, He\,\textsc{i}$\,\lambda5875$, He\,\textsc{ii}$\,\lambda4685$  and  O\,\textsc{i}$\,\lambda 8446$ with UVW2 as reference lightcurve.}
	\label{fig:ccfs_optical}
\end{figure}

\begin{figure}[!ht]
	\centering
	\includegraphics[width=\textwidth]{pictures/Chapter4/lighcurves_and_ccfs_and_time_lag_tables/UVW2_ccfs_and_reference_lightcurves_not_optical.pdf}
	\caption{Compared lightcurves and CCFs of UV lines with UVW2 as reference lightcurve.}
	\label{fig:ccfs_UV}
\end{figure}
\clearpage

\begin{table}[!ht]
	\centering
	\small
	\caption{Centroid and Peak Time Lag for UVW2.}
	\label{tab:lags_UVW2}
	\begin{tabular}{l c }
		\hline
		\hline
		\textbf{Line} & $\tau_{\text{cent}}$ [d]  \\
		\hline
		\hline
		Ly$\alpha$ & $1.0 \ensuremath{_{-0.2}^{+0.4}}$  \\
		H$\alpha$ & $3.2 \ensuremath{_{-0.6}^{+1.0}}$  \\
		H$\beta$ & $2.3 \ensuremath{_{-0.3}^{+1.0}}$  \\
		H$\gamma$ & $1.7 \ensuremath{_{-0.4}^{+0.2}}$  \\
		H$\delta$ & $1.7 \ensuremath{_{-0.4}^{+0.4}}$  \\
		He\,\textsc{i}$\,\lambda5875$ & $3.3 \ensuremath{_{-0.4}^{+0.7}}$  \\
		He\,\textsc{ii}$\,\lambda1640$ & $0.3 \ensuremath{_{-0.2}^{+0.6}}$  \\
		He\,\textsc{ii}$\,\lambda4685$ & $0.7 \ensuremath{_{-0.3}^{+0.1}}$ \\
		O\,\textsc{i}$\,\lambda8446$ & $5.0 \ensuremath{_{-1.7}^{+2.1}}$ \\
		Si\,\textsc{iv}$\,\lambda\lambda 1393,\,1402 $ & $1.2 \ensuremath{_{-0.4}^{+0.1}}$  \\
		C\,\textsc{iv}$\,\lambda 1548$ & $0.8 \ensuremath{_{-0.0}^{+0.5}}$ \\
		\hline
		\hline
	\end{tabular}
\end{table}


\section{Bowen Fluorescence of O\,\textsc{I}$\,\lambda 8446$}

Emission lines in the optical range, like the Balmer and helium lines, are commonly used in many reverberation-mapping campaigns, as they are easily accessible for nearby AGN \parencite{ochmann2026first}. Other lines, such as the low-ionization line O\,\textsc{i}$\,\lambda 8446$, have not been of major interest in such campaigns, mostly because of observational limitations \parencite{ochmann2026first}. This is why the variability of the O\,\textsc{i}$\,\lambda 8446$ line in this campaign is of particular interest, as the line can be enhanced by Bowen fluorescence, pumped by Ly$\beta$ emission \parencite{grandi1980}. The time delay between the ionizing continuum and O\,\textsc{i}$\,\lambda 8446$ should be consistent with a time delay in which continuum
variations drive Ly$\beta$ variations, which in turn drive part of the O\,\textsc{i}$\,\lambda 8446$ response. Unfortunately, Ly$\beta$ lies outside the spectral range of the \cite{cackett2018accretion} campaign. However, because the Ly$\alpha$ and Ly$\beta$ lines are expected to originate under similar physical conditions \parencite{ochmann2026first}, Ly$\alpha$ can be used as a proxy for Ly$\beta$. H$\alpha$ is used to investigate the location of the O\,\textsc{i}$\,\lambda 8446$ emitting region relative to the Balmer-line emitting region.\\
Therefore, the time lags between the UVW2 and emission-line lightcurves, as well as the CCFs and time lags between O\,\textsc{i}$,\lambda 8446$ and Ly$\alpha$, between H$\alpha$ and Ly$\alpha$, and between O,\textsc{i}$\,\lambda 8446$ and H$\alpha$, are calculated and shown in Figure \ref{fig:ccfs_Bowen}. All lightcurve show a strong correlation between $\approx 0.75 - 0.8$, with the exception of the O\,\textsc{i}$\,\lambda 8446$ and UVW2 lightcurves, which show a maximum correlation of only $0.6$. The time lags and their uncertainties are calculated as before. O\,\textsc{i}$,\lambda 8446$ lags behind Ly$\alpha$ by $2.5 \ensuremath{_{-0.4}^{+1.8}}$ days and behind UVW2 by $5.0 \ensuremath{_{-1.7}^{+2.1}}$ days, while H$\alpha$ lags behind Ly$\alpha$ by $1.8 \ensuremath{{-0.5}^{+0.5}}$ days and behind UVW2 by $3.3 \ensuremath{_{-0.7}^{+1.0}}$ days. A difference in the response time of about $0.5$--$1$ days between O\,\textsc{i}$\,\lambda 8446$ and H$\alpha$ is noticeable, but are not significant within the uncertainties. Looking at the O\,\textsc{i}$\,\lambda 8446$ and H$\alpha$ lightcurves, they also show a strong correlation, with O\,\textsc{i}$\,\lambda 8446$ lagging behind H$\alpha$ by about $0.3 \ensuremath{_{-0.1}^{+2.0}}$ days. With the Ly$\alpha$ light curve lagging behind the UVW2 light curve by $1.1 \ensuremath{_{-0.2}^{+0.4}}$ days, the total lag along the Bowen-fluorescence path from UVW2 to Ly$\alpha$ and from Ly$\alpha$ to O\,\textsc{i}$\,\lambda 8446$ sums to $\approx 3.6$ days, which lies within the uncertainties of the time lag between the H$\alpha$ and UVW2 light curves.

\begin{figure}[!ht]
	\centering
	\includegraphics[width=\textwidth]{pictures/Chapter4/lighcurves_and_ccfs_and_time_lag_tables/OI_ccfs_and_reference_lightcurves_paper HAlpha.pdf}
	\caption{Compared lightcurves and CCFs for Bowen Fluorescence.}
	\label{fig:ccfs_Bowen}
\end{figure}
\clearpage
\section{Black Hole Mass}

The final step of the RM analysis it now the estimation of mass of the SMBH. To do that, the virial theorem get applied following the method of \cite{peterson2004}. The FWHM of the RMS profiles is used to describe the velocity dispersion $\Delta V$ which is why a value of $f=1.8$ gets assumed for the scale factor following \cite{probst2025emissionlinecontinuumreverberationmapping} (see Section \ref{subsec:BHM}). Taking equation \ref{eqn:charc_radius}, \ref{eqn:M_vir} and \ref{eqn:M_BH} the SMBH mass can be estimated based on the time lag and FWHM of for each emission line using: 
\begin{equation}
	\label{eqn:BHM}
	M_{\mathrm{BH}} = 1.8 \cdot \frac{c \cdot \tau_{\mathrm{centroid}}\,\mathrm{FWHM}^2}{G}\,.
\end{equation}
As discussed in Section \ref{sec:line_profiles}, H$\delta$, O\,\textsc{i}$\,\lambda8446$, N\,\textsc{v}$\,\lambda\lambda 1238,\,1242$ and Si\,\textsc{iv}$\,\lambda\lambda 1238,\,1242$ are not included, due to not measurable or significant FWHM values.
The resulting SMBH mass results, as well as the corresponding time lag and FWHM values of the broad emission lines are listed in Table \ref{tab:BH_mass}. To obtain a final estimation for the SMBH mass, a inverse-variance weighted mean is calculated. Since the uncertainties are asymmetric a symmetrized 1$\sigma$ uncertainty for the weights gets adopter, $\sigma_i=(\sigma_i^-+\sigma_i^+)/2$.
The weighted mean is then given by
\begin{equation}
	\bar{M}_{\mathrm{BH}}=\frac{\sum_i w_i M_{\mathrm{BH},i}}{\sum_i w_i}\,,\qquad
	w_i=\frac{1}{\sigma_i^2}\,,
\end{equation}
with an uncertainty
\begin{equation}
	\sigma_{\bar{M}}=\left(\sum_i w_i\right)^{-1/2}\,.
\end{equation}
This yields the weighted-mean SMBH mass of $\bar{M}_{\mathrm{BH}}\approx (0.975 \pm 0.131)\times 10^{7}\,M_\odot$.
\begin{table}[!ht]
	\centering
	\small
	\caption{Estimated time lags, FWHM and SMBH masses.}
	\label{tab:BH_mass}
	\begin{tabular}{l c c c}
		\hline
		\hline
		\textbf{Line} & $\tau_{\text{cent}}$ [d] & FWHM (rms)[km/s] & $M_{\text{BH}} [10^7 M_\odot]$ \\
		\hline
		\hline
		Ly$\alpha$ & $1.0 \ensuremath{_{-0.2}^{+0.4}}$ & $4566 \pm 150$ &$0.7 \ensuremath{_{-0.2}^{+0.3}}$ \\
		H$\alpha$ & $3.2 \ensuremath{_{-0.6}^{+1.0}}$ & $3111 \pm 250$ &$1.1 \ensuremath{_{-0.4}^{+0.6}}$ \\
		H$\beta$ & $2.3 \ensuremath{_{-0.3}^{+1.0}}$ & $3437 \pm 200$&$0.9 \ensuremath{_{-0.2}^{+0.6}}$ \\
		H$\gamma$ & $1.7 \ensuremath{_{-0.4}^{+0.2}}$ & $3852 \pm 300$ &$0.9 \ensuremath{_{-0.3}^{+0.3}}$ \\
		He\,\textsc{i}$\,\lambda5876$ & $3.3 \ensuremath{_{-0.4}^{+0.7}}$ & $ 3952 \pm 300$&$1.8 \ensuremath{_{-0.5}^{+0.7}}$ \\
		He\,\textsc{ii}$\,\lambda1640$ & $0.3 \ensuremath{_{-0.2}^{+0.6}}$ & $6891 \pm 600$ & $0.4 \ensuremath{_{-0.3}^{+1.2}}$ \\
		He\,\textsc{ii}$\,\lambda4686$ & $0.7 \ensuremath{_{-0.3}^{+0.1}}$ & $5972 \pm 300$ &$0.9 \ensuremath{_{-0.4}^{+0.2}}$ \\
		C\,\textsc{iv}$\,\lambda\lambda 1548,\,1550$ & $0.8 \ensuremath{_{-0.1}^{+0.5}}$ & $8428 \pm 300$ &$1.9 \ensuremath{_{-0.2}^{+1.7}}$ \\
		
		
		\hline
		\hline
	\end{tabular}
\end{table}\\
It has to be noted, that the scale factor of $f=1.8$ does not account the low-inclination $i \sim 11^\circ$ of the elliptic accretion disk, modeled in \cite{ochmann2024transient}. Adopting the scaling relation $f \sim \sin^{-2}(i)$ introduced by \cite{krolik2001systematic}, a value of $f \sim 27.5$ would be needed to account for the low inclination. \cite{krolik2001systematic} used the line dispersion to parameterize the velocity dispersion, which is why the relation $\sigma_{\mathrm{line}} \approx \mathrm{FWHM}/2$ \parencite{peterson2004} has to be accounted again. Due to the square relation of the velocity dispersion to the black hole mass seen in Equation \ref{eqn:BHM}, the scale factor finally gets corrected to a value $f \sim 6.8$. Subsequently the found BH mass has to be multiplied by a additional factor of $3.77$. \\
Subsequently, the weighted-mean SMBH mass yields $\bar{M}_{\mathrm{BH,corr}} \approx (3.68 \pm 0.49)\times 10^{7}\,M_\odot$.







