\chapter{Reverberation Mapping Analysis of NGC4593}
\label{cap: Results}

\section{Line Identification}

The first step of the reverberation mapping analysis is the identification of emission and absorption lines within the AVG spectrum. Figures  \ref{fig:AVG_RMS_SPECTRUM} and \ref{fig:UV_uncalibrated_AVG_RMS} show the optical to near-infrared range between $3900 \AA$ and $9000 \AA$ and the UV range between $1100 \AA$ and $1700 \AA$, respectively. 
Looking at Figure \ref{fig:AVG_RMS_SPECTRUM}, it is possible to identify prominent broad emission lines in the AVG spectrum are the Balmer-Lines up to H$\epsilon$, with H$\alpha$ being the strongest followed by H$\beta$ and H$\gamma$, and several He-emission lines such as He\,I$\lambda 4471$, He\,II $\lambda 4685$, the He\,I $\lambda 5015$, He\,I $\lambda 5875$, He\,I $\lambda 7065$. Another significant broad emission line complex appears in the NIR part of the spectrum, including the O\textsc{i}$\,\lambda8446$ emission line and the Ca\,\textsc{ii}$\,\lambda8498\,\lambda8542\,\lambda8662$ triplet. All of those lines show significant variation in the RMS, especially the Balmer-lines and the He\,II $\lambda 4685$ and He\,I $\lambda 5875$ with very good distinguishable profiles. The He\,I$\lambda 4471$, the He\,I $\lambda 5015$, the He\,I $\lambda 7065$, the O\textsc{i}$\,\lambda8446$ emission line and the Ca\,\textsc{ii}$\,\lambda8498\,\lambda8542\,\lambda8662$ triplet show smaller but still noticeable variation. The He\,I$\lambda 4471$ and He\,I $\lambda 5875$ lines are blended with a Fe II emission line group in the AVG, which is vanishing in the RMS, as it shows no variation. The same aplies to the Fe II emission line group around 5200 \AA.\\
Apart from the broad emission lines, the AVG spectrum shows strong forbidden [O\,III] narrow emission lines. [O\,III] $\lambda 5007$ and [O\,III] $\lambda 4958$ are strong enough to be distinguishable, while the [O\,III] $\lambda 4363$ blends with H$\gamma$. As narrow lines they and several more weaker narrow lines, show no variability in the RMS spectrum over the period of the campaign. 
\begin{figure}[!htbp]
 	\centering
 	\makebox[\textwidth][c]{%
 		\includegraphics[width=1\textwidth]{pictures/Chapter4/avg_rms_spec/avg_rms_spec.pdf}}
 	\caption{Optical-to-NIR AVG and RMS spectrum with identified emission lines.}
 	\label{fig:AVG_RMS_SPECTRUM}
\end{figure}
\begin{figure}[!htbp]
	\centering
	\makebox[\textwidth][c]{%
		\includegraphics[width=\textwidth]{pictures/Chapter4/avg_rms_spec/UV_uncalibrated_AVG_RMS.pdf}}
	\caption{UV spectrum AVG and RMS spectrum with identified emission lines}
	\label{fig:UV_uncalibrated_AVG_RMS}
\end{figure}\\
The campaign also included an UV spectrum between $1000$ \AA and $1700$ \AA, which is shown in Figure \ref{fig:UV_uncalibrated_AVG_RMS}. The most prominent broad lines in the UV spectrum are the Ly$\alpha$ line, which is blended with the N\,V $\lambda\lambda 1238,1242$ duplet, the Si\,IV$\lambda\lambda 1393,1402$ duplet, which is blended with a partly forbidden O\,IV] line, and the  C\,IV$\lambda\lambda 1548,1550$ duplet. They all show strong variation in the RMS spectrum. Apart of the strong lines, a weaker variable Helium II line can be identified at $1640$ \AA.  \\
The Bowen fluorescence of O\,I $\lambda 8446$, which is investigated in this analysis, is typical driven by the emission of the Ly$\beta$ line \parencite{grandi1980}, which lies outside the spectral range of the HST Campaign. However, as can be seen in figure \ref{fig:UV_uncalibrated_AVG_RMS}, Ly$\alpha$  still lies in the spectral range of the campaign, as the bluest broad line of the taken spectra. As Ly$\alpha$ and Ly$\beta$ are assumed to originate in the same physical region, Ly$_\alpha$ can be used to examine the correlation of O\,I $\lambda 8446$ and Ly$\alpha$ for Bowen fluorescence.


\section{Emission Line and Continua Measurement}

After identifying the emission lines, the next step of the analysis is to measure the fluxes of those lines. This is done with the help of a python-based tool called GECHO created by M. Probst. It enables to import full campaigns, determine the AVG and RMS spectra, extracting lightcurves and do further measurements, naming here line-width measurements and lags measurement using the ICCF methode \parencite{gaskell_peterson1986, peterson2004}, which will be further discussed in Section \ref{sec:line_profiles} and \ref{sec:time_lag_bh_mass}.\\
With help of GECHO the flux density gets integrated over the extent of each emission line for every observed spectrum of the campaign. Here it is important to define the integration limits so that all of the emission line flux is measured in this way. To ensure this, a parallel view of the AVG- and RMS-spectrum was used to define those integration limits, and to ensured, that no component of any other line is contributing to the line flux of the measured line. Additionally, the integration boundaries get selected, making sure they span a similar area in velocity space for every emission line. To accounting the surrounding continuum a linear underlying continuum gets interpolated and subtracted first. For this interpolation, sections on the blue and red side of the emission line with no line contribution gets selected. The chosen integration limits and pseudo-continua are listed in Table \ref{tab:emission_lines}. \\
To measure the delay between the ionizing continuum and the emission lines it is necessary to extract lightcurves of selected continua in the AVG spectrum. Continua lightcurves are getting extracted by measuring the mean flux density of a sufficiently large region without line emission or absorption. The chosen integration limits for the continua can be found in Table \ref{tab:continua}.\\



\begin{table}[h!]
	\centering
	\small
	\caption{Integration Limits and Pseudo-Continua of the measured emission lines}
	\label{tab:emission_lines}
	\begin{tabular}{lcc}
		\hline
		\hline
		\textbf{Line} & \textbf{Integration Limits $[\AA]$} & \textbf{Pseudo-Continua $[\AA]$}  \\
		\hline
		Ly$\alpha$ & $1207-1238$ & $1151-1161, 1461-1469$\\
		NV$\,\lambda\lambda 1238,\,1242$ & $1207-1238$ & $1151-1161, 1461-1468$\\
		SiIV$\,\lambda\lambda 1393,\,1402$ & $1358-1423$ & $1151-1161, 1461-1469$\\
		CIV$\,\lambda\lambda 1548,\,1550$ & $1511-1578$ & $1461-1469, 1680-1685$\\
		HeII$\,\lambda1640$ & $1599-1645$ & $1461-1468, 1680-1685$\\
		\hline
		H$\alpha$ & $6453-6695$ & $6107-6129, 6861-6900$ \\
		H$\beta$ & $4779-4944$ & $4762-4774, 5085-5112$ \\
		H$\gamma$ & $4230-4427$ & $4197-4220, 4435-4450$ \\
		
		HeI$\,\lambda5875$ & $5742-6039$ & $5645-5653, 6044-6057$ \\
		HeII$\,\lambda4685$ & $4545-4758$ & $4435-4450, 4762-4774$ \\
		OI$\,\lambda 8446$ & $8380-8498$ & $8005-8031, 8850-8955$ \\
		\hline
		OIII$\,\lambda 5007$ & $4982-5033$ & $4762-4774, 5085-5112$ \\
		\hline
		\hline
	\end{tabular}
\end{table}
\begin{table}[h!]
	\centering
	\small
	\caption{Integration Limits of the measured continua}
	\label{tab:continua}
	\begin{tabular}{lc}
		\hline
		\hline
		\textbf{Line} & \textbf{Integration Limits $[\AA]$}  \\
		\hline
		Cont. 1150 & $1151-1161$\\
		\hline
		Cont. 4010 & $4026-4033$\\
		Cont. 4440 & $4435-4450$\\
		Cont. 5100 & $5085-5112$\\
		Cont. 6110 & $6107-6129$\\
		Cont. 6880 & $6861-6900$\\
		Cont. 8015 & $8005-8031$\\
		Cont. 8900 & $8864-8955$\\
		\hline
		\hline
	\end{tabular}
\end{table}

\subsection{Variability Statistics}

To quantify the variability of the emission lines and continua variability statistics the definition by \cite{rodriguez1997steps} are adopted. They name the  maximum-to-minimum flux ratio $R_\mathrm{max}$ and the fractional variability	$F_\mathrm{var}$ as two common measures of variability. $R_\mathrm{max}$ is defined as the ratio of the the extreme of the integrated fluxes, $F_\mathrm{min}$ and $F_\mathrm{max}$, and  $F_\mathrm{var}$ as: 
\begin{equation}
	F_\mathrm{var}=\frac{\sqrt{\sigma_F^2-\Delta^2}}{\left<F\right>}
\end{equation}
Here, $\sigma_F^2$ denotes the standard deviation, $\left<F\right>$ the mean flux and $\Delta^2$ the mean square value of the flux uncertainties, which is defined by:
\begin{equation}
	\Delta^2 = \frac{1}{N} \sum_{i=1}^{N}\Delta_i^2
\end{equation}
The results for all parameters can be found in the Tables \ref{tab:varstatistics}.\\
First, considering $R_\mathrm{max}$ and $F_\mathrm{var}$ of the continuum lightcurves, all measured continua show similar variability, except for the UV continuum around $1150$ \AA, which shows significantly higher variability with $R_\mathrm{max} = 2.58$ and $F_\mathrm{var} = 0.28$ than the optical and NIR continua. These continua show a fairly uniform variation between $R_\mathrm{max} \simeq 1.33-1.59$ and $F_\mathrm{var} \simeq 0.08-0.14$. \\
The lightcurves of the Balmer emission lines exhibit lower variability, which increases towards the higher-order Balmer lines, with values between $R_\mathrm{max} \simeq 1.15-1.52$ and $F_\mathrm{var} \simeq 0.03-0.1$. The helium lightcurves show a similar variability to H$\delta$ with values  between $R_\mathrm{max} \simeq 1.48-1.62$ and $F_\mathrm{var} \simeq 0.09-0.11$ and the OI$\,\lambda 8446$ a similar variability to the lower-order balmer lines with $R_\mathrm{max} \simeq 1.22$ and $F_\mathrm{var} \simeq 0.04$.\\
The variability of the emission line lightcurves in the UV region is on a similar level as that of the helium light curves in the optical region, with $R_\mathrm{max} \simeq 1.42-1.62$ and $F_\mathrm{var} \simeq 0.14$, with the exception of the HeII$,\lambda1640$ lightcurve, which shows significant higher variability with $R_\mathrm{max} \simeq 3.12$ and $F_\mathrm{var} \simeq 0.31$.

\begin{table}[!htb] 
\centering 
\caption{Variability statistics of the measured continua and emission lines with minimum flux(2) and maximum flux density or integrated flux (3), peak-to-peak ratio (4), mean (5), standard deviation (6) and fractional variation (7).} 
\begin{tabular}{lrrrrrr} 
	\hline 
	\hline 
	Continuum/Line &  \multicolumn{1}{c}{$F_{\text{min}}$} &  \multicolumn{1}{c}{$F_{\text{max}}$} &  \multicolumn{1}{c}{$R_{\text{max}}$} &  \multicolumn{1}{c}{$\langle F \rangle$} &  \multicolumn{1}{c}{$\sigma_F$} &  \multicolumn{1}{c}{$F_{\text{var}}$} \\ 
	\multicolumn{1}{c}{(1)} & \multicolumn{1}{c}{(2)} & \multicolumn{1}{c}{(3)} & \multicolumn{1}{c}{(4)} & \multicolumn{1}{c}{(5)} & \multicolumn{1}{c}{(6)} & \multicolumn{1}{c}{(7)} \\ 
	\hline
	Cont. 1150  & $0.52$ & $1.35$ & $2.58$ & $0.86$ & $0.25$ & $0.28$ \\
	Cont. 4010  & $2.68$ & $4.21$ & $1.57$ & $3.49$ & $0.47$ & $0.14$ \\
	Cont. 4440  & $2.42$ & $3.73$ & $1.54$ & $3.14$ & $0.39$ & $0.12$ \\
	Cont. 5600  & $1.36$ & $2.15$ & $1.59$ & $1.82$ & $0.25$ & $0.14$ \\
	Cont. 6110  & $1.49$ & $2.27$ & $1.53$ & $1.9$ & $0.23$ & $0.12$ \\
	Cont. 6880  & $1.33$ & $2.01$ & $1.5$ & $1.72$ & $0.2$ & $0.11$ \\
	Cont. 8015  & $1.18$ & $1.69$ & $1.43$ & $1.48$ & $0.15$ & $0.1$ \\
	Cont. 8900  & $1.14$ & $1.52$ & $1.33$ & $1.38$ & $0.11$ & $0.08$ \\
	\hline 
	H$\alpha$  & $130.83$ & $149.85$ & $1.15$ & $141.35$ & $4.83$ & $0.03$ \\
	H$\beta$  & $38.29$ & $45.4$ & $1.19$ & $42.32$ & $1.94$ & $0.05$ \\
	H$\gamma$  & $18.82$ & $24.38$ & $1.3$ & $21.91$ & $1.29$ & $0.06$ \\
	H$\delta$  & $7.17$ & $10.92$ & $1.52$ & $9.13$ & $0.94$ & $0.1$ \\
	HeII$\,\lambda4685$  & $11.93$ & $17.7$ & $1.48$ & $14.82$ & $1.6$ & $0.11$ \\
	HeI$\,\lambda5875$   & $8.53$ & $13.82$ & $1.62$ & $11.58$ & $1.01$ & $0.09$ \\
	OI$\,\lambda 8446$ & $7.47$ & $9.13$ & $1.22$ & $8.32$ & $0.37$ & $0.04$ \\
	\hline 
	Ly$\alpha$  & $66.87$ & $94.88$ & $1.42$ & $82.21$ & $8.03$ & $0.1$ \\
	SiIV$\,\lambda\lambda 1393,\,1402$  & $21.93$ & $35.5$ & $1.62$ & $27.92$ & $3.88$ & $0.14$ \\
	CIV$\,\lambda\lambda 1548,\,1550$  & $115.31$ & $165.57$ & $1.44$ & $138.77$ & $12.12$ & $0.09$ \\
	HeII$\,\lambda1640$  & $6.83$ & $21.29$ & $3.12$ & $13.29$ & $4.13$ & $0.31$ \\
	\hline 
	\hline 
	\label{tab:varstatistics} 
\end{tabular} 

\end{table}


\subsection{Uncertainties Estimation}
The uncertainties of the continuum and emission-line lightcurves are estimated from two components. The first component originates from the noise in the flux densities over the pixels within the integration bounds, $N_\mathrm{pixel}$. For the continuum lightcurves, the noise uncertainty is estimated as 
\begin{equation}
	\sigma_\mathrm{noise}^{\mathrm{cont}} = \frac{\sigma_F}{\sqrt{N_\mathrm{pixel}^{\mathrm{cont}}}}, 
\end{equation} 
where $\sigma_F$ denotes the standard deviation of the flux density within the selected continuum window. Since the emission-line lightcurves are extracted by integrating the continuum-subtracted flux density, the corresponding noise uncertainty is estimated using the noise measured from the used  pseudo-continua, $\sigma_\mathrm{noise}^{\mathrm{p.cont.}}$:
\begin{equation}
	\sigma_\mathrm{noise}^{\mathrm{lines}} = \frac{\sigma_\mathrm{noise}^{\mathrm{p.cont.}} \cdot \Delta\lambda}{\sqrt{N_\mathrm{pixel}^{\mathrm{line}}}}.
\end{equation}
Here, $\Delta\lambda$ denotes the integration window used for the emission line, and $N_\mathrm{pixel}^{\mathrm{line}}$ denotes the number of pixels within this window.
\\
The second component originates from the intercalibration of the epochs of the campaign. The flux of each epoch is scaled based on the narrow emission line [O\,\textsc{iii}] $\lambda5007$, which is assumed to be constant over the campaign. To account for the systematic uncertainty introduced by the intercalibration, the fractional variability of [O\,\textsc{iii}] $\lambda5007$ is multiplied by the integrated flux values $f_i$  and used as a proxy for the intercalibration uncertainty:
\begin{equation}
	\sigma_i^\mathrm{cal.} = F_\mathrm{var}^{\mathrm{[O\,\textsc{iii}]}\,\lambda5007} \cdot f_i .
\end{equation}
This yields the total light-curve uncertainty:
\begin{equation}
	\sigma_i = \sqrt{\left(\sigma_i^\mathrm{cal.}\right)^2 + \left(\sigma_{\mathrm{noise}}\right)^2} ,
\end{equation}
where $\sigma_{\mathrm{noise}}$ corresponds to $\sigma_\mathrm{noise}^{\mathrm{cont}}$ for continuum light curves and to $\sigma_\mathrm{noise}^{\mathrm{lines}}$ for emission-line light curves.





\section{Lightcurves}
\label{sec:lightcurves}

The variability of the lightcurves can be seen in the visualization of the lightcurves in Figure \ref{fig:continua_lightcurves} and \ref{fig:emission_line_lightcurves}. In addition to the measured lightcurves of the HST campaign, the UVOT UVW2 lightcurve taken with SWIFT by \cite{mchardy2018x} is displayed for comparison which shows a higher sample-size that the lightcurves of the HST campaign. Its central wavelength is located at about 1930\,\AA\, \parencite{mchardy2018x} and was used as a reference lightcurve in \cite{cackett2018accretion} as well, which makes is interesting as a reference lightcurve for this analysis too. \\
Looking at the  continuum lightcurves in Figure \ref{fig:continua_lightcurves} they show a broadly similar overall shape. All lightcurves begin at the highest flux level in most cases, except for UVW2, whose peaks have comparable heights throughout, and the 1150 continuum, which reaches its highest flux at the second maximum. The initial maximum is followed by a pronounced flux minimum, with a decrease of about $25$--$46\%$ relative to the overall maximum flux of each curve. Subsequently, the light curves rise again toward another peak, which is more distinct in the UV and blue continua and is followed by another drop in flux. While all continua except Cont. 4010 exhibit an additional another local maximum, the flux then decline steadily toward a global minimum near the end of the campaign.
\begin{figure}[!ht]
	\centering
	\includegraphics[width=0.9\textwidth]{pictures/Chapter4/lightcurves/NGC4593_Continua.pdf}
	\caption{Comparison of the continua lightcurves. The first panel shows the UVW2 continuum lightcurve obtained from \cite{mchardy2018x}, while the other panels show the measured continua defined in Table \ref{tab:continua} }
	\label{fig:continua_lightcurves}
\end{figure}
\begin{figure}[!ht]
	\centering
	\includegraphics[width=0.9\textwidth]{pictures/Chapter4/lightcurves/NGC4593_Lines_UVW2.pdf}
	\caption{Comparison of the emission-line lightcurves to the UVW2 reference lightcurve in the first panel. The UVW2 lightcurve was obtained from \cite{mchardy2018x}}
	\label{fig:emission_line_lightcurves}
\end{figure}
\begin{figure}[!ht]
	\centering
	\includegraphics[width=0.9\textwidth]{pictures/Chapter4/lightcurves/NGC4593_UV_Lines_UVW2.pdf}
	\caption{Comparison of the emission-line lightcurves to the UVW2 reference lightcurve in the first panel. The UVW2 lightcurve was obtained from \cite{mchardy2018x}}
	\label{fig:UV_emission_line_lightcurves}
\end{figure}
\\
For the RM analysis the real distance between the SMBH and the BLR is of most interests. Assuming that the continuum radiation originate from the accretion disk it is common to use the bluest available continuum band, which is
expected to originate closest to the SMBH, which would be the continuum around 1150 \AA. But because the higher sample rate, the UVW2 continuum was selected as the main reference lightcurve for the further analysis. \\
Looking now at the emission-line lightcurves in Figure \ref{fig:emission_line_lightcurves} correlation between them and the UVW2 continuum can already be spotted. The three major features of the UVW2 lightcurve, the first strong minimum, the two maxima in the center and the strong drop at the end, can be also identified in the emission line lightcurves with respective shifts in time, which will be further investigated in Section \ref{sec:time_lag_bh_mass}.





%\begin{figure}[!ht]
%	\centering
%	\includegraphics[width=1\textwidth]{pictures/Chapter4/lightcurves/NGC4593_Lines_Cont1150_not_optical_calibrated.pdf}
%	\caption{AVG RMS Spektrum}
%	\label{fig:emission_line_lightcurves_hst}
%\end{figure}







\section{Line Profiles}
\label{sec:line_profiles}

The next step of the analysis is the extraction of line profiles of the broad emission lines from both, the AVG and RMS spectrum. By investigation their shape and line width it allows conclusions about the kinematics of the region their originated from. Following the same procedure as for the lightcurve extraction, an underlying linear continuum is interpolated using a chosen blue and red pseudo-continuum for each line in both the AVG and RMS spectra. The selected boundaries of the pseudo-continua are listed in Table \ref{tab:line_profiles_pseudo}. By subtracting this interpolated continuum, a new zero-flux baseline for each line profile gets defined, which allows a better comparison between the line profiles.
\begin{table}[h!]
	\centering
	\small
	\caption{Boundaries of the blue and red pseudo-continua used for the interpolation of underlying continua for line profile extraction.}
	\label{tab:line_profiles_pseudo}
	\begin{tabular}{lcc}
		\hline
		\hline
		\textbf{Line} & \textbf{\textbf{Pseudo-Continua AVG $[\AA]$}} & \textbf{Pseudo-Continua RMS $[\AA]$}  \\
		\hline
		H$\alpha$ & $6194-6216, 6861-6900$ & $6279-6301, 6742-6781$ \\
		H$\beta$ & $4762-4774, 5085-5112$ & $4762-4774, 4967-4984$ \\
		H$\gamma$ & $4197-4220, 4435-4450$ & $4197-4220, 4417-4429$ \\
		H$\delta$ & $4026-4033, 4197-4221$ &$4006-4016, 4197-4211$  \\
		\hline
		OI$\,\lambda 8446$ & $7999-8025, 8775-8798$ & $8222-8238, 8748-8767$ \\
		\hline
		HeI$\,\lambda5875$ & $5679-5697, 6044-6057$ & $5736-5753, 6027-6045$ \\
		HeII$\,\lambda4685$ & $4198-4225, 4762-4774$ & $4543-4554, 4766-4778$ \\
		\hline
		Ly$\alpha$ & $1151-1161, 1270-1285$  &  $1151-1161, 1340-1355$\\
		SiIV$\,\lambda\lambda 1393,\,1402 $ & $1350-1360, 1430-1440$ &  $1340-1355, 1430-1440$\\
		CIV$\,\lambda\lambda 1548,\,1550$ & $1461-1468, 1679-1685$  &  $ 1461-1468, 1679-1685$\\
		HeII$\,\lambda1640$  & $1461-1468, 1679-1685$ & $1461-1468, 1679-1685$\\
		\hline
		\hline
	\end{tabular}
\end{table}\\
Subsequently, the line profile is converted to velocity space using\\ 
\begin{equation}
	\label{eqn:velocity_space}
	v_i = c \cdot \frac{\lambda_i^2 - \lambda_\mathrm{central}^2}{\lambda_i^2 + \lambda_\mathrm{central}^2}.
\end{equation}\\
Here, $\lambda_i$ denotes the wavelength values, $\lambda_\mathrm{central}$ the central wavelength of the emission line in rest-frame and $c$ the speed of light. Finally, the flux is getting normalized to the maximum of each line profile. Figure \ref{fig:line_profiles} shows an overview of the extracted and normalized line profiles of H$\alpha$, H$\beta$, H$\gamma$, O\,\textsc{i}$\,\lambda8446$, He\,\textsc{i}$\,\lambda5875$, and He\,\textsc{ii}$\,\lambda4685$ in both the AVG and RMS spectra.
The Balmer lines and the helium lines show well-defined profiles in both AVG and RMS, making them suitable for further kinematic analysis. The O\,\textsc{i}$\,\lambda8446$ profile is blended with the Ca\,\textsc{ii} $\lambda8498$\, $\lambda8542$\, $\lambda8662$ triplet in the AVG spectrum. Moreover, the RMS profile of O\,\textsc{i}$\,\lambda8446$ shows low signal-to-noise, which is why its kinematics will not be further investigated. 
\begin{figure}[!ht]
	\centering
	\includegraphics[width=0.95\textwidth]{pictures/Chapter4/line_profiles/Normalized_Line_Profiles.pdf}
	\caption{A plot of the AVG and RMS line profiles of the broad emission lines H$\alpha$, H$\beta$, H$\gamma$, O\,\textsc{i}$\,\lambda 8446$, He\,\textsc{i}$\,\lambda5875$, He\,\textsc{ii}$\,\lambda4685$, He\,\textsc{ii}\,1640 and O\,\textsc{i}$\,\lambda8446$.}
	\label{fig:line_profiles}
\end{figure}

\begin{figure}[!ht]
	\centering
	\includegraphics[width=0.95\textwidth]{pictures/Chapter4/line_profiles/Normalized_Line_Profiles_UV.pdf}
	\caption{A plot of the AVG and RMS line profiles of the broad emission lines Ly$\alpha$, SiIV$\,\lambda\lambda 1393,\,1402$ and CIV$\,\lambda\lambda 1548,\,1550$.}
	\label{fig:line_profiles_UV}
\end{figure}


\subsection{FWHM}
After selecting the suitable broad emission lines, the FWHM of the normalized AVG and RMS profiles are calculated, which are listed in Table \ref{tab:line_width_FWHM}. Here the height of the line profile is defined between the zero-flux baseline and the maximum of the profile. The FWHM of the AVG profile of He\,\textsc{ii}$\,\lambda4685$ was not attempted to be calculated, as the blue wing is mostly overlapped by the Fe\,\textsc{ii} band between $4489 \, \AA$  and $4629 \, \AA$.\\
The uncertainties are estimated based on the instrumental dispersion of the grating, the shape and noise of the line profiles (see Table \ref{tab:stis_gratings} for the gratings). H$\alpha$ and He\,\textsc{i}$\,\lambda5875$ are measured with the G750L grating with a dispersion of $4.97\,(\text{\AA}/\mathrm{pixel})$, while the G430L grating has been use for H$\beta$, H$\gamma$, and He\,\textsc{ii}$\,\lambda4685$ with a dispersion of $2.73\,(\text{\AA}/\mathrm{pixel})$. By using Equation \ref{eqn:velocity_space} and the rest-frame central wavelengths of the emission lines, this corresponds to an equivalent dispersion listed in Table \ref{tab:grating_dispersion}. Taking the dispersion of the grating, the shape and the noise within the profiles, the uncertainties get estimated accordingly.
\begin{table}[h!]
	\centering
	\small
	\caption{Dispersion of grating in velocity space.}
	\label{tab:grating_dispersion}
	\begin{tabular}{lc}
		\hline
		\hline
		\textbf{Line} & \textbf{Dispersion [$\frac{\mathrm{km}}{\mathrm{s}}/\mathrm{pixel}$]}   \\
		\hline
		H$\alpha$  & $222.9$\\
		H$\beta$ & $169.3$ \\
		H$\gamma$ & $189.6$ \\
		H$\delta$ & $200.6$ \\
		\hline
		He\,\textsc{i}$\,\lambda5875$ & $249.0$  \\
		He\,\textsc{ii}$\,\lambda4685$& $175.6$ \\
		
		\hline
		Ly$\alpha$ & $144.0$\\
		SiIV$\,\lambda\lambda 1393,\,1402$ & $125.6$\\
		CIV$\,\lambda\lambda 1548,\,1550$ & $113.1$\\
		HeII$\,\lambda1640$  & $106.7$\\
		\hline
		\hline
	\end{tabular}
\end{table}


\begin{table}[h!]
	\centering
	\small
	\caption{Measured FWHM and of the AVG and RMS line profiles.}
	\label{tab:line_width_FWHM}
	\begin{tabular}{lcc}
		\hline
		\hline
		\textbf{Line} & \textbf{FWHM AVG[km/s]} & \textbf{FWHM RMS[km/s]}   \\
		\hline
		H$\alpha$  & $2974 \pm 250$ &  $3111 \pm 250$\\
		H$\beta$ & $3439 \pm 200$&  $3437 \pm 200$ \\
		H$\gamma$ & $4067 \pm 200$&  $3852 \pm 300$ \\
		H$\delta$ & $3398 \pm 210$&  $5905 \pm 300$ \\
		\hline
		He\,\textsc{i}$\,\lambda5875$ & $3477 \pm 300$&  $3952 \pm 300$  \\
		He\,\textsc{ii}$\,\lambda4685$&  -&  $5971 \pm 300$ \\
		
		\hline
		Ly$\alpha$ & $3819 \pm 150$  &  $4566\pm 150$\\
		SiIV$\,\lambda\lambda 1393,\,1402 $ & $5330\pm 150$ &  $10005 \pm 300$\\
		CIV$\,\lambda\lambda 1548,\,1550$ & $ 5413\pm 200$  &  $ 8428 \pm 300$\\
		HeII$\,\lambda1640$  & $2847 \pm 200$ & $7459 \pm 200$\\
		\hline
		\hline
	\end{tabular}
\end{table}

\subsection{Balmer-Lines}

A comparison of the Balmer-line profiles H$\alpha$, H$\beta$, and H$\gamma$ is shown in Figure \ref{fig:AVG_RMS_Balmer}. With the exception of the AVG profile of H$\gamma$, which is affected by the narrow [O\,\textsc{iii}] $\lambda4363$ line overlapping the H$\gamma$ profile, the lines exhibit a similar asymmetric shape their AVG profiles. All three lines show a steep blue wing, that broads to a very broad flank at about $\approx -2500\,\mathrm{km\,s^{-1}}$. H$\alpha$ and H$\beta$ exhibit nearly identical gradients, while H$\gamma$ is blended with  Fe II\,\textsc{ii} emission up to $\approx -2000\,\mathrm{km\,s^{-1}}$ before matching the slope of the other profiles, which could be explained underlying components of other broad and narrow emission lines. The red wings of H$\alpha$ and H$\beta$ follow a similar pattern, showing a bump at around $\approx 1000\,\mathrm{km\,s^{-1}}$ and $\approx 2500\,\mathrm{km\,s^{-1}}$, respectively, resulting in the asymmetric profile and therefore velocity distribution. Looking at the line widths of the AVG profiles, the FWHM differs significantly from each other as the profiles consists of broad and narrow components and in case of H$\gamma$ is overlapped by the [O\,\textsc{iii}] $\lambda4363$ emission line. These narrow components are hard to isolate in the profiles, no attempt was made to decompose the AVG profiles. Instead, the RMS line widths are adopted for the subsequent analysis as they show no narrow components of the timescale of the campaign. \\
The RMS profiles of the Balmer lines show an asymmetric double-peaked shape. Again, the profiles of H$\alpha$ and H$\beta$ show a very similar shape, while the blue wing of H$\gamma$ is more extended and becomes noisy towards increasing negative velocities, resulting in a higher line width compared to H$\alpha$ and H$\beta$. Since [O III] $\lambda4363$ vanish in the RMS profiles, the red wing of H$\gamma$ becomes clearly visible and follows a similar shape of H$\alpha$ and H$\beta$. For all three profiles, the double-peaked structure is red-shifted, with the blue peak located at $\approx 200\,\mathrm{km\,s^{-1}}$ and the red peak at $\approx 2000\,\mathrm{km\,s^{-1}}$, showing that most of the flux variation within the red wing of the emission lines. \\
Taking the strong similarities of the RMS profiles, it can be assumed that H$\alpha$, H$\beta$ and H$\gamma$ emerge under similar kinematic properties.
\begin{figure}[!ht]
	\centering
	\includegraphics[width=\textwidth]{pictures/Chapter4/line_profiles/AVG_overlay_Balmer.pdf}
	\caption{Comparison of the normalized AVG line profiles of the Balmer lines H$\alpha$, H$\beta$, H$\gamma$ and H$\delta$ in velocity space.}
	\label{fig:AVG_Balmer}
\end{figure}

\begin{figure}[!ht]
	\centering
	\includegraphics[width=\textwidth]{pictures/Chapter4/line_profiles/RMS_overlay_Balmer.pdf}
	\caption{Comparison of the normalized  RMS line profiles of the Balmer lines H$\alpha$, H$\beta$, H$\gamma$ and H$\delta$ in velocity space.}
	\label{fig:RMS_Balmer}
\end{figure}


\subsection{Helium-Lines}


Figure \ref{fig:AVG_RMS_Helium} shows a comparison of the AVG and RMS line profile of He\,\textsc{i}$\,\lambda5875$  and He\,\textsc{ii}$\,\lambda4685$. The blue wing of the He\,\textsc{ii}$\,\lambda4685$ AVG profile is partly overlapped by the Fe\,\textsc{ii} band between $4489 \, \AA$  and $4629 \, \AA$ at velocities above about $-1000\,\mathrm{km\,s^{-1}}$. While the visible part of its blue wing follows the same shape as the He\,\textsc{i}$\,\lambda5875$  blue wing shape, the red wing of He\,\textsc{i}$\,\lambda5875$  is much broader. Still, both red wings show a similar shape and an additional smaller peak at around $1000\,\mathrm{km\,s^{-1}}$. Since more than half of He\,\textsc{i}$\,\lambda5875$  blue wing is overlapped by the Fe\,\textsc{ii} band, a calculation of its FWHM was not attempted. Still, looking at the the line widths at 0.6 height of the line with $\approx 2500 \,\mathrm{km\,s^{-1}}$ for He\,\textsc{i}$\,\lambda5875$  and $\approx 1600 \,\mathrm{km\,s^{-1}}$ for He\,\textsc{ii}$\,\lambda4685$ with using the uncertainties from Table \ref{tab:line_width_FWHM}, it can be assumed that He\,\textsc{i}$\,\lambda5875$  show a much broader FWHM in the AVG than He\,\textsc{ii}$\,\lambda4685$. For the same reasons as for the balmer lines, a decomposition of the narrow and broad components was not attempted, which is why the FWHM of the RMS profiles will be used for the subsequent analysis.\\
Other than in the AVG profile, He\,\textsc{i}$\,\lambda5875$  now shows a smaller line width than He\,\textsc{ii}$\,\lambda4685$ and higher noise. Looking at the He\,\textsc{i}$\,\lambda5875$  profile, its maximum is red-shifted to $\approx 2000\,\mathrm{km\,s^{-1}}$, while the maximum of the He\,\textsc{ii}$\,\lambda4685$ profile is located around $0 \,\mathrm{km\,s^{-1}}$. Following that it He\,\textsc{i}$\,\lambda5875$  shows more variation in its red wing than He\,\textsc{ii}$\,\lambda4685$ does. The blue wing of He\,\textsc{i}$\,\lambda5875$ extend up to about $\approx -5000\,\mathrm{km\,s^{-1}}$ and its red wing about $\approx 4000\,\mathrm{km\,s^{-1}}$ while He\,\textsc{ii}$\,\lambda4685$ blue wing extend up to about $\approx -9000\,\mathrm{km\,s^{-1}}$ its red wing about $\approx 5000\,\mathrm{km\,s^{-1}}$. Going from the maxima, it shows that both profiles show a much more extended blue wing and a very steep red wing. 
 

\begin{figure}[!ht]
	\centering
	\includegraphics[width=\textwidth]{pictures/Chapter4/line_profiles/AVG_and_RMS_overlay_Helium.pdf}
	\caption{Comparison of the AVG and RMS line profiles of the Helium lines HeI$\,\lambda5875$ vs HeII$\,\lambda$4685.}
	\label{fig:AVG_RMS_Helium}
\end{figure}




\section{Time Lag and BH Masses}
\label{sec:time_lag_bh_mass}

\begin{table}[!htb]
	\centering
	\small
	\caption{Centroid and Peak Time Lag for UVW2.}
	\label{tab:lags_UVW2}
	\begin{tabular}{l c  c}
		\hline
		\hline
		Name & $\tau_{\text{cent}}$ [d] &  $M_{\text{BH}} [10^7 M_\odot]$ \\
		\hline
		H$\alpha$ & $3.2 \ensuremath{_{-0.6}^{+1.0}}$ &  $1.1 \ensuremath{_{-0.4}^{+0.6}}$ \\
		H$\beta$ & $2.3 \ensuremath{_{-0.3}^{+1.0}}$ & $0.9 \ensuremath{_{-0.2}^{+0.6}}$ \\
		H$\gamma$ & $1.7 \ensuremath{_{-0.4}^{+0.2}}$ &  $0.9 \ensuremath{_{-0.3}^{+0.3}}$ \\
		H$\delta$ & $1.7 \ensuremath{_{-0.4}^{+0.4}}$ & $1.5 \ensuremath{_{-0.4}^{+0.6}}$ \\
		He\,\textsc{i}\,5875 & $3.3 \ensuremath{_{-0.4}^{+0.7}}$ & $1.8 \ensuremath{_{-0.5}^{+0.7}}$ \\
		He\,\textsc{ii}\,4685 & $0.7 \ensuremath{_{-0.3}^{+0.1}}$ &  $0.9 \ensuremath{_{-0.4}^{+0.2}}$ \\
		O\,\textsc{i}\,8446 & $5.0 \ensuremath{_{-1.7}^{+2.1}}$  & -\\
		
		
		\hline
		\hline
	\end{tabular}
\end{table}


\begin{table}[!htb]
	\centering
	\caption{Centroid and Peak Time Lag for UVW2.}
	\label{tab:lags_UVW2_UV}
	\begin{tabular}{l c c}
		\hline
		\hline
		Name & $\tau_{\text{cent}}$ [d] &  $M_{\text{BH}} [10^7 M_\odot]$ \\
		\hline
		Ly$\alpha$ & $1.0 \ensuremath{_{-0.2}^{+0.4}}$ &  $0.7 \ensuremath{_{-0.2}^{+0.3}}$ \\
		Si\,\textsc{iv}\,1393 & $1.2 \ensuremath{_{-0.4}^{+0.1}}$ &  $4.4 \ensuremath{_{-1.7}^{+0.8}}$ \\
		C\,\textsc{iv}\,1548 & $0.8 \ensuremath{_{-0.0}^{+0.5}}$ &  $1.9 \ensuremath{_{-0.2}^{+1.7}}$ \\
		He\,\textsc{ii}\,1640 & $0.3 \ensuremath{_{-0.2}^{+0.6}}$ &  $0.5 \ensuremath{_{-0.3}^{+1.2}}$ \\
		
		\hline
		\hline
	\end{tabular}
\end{table}

\begin{figure}[!ht]
	\centering
	\includegraphics[width=0.9\textwidth]{pictures/Chapter4/lighcurves_and_ccfs_and_time_lag_tables/UVW2_ccfs_and_reference_lightcurves_optical.pdf}
	\caption{Compared lightcurves and CCFs H$\alpha$, H$\beta$, H$\gamma$, He\,\textsc{i}$\,\lambda5875$, He\,\textsc{ii}$\,\lambda4685$  and  O\,\textsc{i}$\,\lambda 8446$ with UVW2 as reference lightcurve.}
	\label{fig:ccfs_optical}
\end{figure}

\begin{figure}[!ht]
	\centering
	\includegraphics[width=0.9\textwidth]{pictures/Chapter4/lighcurves_and_ccfs_and_time_lag_tables/UVW2_ccfs_and_reference_lightcurves_not_optical.pdf}
	\caption{Compared lightcurves and CCFs of UV lines with UVW2 as reference lightcurve.}
	\label{fig:ccfs_UV}
\end{figure}



\begin{figure}[!ht]
	\centering
	\includegraphics[width=0.9\textwidth]{pictures/Chapter4/lighcurves_and_ccfs_and_time_lag_tables/OI_ccfs_and_reference_lightcurves_paper HAlpha.pdf}
	\caption{Compared lightcurves and CCFs for Bowen Fluorescence.}
	\label{fig:ccfs_Bowen}
\end{figure}



