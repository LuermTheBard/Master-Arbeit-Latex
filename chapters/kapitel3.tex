\chapter{Reverberation Mapping Analysis of NGC4593}
\label{cap: Results}

\section{Line Identification}



To begin the RM analysis, the emission lines in the AVG spectrum are identified. This is done for the optical–NIR range $3900\,\AA$–$9000\,\AA$  and for the UV range $1100\,\AA$–$1700\,\AA$.\\ 
In the optical–NIR range (see Figure \ref{fig:AVG_RMS_SPECTRUM}), several prominent broad emission lines are identified: Balmer lines from H$\alpha$ -- H$\epsilon$; He\,\textsc{i}$\,\lambda 4471$, He\,\textsc{i}$\,\lambda 5016$, He\,\textsc{i}$\,\lambda 5876$ and He\,\textsc{i}$\,\lambda 7065$; He\,\textsc{ii}$\,\lambda 4686$; and the low-ionization lines O\textsc{i}$\,\lambda8446$ and the Ca\,\textsc{ii}$\,\lambda8498$ $\lambda8542$\,$\lambda8662$ triplet. 
All of these lines show some level of variability in the RMS spectrum. For the subsequent analysis, the Balmer lines H$\alpha$ -- H$\delta$, He\,\textsc{ii}$\,\lambda 4686$ and He\,\textsc{i}$\,\lambda 5876$ are selected for the subsequent analysis, as they show the most pronounced variation in the RMS spectrum. The low-ionization line O\textsc{i}$\,\lambda8446$ shows noticeable variability, which was not monitored in a RM analysis yet and is therefore included as well. Two prominent Fe\,\textsc{ii} emission line bands are also present in the optical spectrum as well, one between $\sim 4489\,\AA$ -- $4629\,\AA$, blending with He\,\textsc{i}$\,\lambda 4471$ and He\,\textsc{ii}$\,\lambda 4686$, and another between $\sim 5169\,\AA$ -- $5336\,\AA$. The AVG spectrum also shows several narrow forbidden emission lines, with [O\,\textsc{iii}] $\lambda4363$, $\lambda4959$, and $\lambda5007$ being the most prominent. As they are assumed to be constant over the timescale of the campaign, they show no significant variability in the RMS spectrum. \\
In the UV spectrum (Figure \ref{fig:UV_uncalibrated_AVG_RMS}), five broad emission lines are identified: Ly$\alpha$, which overlaps with the N\,\textsc{v} $\lambda\lambda1238,\,1242$ doublet; Si\,\textsc{iv} $\lambda\lambda1393,\,1402$; C\,\textsc{iv} $\lambda\lambda1548,\,1550$; and He\,\textsc{ii}$\,\lambda1640$. Since the UV emission lines of NGC\,4593 were not monitored in a RM analysis yet, these emission lines will be included as well in the subsequent analysis. In addition to the broad emission lines, the UV spectrum exhibits semi-forbidden emission lines, like the O\,\textsc{iii}]\,$\lambda\lambda\, 1660,\,1666$ doublet, which partially blends with He\,\textsc{ii}$\,\lambda1640$.  
\begin{figure}[htbp]
 	\centering
 	\includegraphics[width=1\textwidth]{pictures/Chapter4/avg_rms_spec/avg_rms_spec.pdf}
 	\caption{Optical-to-NIR AVG and RMS spectrum with identified emission lines.}
 	\label{fig:AVG_RMS_SPECTRUM}
\end{figure}
\begin{figure}[htbp]
	\centering
	\includegraphics[width=\textwidth]{pictures/Chapter4/avg_rms_spec/UV_uncalibrated_AVG_RMS.pdf}
	\caption{UV spectrum AVG and RMS spectrum with identified emission lines}
	\label{fig:UV_uncalibrated_AVG_RMS}
\end{figure}




\section{Emission Line and Continua Measurement}

After the identification and selection of the broad emission lines, their fluxes are measured in every epoch to extract their light curves. This is performed with a Python-based tool called GECHO, developed by M.\,Probst. This tool is able to import full campaigns, determine AVG and RMS spectra, extracting lightcurves and conduct further measurements, naming here line-width measurements and lags measurement based on the methods of \cite{gaskell_peterson1986} and \cite{peterson2004}, which are further discussed in Sections \ref{sec:line_profiles} and \ref{sec:time_lag_bh_mass}.\\
The emission line light curve extraction of follows the same principle as in previous RM campaigns (e.g. \cite{kollatschny1997balmer, probst2025emissionlinecontinuumreverberationmapping}). Figure \ref{fig:gecho_example_line} shows the GECHO graphical user interface (GUI), which provides a side-by-side view of the campaign’s AVG and RMS spectra. The line flux at each epoch is obtained by integrating the flux density over the wavelength range, marked in red in Figure \ref{fig:gecho_example_line}. The integration boundaries are chosen to include the variable part of the emission line while excluding contributions from neighboring lines. To account for the surrounding continuum, a linear pseudo-continuum is interpolated from two line-free wavelength windows on the blue and red side of the emission line, shown in grey in Figure \ref{fig:gecho_example_line}. The adopted integration limits and pseudo-continua for each line are listed in Table \ref{tab:emission_lines}. 
\begin{figure}[!htbp]
	\centering
	\includegraphics[width=0.95\textwidth]{pictures/Chapter4/gecho_example/gecho_example_line}
	\caption{Screenshot of the GECHO graphical user interface (GUI). Shown is an example of the selection of the integration boundaries for H$\beta$ and corresponding blue and red pseudo-continuum used to measure the integrated line flux.}
	
	\label{fig:gecho_example_line}
\end{figure}\\
The light curves of the continuum emission, that will be used as proxy for the ionizing continuum, are derived from the mean flux density of areas free from line emission. The continuum windows for the light curve extraction of the continuum emission are adopted from \cite{cackett2018accretion}. The wavelength range for the continua are listed in Table \ref{tab:continua}.

\begin{table}[htbp]
	\centering
	\small
	\caption{Integration Limits and Pseudo-Continua range of the measured emission lines}
	\label{tab:emission_lines}
	\begin{tabular}{lcc}
		\hline
		\hline
		\textbf{Line} & \textbf{Integration Limits $[\AA]$} & \textbf{Pseudo-Continua $[\AA]$}  \\
		\hline
		\hline
		Ly$\alpha$ & $1207-1238$ & $1151-1161, 1461-1469$\\
		N\,\textsc{v}$\,\lambda\lambda 1238,\,1242$ & $1207-1238$ & $1151-1161, 1461-1468$\\
		Si\,\textsc{iv}$\,\lambda\lambda 1393,\,1402$ & $1358-1423$ & $1151-1161, 1461-1469$\\
		C\textsc{iv}$\,\lambda\lambda 1548,\,1550$ & $1511-1578$ & $1461-1469, 1680-1685$\\
		He\,\textsc{ii}$\,\lambda1640$ & $1599-1645$ & $1461-1468, 1680-1685$\\
		\hline
		H$\alpha$ & $6453-6695$ & $6107-6129, 6861-6900$ \\
		H$\beta$ & $4779-4944$ & $4762-4774, 5085-5112$ \\
		H$\gamma$ & $4230-4427$ & $4197-4220, 4435-4450$ \\
		H$\delta$ & $4035-4165 $ & $4026-4033, 4197-4220 $ \\
		
		He\,\textsc{i}$\,\lambda5875$ & $5742-6039$ & $5645-5653, 6044-6057$ \\
		He\,\textsc{ii}$\,\lambda4685$ & $4545-4758$ & $4435-4450, 4762-4774$ \\
		O\,\textsc{i}$\,\lambda 8446$ & $8380-8500$ & $8005-8031, 8850-8955$ \\
		\hline
		O\,\textsc{iii}$\,\lambda 5007$ & $4982-5033$ & $4762-4774, 5085-5112$ \\
		\hline
		\hline
	\end{tabular}
\end{table}

\begin{table}[htbp]
	\centering
	\small
	\caption{Wavelength range of the measured continua}
	\label{tab:continua}
	\begin{tabular}{lc}
		\hline
		\hline
		\textbf{Line} & \textbf{Integration Limits $[\AA]$}  \\
		\hline
		\hline
		Cont. 1150 & $1151-1161$\\
		\hline
		Cont. 4010 & $4026-4033$\\
		Cont. 4440 & $4435-4450$\\
		Cont. 5100 & $5085-5112$\\
		Cont. 6110 & $6107-6129$\\
		Cont. 6880 & $6861-6900$\\
		Cont. 8015 & $8005-8031$\\
		Cont. 8900 & $8864-8955$\\
		\hline
		\hline
	\end{tabular}
\end{table}








\begin{figure}[htbp]

	\centering
	\includegraphics[width=0.9\textwidth]{pictures/Chapter4/lightcurves/Continua.pdf}
	\caption{Comparison of the continua lightcurves. The first panel shows the photometric \textit{Swift} UVW2  light curve obtained from \cite{mchardy2018x}, while the other panels show the measured continua defined in Table \ref{tab:continua} }
	\label{fig:continua_lightcurves}
\end{figure}

\begin{figure}[htbp]
	\centering
	\includegraphics[width=0.9\textwidth]{pictures/Chapter4/lightcurves/Balmer_Lyman_and_O_lines.pdf}
	\caption{Comparison of the Balmer-line, Ly$\alpha$, and O,\textsc{i}$,\lambda8446$ light curves with the photometric \textit{Swift} UVW2  light curve as reference in the first panel. The \textit{Swift} UVW2 light curve is adopted from \cite{mchardy2018x}.}
	\label{fig:emission_line_lightcurves}
\end{figure}

\begin{figure}[htbp]
	\centering
	\includegraphics[width=0.9\textwidth]{pictures/Chapter4/lightcurves/He_and_UV_lines.pdf}
	\caption{Comparison of the Helium and UV light curves with the photometric \textit{Swift} UVW2  light curve as reference in the first panel. The \textit{Swift} UVW2 light curve is adopted from \cite{mchardy2018x}.}
	\label{fig:UV_emission_line_lightcurves}
\end{figure}
\clearpage
\section{Lightcurves}
\label{sec:lightcurves}

Thanks to the covered wavelength range, it was possible to measure several continuum lightcurves in the UV-, optical- and NIR regime. In addition, the photometric UVOT UVW2 lightcurve is included in the subsequent analysis, which was taken with \textit{Swift} and published in \cite{mchardy2018x}. It  has also been used in \cite{cackett2018accretion} and exhibits a denser sampling than the HST lightcurves, with a cadence of 96 minutes for $6.4\,\mathrm{d}$ between July 13 and July 18, 2016 and a central wavelength of about $\sim 1930\,\AA$ \parencite{mchardy2018x}.\\ 
All continuum lightcurves show a broadly similar shape (see Figure \ref{fig:continua_lightcurves}). The light curves start at high flux and then decrease by about $50$--$80\%$ relative to their maxima within the first 1–4 days, reaching a minimum between days 4 and 5. Afterwards, the flux increases again in all light curves. The UV continua show a plateau-like behavior from days 6 to 9, followed by a smaller decline between days 9 and 13, before rising again to a peak around day 13. A similar pattern is also noticeable in the other continuum light curves, but becomes less clear at higher wavelengths. While the optical continua also show higher flux levels between days 6 and 9 than between days 9 and 13, the near-IR continua exhibits a plateau over this interval. Towards the end of the campaign, the flux decreases again towards a minimum, with smaller short-term fluctuations, followed by a slight rise in the final days of the campaign in the UV and NIR bands. \\
The estimation of the physical distance between the SMBH and the region from which the emission lines originate is the main goal of this RM analysis. Assuming that the continuum radiation originates from the accretion disk, it is common to use the bluest available continuum as reference for the time lag estimation, which is expected to originate closest to the SMBH \parencite{peterson1993}. While the bluest continuum is measured  around $1150\,\AA$ in the HST dataset, the UVW2 continuum was selected as the main reference lightcurve for the subsequent analysis due to its higher sampling rate. Therefore, the delay between the continuum light curve around $1150\,\AA$ and the UVW2 light curve has to be taken into account in the subsequent lag estimation (see. Section \ref{sec:time_lag_bh_mass}).\\
Comparing the UVW2 light curve to the measured  broad emission line light curves, they exhibit a equally strong decrease in flux, followed by a central part with higher flux and another decrease (see. Figure \ref{fig:emission_line_lightcurves} and \ref{fig:UV_emission_line_lightcurves}). By comparing these features, a correlation of the emission line light curves to the UVW2 light curve is already apparent with a noticeable time lags which is further investigated in Section \ref{sec:time_lag_bh_mass}. 

\begin{table}[htp] 
	\centering 
	\caption{Variability statistics of the measured continua and emission lines with minimum flux $F_{\text{min}}$ and maximum flux density or integrated flux $F_{\text{max}}$, peak-to-peak ratio $R_{\text{max}}$, mean $\langle F \rangle$, standard deviation $\sigma_F$ and fractional variation $F_{\text{var}}$.} 
	\begin{tabular}{lrrrrrr} 
		\hline 
		\hline 
		\textbf{Continuum/Line} &  {$F_{\text{min}}$} &  {$F_{\text{max}}$} &  {$R_{\text{max}}$} &  {$\langle F \rangle$} &  {$\sigma_F$} &  {$F_{\text{var}}$} \\ 
		\hline
		\hline
		Cont. 1150  & $0.52$ & $1.35$ & $2.58$ & $0.86$ & $0.25$ & $0.28$ \\
		Cont. 4010  & $2.68$ & $4.21$ & $1.57$ & $3.49$ & $0.47$ & $0.14$ \\
		Cont. 4440  & $2.42$ & $3.73$ & $1.54$ & $3.14$ & $0.39$ & $0.12$ \\
		Cont. 5100  & $1.77$ & $2.77$ & $1.57$ & $2.29$ & $0.3$ & $0.13$ \\
		Cont. 6110  & $1.49$ & $2.27$ & $1.53$ & $1.9$ & $0.23$ & $0.12$ \\
		Cont. 6880  & $1.33$ & $2.01$ & $1.5$ & $1.72$ & $0.2$ & $0.11$ \\
		Cont. 8015  & $1.18$ & $1.69$ & $1.43$ & $1.48$ & $0.15$ & $0.1$ \\
		Cont. 8900  & $1.14$ & $1.52$ & $1.33$ & $1.38$ & $0.11$ & $0.08$ \\
		\hline 
		Ly$\alpha$  & $66.87$ & $94.88$ & $1.42$ & $82.21$ & $8.03$ & $0.1$ \\
		N\,\textsc{v}$\,\lambda\lambda 1238,\,1242$ & $18.87$ & $32.7$ & $1.73$ & $24.23$ & $3.54$ & $0.15$ \\
		Si\,\textsc{iv}$\,\lambda\lambda 1393,\,1402$  & $21.93$ & $35.5$ & $1.62$ & $27.92$ & $3.88$ & $0.14$ \\
		C\,\textsc{iv}$\,\lambda\lambda 1548,\,1550$ & $115.31$ & $165.57$ & $1.44$ & $138.77$ & $12.12$ & $0.09$ \\
		He\,\textsc{ii}$\,\lambda1640$  & $6.83$ & $21.29$ & $3.12$ & $13.29$ & $4.13$ & $0.31$ \\
		\hline 
		H$\alpha$  & $130.83$ & $149.85$ & $1.15$ & $141.35$ & $4.83$ & $0.03$ \\
		H$\beta$  & $38.29$ & $45.4$ & $1.19$ & $42.32$ & $1.94$ & $0.05$ \\
		H$\gamma$  & $18.82$ & $24.38$ & $1.3$ & $21.91$ & $1.29$ & $0.06$ \\
		H$\delta$  & $7.17$ & $10.92$ & $1.52$ & $9.13$ & $0.94$ & $0.1$ \\
		He\,\textsc{ii}$\,\lambda4685$  & $11.93$ & $17.7$ & $1.48$ & $14.82$ & $1.6$ & $0.11$ \\
		He\,\textsc{i}$\,\lambda5875$   & $8.53$ & $13.82$ & $1.62$ & $11.58$ & $1.01$ & $0.09$ \\
		O\,\textsc{i}$\,\lambda 8446$ & $7.47$ & $9.13$ & $1.22$ & $8.32$ & $0.37$ & $0.04$ \\
		\hline 
		\hline 
		\label{tab:varstatistics} 
	\end{tabular} 
\end{table}


\subsection{Variability Statistics}
To quantify the variability of the emission lines and continua variability statistics the definition by \cite{rodriguez1997steps} are adopted. They name the  maximum-to-minimum flux ratio $R_\mathrm{max}$ and the fractional variability	$F_\mathrm{var}$ as two common measures of variability. $R_\mathrm{max}$ is defined as the ratio of the the extreme of the integrated fluxes, $F_\mathrm{min}$ and $F_\mathrm{max}$, and  $F_\mathrm{var}$ as: 
\begin{equation}
	F_\mathrm{var}=\frac{\sqrt{\sigma_F^2-\Delta^2}}{\left<F\right>}
\end{equation}
Here, $\sigma_F^2$ denotes the standard deviation, $\left<F\right>$ the mean flux and $\Delta^2$ the mean square value of the flux uncertainties, which is defined by:
\begin{equation}
	\Delta^2 = \frac{1}{N} \sum_{i=1}^{N}\Delta_i^2
\end{equation}
The results for all parameters can be found in the Tables \ref{tab:varstatistics}.\\
The obtained $R_\mathrm{max}$ and $F_\mathrm{var}$ values of the continuum lightcurves, show a decreasing trend towards continua around higher wavelengths. While the UV continuum around $1150$ \AA exhibits the highest values for $R_\mathrm{max} = 2.58$ and $F_\mathrm{var} = 0.28$, they continuously drop towards the most red continua which shows values of $R_\mathrm{max} = 1.33$ and $F_\mathrm{var} \simeq 0.08$. \\
The lightcurves of the Balmer emission lines exhibit lower variability, which increases towards the higher-order Balmer lines, with values between $R_\mathrm{max} \simeq 1.15-1.52$ and $F_\mathrm{var} \simeq 0.03-0.1$. The helium lightcurves show a similar variability to H$\delta$ with values  between $R_\mathrm{max} \simeq 1.48-1.62$ and $F_\mathrm{var} \simeq 0.09-0.11$ and the OI$\,\lambda 8446$ a similar variability to the Balmer lines with $R_\mathrm{max} \simeq 1.22$ and $F_\mathrm{var} \simeq 0.04$.\\
The variability of the emission line lightcurves in the UV region is on a similar level as that of the helium light curves in the optical region, with $R_\mathrm{max} \simeq 1.42-1.62$ and $F_\mathrm{var} \simeq 0.14$, with the exception of the He\,\textsc{ii}$\,\lambda1640$ lightcurve, which shows significant higher variability with $R_\mathrm{max} \simeq 3.12$ and $F_\mathrm{var} \simeq 0.31$.


\subsection{Uncertainties Estimation}
The uncertainties of the continuum and emission-line light curves are estimated based on the spectral noise and a systematic uncertainty introduced by the intercalibration.\\
For the continuum light curves, the per-epoch noise is estimated as 
\begin{equation}
	\sigma_i^{\mathrm{cont}} = \frac{\sigma_{f_i}}{\sqrt{N}}, 
\end{equation} 
where $\sigma_{f_i}$ denotes the standard deviation of the flux density within the selected continuum window in epoch $i$, and $N$ is the number of pixels within that window. For the emission-line light curves, the noise is estimated from the scatter in the interpolated pseudo-continuum, $\sigma_{i}^{\mathrm{noise}}$ , as
\begin{equation}
	\sigma_{i}^{\mathrm{line}} = \frac{\sigma_{i}^{\mathrm{p.cont.}} \, \Delta\lambda}{\sqrt{N}},
\end{equation}
where $\Delta\lambda$ denotes the integration range of the emission line and $N$ is the number of pixels across the integration window.\\
As each epoch has been scaled to the flux of the narrow [O\,\textsc{iii}] $\lambda5007$ line after the intercalibration, the systematic uncertainty introduced by the intercalibration is estimated from its fractional variability, $F_{\mathrm{var}}^{\mathrm{[O\,\textsc{iii}]}\,\lambda5007}$:
\begin{equation}
	\sigma_i^{\mathrm{cal.}} = F_{\mathrm{var}}^{\mathrm{[O\,\textsc{iii}]}\,\lambda5007} \, f_i.
\end{equation}
The total uncertainty of the measured flux at epoch $i$, $f_i$, is then given with
\begin{equation}
	\sigma_i^\mathrm{tot} = \sqrt{\left(\sigma_i^{\mathrm{cal.}}\right)^2 + \left(\sigma_i^{\mathrm{cont}}\right)^2},
\end{equation}
for the continuum light curves and 
\begin{equation}
	\sigma_i^\mathrm{tot} = \sqrt{\left(\sigma_i^{\mathrm{cal.}}\right)^2 + \left(\sigma_{i}^{\mathrm{line}}\right)^2},
\end{equation}
for the emission line light curves. 



\section{Time Lag}
\label{sec:time_lag_bh_mass}
The time lag of the measured emission lines relative to the ionizing continuum is determined from the lag between the emission-line light curve and the UVW2 light curve. The time lag between two lightcurves is measured, by determining the cross-correlation function (CCF) of these curves and measuring the centroid of the CCF for values above 80\% of the peak, as described in Section \ref{subsec:rm_ccf}. This calculation is done with GECHO, which applies the interpolated cross-correlation function (ICCF) method introduced by \cite{gaskell_peterson1986}. The emission-line and reference light curves are normalized by subtracting the respective mean value from each data point. To increase the effective sampling and to measure the correlation between two unevenly sampled light curves, flux values between data points are linearly interpolated for both light curves.\\
The normalized light curves and the resulting CCFs are presented in Figures \ref{fig:ccfs_optical} and \ref{fig:ccfs_UV}. Additionally, the top panel of each figure displays the UVW2 light curve together with its auto-correlation function, while the bottom panel compares both lightcurves of the UVW2 and the UV continuum around $1150\,\AA$ and shows the corresponding CCF and time lag which has to be added for the final estimation of the characteristic BLR radii for each emission line. The resulting time lags towards the UVW2 continuum and the corresponding characteristic BLR radii are listed in Table \ref{tab:lags_UVW2}. The centroid uncertainties are estimated using the flux randomization/random subset selection (FR/RSS) method described by \cite{Peterson_1998b}, with 10000 iterations and a lag bin size of $0.5$ days (see Section \ref{subsec:rm_ccf}). The resulting centroid distribution is displayed by the gray histogram in the above named figures.\\
Overall, the CCFs of all measured emission lines show strong correlations with pronounced peak values between $\sim 0.7$ and $\sim 0.9$, except for the CCF of O\,\textsc{i}\,8446, which exhibits two local maxima at values of around $\sim 0.6$. \\
H$\alpha$ and H$\beta$ lag behind the UVW2 light curve by $3.3 \ensuremath{_{-0.6}^{+1.1}}$ and $2.3 \ensuremath{_{-0.3}^{+1.0}}$ days, respectively, while H$\gamma$ and H$\delta$ show shorter lags of $1.7 \ensuremath{_{-0.4}^{+0.2}}$ and $1.7 \ensuremath{_{-0.4}^{+0.4}}$ days. Therefore the lower order Balmer lines H$\alpha$ and H$\beta$ show higher time lags, that the higher ordered Balmer lines H$\gamma$ and H$\delta$. Ly$\alpha$ exhibits a even shorter time lag than the Balmer lines with $1.0 \ensuremath{_{-0.2}^{+0.4}}$ days within uncertainties, which implies that the hydrogen emitting region is stratified.\\
He\,\textsc{i}\,5875 shows a lag of $3.3 \ensuremath{_{-0.4}^{+0.7}}$ days, which is of the same order as the lags measured for H$\alpha$ and H$\beta$ within the uncertainties.  Opposing to that, the two higher ionized He\,\textsc{ii}\,1640 and He\,\textsc{ii}\,4685 helium lines, shows much shorter lags, with a with $0.3 \ensuremath{_{-0.2}^{+0.6}}$days and $0.7 \ensuremath{_{-0.3}^{+0.1}}$ days, respectively. 
The high ionized UV emission lines N\,\textsc{v}\,1238, Si\,\textsc{iv}\,1393 and C\,\textsc{iv}\,1548 also show short time lags of about $\sim 1$ compared to the lower ionized Balmer and He\,\textsc{i} lines. Therefore a tendency becomes apparent, that the higher ionized emission lines typical show shorter time lags than the lower ionized emission lines. \\
While the low-ionized O\,\textsc{i}\,8446 line exhibits and lag of $4.7 \ensuremath{_{-1.4}^{+2.4}}$, which is slightly longer then the lag of H$\alpha$ it should not be compared directly to UVW2, which is further discussed in section \ref{sec:BF_Analysis}.\\



\begin{table}[htbp]
	\centering
	\caption{Centroid between the emission lines and UVW2 light curve and the characteristic BLR radii with an included shift of 0.5 light days due to the delay between UVW2 and the UV continuum around 1150\,\AA.}
	\label{tab:lags_UVW2}
	\begin{tabular}{l c c}
		\hline\hline
		Line & $\tau_{\text{cent}}$ [d] & $R_{\mathrm{BLR}}$ [ld] \\
		\hline\hline
		H$\alpha$             & $3.3 \ensuremath{_{-0.6}^{+1.1}}$ & $3.8 \ensuremath{_{-0.6}^{+1.1}}$ \\
		H$\beta$              & $2.3 \ensuremath{_{-0.3}^{+1.0}}$ & $2.8 \ensuremath{_{-0.3}^{+1.0}}$ \\
		H$\gamma$             & $1.7 \ensuremath{_{-0.4}^{+0.2}}$ & $2.2 \ensuremath{_{-0.4}^{+0.2}}$ \\
		H$\delta$             & $1.7 \ensuremath{_{-0.4}^{+0.4}}$ & $2.2 \ensuremath{_{-0.4}^{+0.4}}$ \\
		Ly$\alpha$            & $1.0 \ensuremath{_{-0.2}^{+0.4}}$ & $1.5 \ensuremath{_{-0.2}^{+0.4}}$ \\
		\hline
		He\,\textsc{i}\,5875  & $3.3 \ensuremath{_{-0.4}^{+0.7}}$ & $3.8 \ensuremath{_{-0.4}^{+0.7}}$ \\
		He\,\textsc{ii}\,1640 & $0.3 \ensuremath{_{-0.2}^{+0.6}}$ & $0.8 \ensuremath{_{-0.2}^{+0.6}}$ \\
		He\,\textsc{ii}\,4685 & $0.7 \ensuremath{_{-0.3}^{+0.1}}$ & $1.2 \ensuremath{_{-0.3}^{+0.1}}$ \\
		\hline
		N\,\textsc{v}$\,\lambda\lambda 1238,\,1242$    & $1.0 \ensuremath{_{-0.4}^{+0.2}}$ & $1.5 \ensuremath{_{-0.4}^{+0.2}}$ \\
		Si\,\textsc{iv}$\,\lambda\lambda 1393,\,1402$  & $1.2 \ensuremath{_{-0.4}^{+0.1}}$ & $1.7 \ensuremath{_{-0.4}^{+0.1}}$ \\
		C\textsc{iv}$\,\lambda\lambda 1548,\,1550$     & $0.8 \ensuremath{_{-0.1}^{+0.5}}$ & $1.3 \ensuremath{_{-0.1}^{+0.5}}$ \\
		\hline
		O\,\textsc{i}\,8446   & $4.7 \ensuremath{_{-1.4}^{+2.4}}$ & $5.2 \ensuremath{_{-1.4}^{+2.4}}$ \\
		\hline\hline
	\end{tabular}
\end{table}

\begin{figure}[htbp]
	\centering
	\includegraphics[width=\textwidth]{pictures/Chapter4/lighcurves_and_ccfs_and_time_lag_tables/UVW2_ccfs_Balmer_Ly_O.pdf}
	\caption{Comparison of normalized lightcurves and CCFs of H$\alpha$, H$\beta$, H$\gamma$, He\,\textsc{i}$\,\lambda5875$, He\,\textsc{ii}$\,\lambda4685$  and  O\,\textsc{i}$\,\lambda 8446$ with UVW2 as reference lightcurve.}
	\label{fig:ccfs_optical}
\end{figure}

\begin{figure}[htbp]
	\centering
	\includegraphics[width=\textwidth]{pictures/Chapter4/lighcurves_and_ccfs_and_time_lag_tables/UVW2_ccfs_Helium_UV.pdf}
	\caption{Comparison of normalized lightcurves and CCFs of UV lines with UVW2 as reference lightcurve.}
	\label{fig:ccfs_UV}
\end{figure}


\section{Line Profiles}
\label{sec:line_profiles}
The velocity dispersion of the BLR can be parameterized by the line widths of the broad emission line \parencite{peterson2004} and is common in reverberation mapping campaigns(e.g. \cite{kollatschny1997balmer, denney2006ngc4593}). Therefore the line profiles of the emission lines are extracted from the mean and rms spectrum. Equally to the lightcurve extraction, a correction of the underlying continuum is necessary to enable comparison between the extracted profiles. Following the same procedure done for the lightcurves, an underlying linear continuum is interpolated, using line-free wavelength widows on the blue and red side for each emission line in both the mean and rms spectra. The selected boundaries of the pseudo-continua are listed in Table \ref{tab:line_profiles_pseudo}. By subtracting this interpolated continuum, a new zero-flux baseline for each line profile is defined. The flux is normalized to the maximum of each line profile and subsequently, the line profile is converted to velocity space using the relativistic Doppler equation \parencite{sher1968relativistic}\\ 
\begin{equation}
	\label{eqn:velocity_space}
	v_i = c \cdot \frac{\lambda_i^2 - \lambda_\mathrm{central}^2}{\lambda_i^2 + \lambda_\mathrm{central}^2}.
\end{equation}\\
Here, $\lambda_i$ denotes the wavelength values, $\lambda_\mathrm{central}$ the central wavelength of the emission line in rest-frame and $c$ the speed of light. For the doublets N\,\textsc{v}$\,\lambda\lambda 1283,\,1242 $, Si\,\textsc{iv}$\,\lambda\lambda 1393,\,1402 $ and C\,\textsc{iv}$\,\lambda\lambda 1548,\,1550$ the rest-frame wavelength of the second line gets used as the central wavelength of the emission line. An overview of the comparison of the mean and rms profiles for each emission line is displayed in Figure \ref{fig:line_profiles}. \\
Comparing the mean and rms line profiles, the maxima of the rms profiles of the Balmer lines, Ly$\alpha$ and He\,\textsc{i}$\,\lambda5875$ show a shift towards positive velocities compared to the mean profile. Therefore most parts of the rms profiles is located within the red wing of the mean profiles. This implies that most of the variability of these lines emerge from the receding region of their emitting region. Opposing to that, the maxima of the rms profiles of the higher ionized helium lines He\,\textsc{ii}$\,\lambda1640$ and 	He\,\textsc{ii}$\,\lambda4685$ as well as from the high ionized UV emission lines N\,\textsc{v}$\,\lambda\lambda 1283,\,1242$, Si\,\textsc{iv}$\,\lambda\lambda 1393,\,1402$ and C\,\textsc{iv}$\,\lambda\lambda 1548,\,1550$ are found at the same location as for the mean profiles. With the exception for N\,\textsc{v}$\,\lambda\lambda 1283,\,1242$ which is blended with the Ly$\alpha$ profile, a more blue weighted rms profile can be seen for these emission lines.\\\\
The mean line profiles includes narrow components in addition to \parencite{peterson1997introduction}. A decomposition from the broad component was not attempted, since it was not possible to isolate the narrow component. Therefore the rms line profiles are used in the subsequent analysis to examine the kinematics of the BLR, since their only include the variable broad component of the emission lines.\\



\begin{figure}[htbp]
	\centering
	\includegraphics[width=\textwidth]{pictures/Chapter4/line_profiles/Normalized_Line_Profiles.pdf}
	\caption{Normalized line profiles in the AVG and RMS spectra}
	\label{fig:line_profiles}
\end{figure}

\begin{table}[htbp]
	\centering
	\small
	\caption{Boundaries of the blue and red pseudo-continua used for the interpolation of underlying continua for line profile extraction.}
	\label{tab:line_profiles_pseudo}
	\begin{tabular}{lcc}
		\hline
		\hline
		\textbf{Line} & \textbf{\textbf{Pseudo-Continua (avg) $[\AA]$}} & \textbf{Pseudo-Continua (rms) $[\AA]$}  \\
		\hline
		\hline
		H$\alpha$ & $6194-6216, 6861-6900$ & $6279-6301, 6742-6781$ \\
		H$\beta$ & $4762-4774, 5085-5112$ & $4762-4774, 4967-4984$ \\
		H$\gamma$ & $4197-4220, 4435-4450$ & $4197-4220, 4417-4429$ \\
		H$\delta$ & $4026-4033, 4197-4221$ &$4006-4016, 4197-4211$  \\
		\hline
		O\,\textsc{i}$\,\lambda 8446$ & $7999-8025, 8775-8798$ & $8222-8238, 8748-8767$ \\
		\hline
		He\,\textsc{i}$\,\lambda5875$ & $5679-5697, 6044-6057$ & $5736-5753, 6027-6045$ \\
		He\,\textsc{ii}$\,\lambda1640$  & $1461-1468, 1679-1685$ & $1461-1468, 1679-1685$\\
		He\,\textsc{ii}$\,\lambda4685$ & $4198-4225, 4762-4774$ & $4543-4554, 4766-4778$ \\
		\hline
		Ly$\alpha$ & $1151-1161, 1270-1285$  &  $1151-1161, 1340-1355$\\
		N\,\textsc{v}$\,\lambda\lambda 1283,\,1242 $ & $1151-1161, 1270-1285$  &  $1151-1161, 1340-1355$\\
		Si\,\textsc{iv}$\,\lambda\lambda 1393,\,1402 $ & $1350-1360, 1430-1440$ &  $1340-1355, 1430-1440$\\
		C\,\textsc{iv}$\,\lambda\lambda 1548,\,1550$ & $1461-1468, 1679-1685$  &  $ 1461-1468, 1679-1685$\\
		
		\hline
		\hline
	\end{tabular}
\end{table}
\clearpage

\subsection{FWHM estimation}
Equally to earlier RM studies (e.g. \cite{kollatschny1997balmer, denney2006ngc4593, probst2025emissionlinecontinuumreverberationmapping}), the line width of the line profiles is estimated by their Full Width Half Maximum (FWHM), which is conducted with gecho. For this the height of the profiles is defined between the profiles maxima and the zero-flux base line. The mean profile of He\,\textsc{ii}$\,\lambda4685$ is blended with Fe\,\textsc{ii} emission in the blue wing, which makes a measurement of its FWHM not possible. Instead, the FWHM is estimated based on the width of its red wing, which is consistent with the order of magnitude of the FWHM of He\,\textsc{ii}$\,\lambda1640$ within uncertainties. \\
The FWHM of profiles of O\,\textsc{i}$\,\lambda 8446$, N\,\textsc{v}$\,\lambda\lambda 1283,\,1242 $, and Si\,\textsc{iv}$\,\lambda\lambda 1393,\,1402 $ were not measured due to following reasons. \\
The mean profile of the NIR emisson line O\,\textsc{i}$\,\lambda 8446$ is blended with the Ca\,\textsc{ii} $\lambda8498$, $\lambda8542$, $\lambda8662$ triplet. In a earlier study, \cite{Ochmann_2025} was already able to decompose the line profile O\,\textsc{i}$\,\lambda 8446$ based on a MUSE campaign in 2019, which is why this was not attempted in this analysis. While O\,\textsc{i}$\,\lambda 8446$ exhibits variation in the rms spectrum, its profile is to noisy for a FWHM measurement due to the low signal to noise ratio.\\
The profiles of N\,\textsc{v}$\,\lambda\lambda 1283,\,1242 $ is partly blended with the Ly$\alpha$ profile in both mean and rms and  in addition exhibits absorption in its blue wing in the rms spectrum. Therefore it is not possible to measure reliable FWHM for the N\,\textsc{v}$\,\lambda\lambda 1283,\,1242 $ profiles.\\
Finally, the profiles of Si\,\textsc{iv}$\,\lambda\lambda 1393,\,1402 $ are blended with the semi-forbidden line doublet O\textsc{IV}]$\,\lambda\lambda\,1397,\,1400$ and the FWHM of the profiles would therefore include this components in addition to the Si\,\textsc{iv}$\,\lambda\lambda 1393,\,1402 $ component. \\
The resulting FWHM values are listed in Table \ref{tab:line_width_FWHM}. The uncertainties are estimated based on the instrumental dispersion of the gratings (see Table \ref{tab:stis_gratings}). The uncertainties are estimated primarily on the instrumental dispersion of the gratings (see Table \ref{tab:stis_gratings}) and selected higher accordingly the noise and shape of the profiles.\\ 


\begin{table}[htbp]
	\centering
	\small
	\caption{Measured FWHM and of the AVG and RMS line profiles.}
	\label{tab:line_width_FWHM}
	\begin{tabular}{lcc}
		\hline
		\hline
		\textbf{Line} & FWHM (avg)[km/s] & FWHM (rms)[km/s]   \\
		\hline
		\hline
		H$\alpha$  & $3000 \pm 300$ &  $3100\pm 300$\\
		H$\beta$ & $3400 \pm 300$&  $3400 \pm 300$ \\
		H$\gamma$ & $4100 \pm 300$&  $3900 \pm 400$ \\
		H$\delta$ & $3400 \pm 300$&  $5900 \pm 1000$ \\
		Ly$\alpha$ & $3800 \pm 350$  &  $4600\pm 500$\\
		\hline
		He\,\textsc{i}$\,\lambda5876$ & $3500 \pm 300$&  $4000 \pm 400$  \\
		He\,\textsc{ii}$\,\lambda1640$  & $2900 \pm 500$ & $6900 \pm 1000$\\
		He\,\textsc{ii}$\,\lambda4686$&  $1800 \pm 1000$&  $6000 \pm 400$ \\
		
		\hline
		%N\,\textsc{v}$\,\lambda\lambda 1238,\,1242$ &  $3200 \pm 1000$&  $3400 \pm 1000$\\
		%Si\,\textsc{iv}$\,\lambda\lambda 1393,\,1402$ &  $5200 \pm 500$&  $10000 \pm 1500$\\
		C\textsc{iv}$\,\lambda\lambda 1548,\,1550$ & $ 5400\pm 500$  &  $ 8400 \pm 1000$\\
		
		\hline
		\hline
	\end{tabular}
\end{table}

\subsection{Comparison of the rms line profiles}


The rms profiles of the Balmer line H$\alpha$ -- H$\delta$ show a asymmetric double peaked shape with higher flux in the blue peak (see. \ref{fig:RMS_Balmer}). They show strong similarities, with the double-peaks located around $\sim 500\,\mathrm{km\,s^{-1}}$ and $\sim 2000\,\mathrm{km\,s^{-1}}$ for every profile. While the profiles of H$\alpha$ and H$\beta$ exhibiting similar steep slope in both wings as well, the H$\gamma$ and H$\delta$ profiles show more extended blue wings. Therefore the FWHM shows a increasing tendency towards the higher ordered Balmer lines with $3100\pm 300\,\mathrm{km\,s^{-1}}$ and $3400 \pm 300\,\mathrm{km\,s^{-1}}$ for H$\alpha$ and H$\beta$ and $3900 \pm 400\,\mathrm{km\,s^{-1}}$ and $5900 \pm 1000\,\mathrm{km\,s^{-1}}$, respectively. \\
%The rms profile of Ly$\alpha$ exhibits a similar shift towards positive velocities compared to the Balmer lines, with a comparable broad peak located between approximately $1000\,\mathrm{km\,s^{-1}}$ and $2000\,\mathrm{km\,s^{-1}}$. It shows absorption in both wings and a FWHM was found with $4600\pm 500\,\mathrm{km\,s^{-1}}$ and therefore shows similar velocities than the higher order Balmer lines.
\begin{figure}[htbp]
	\centering
	\includegraphics[width=1\textwidth]{pictures/Chapter4/line_profiles/RMS_overlay_Balmer.pdf}
	\caption{Comparison of the normalized RMS line profiles of the Balmer lines H$\alpha$, H$\beta$, H$\gamma$ and H$\delta$.}
	\label{fig:RMS_Balmer}
\end{figure}\\
Comparing the rms profiles of the helium lines, differences in shape become apparent depending on the line ionization (see. Figure \ref{fig:RMS_Helium}). 

\textbf{Compare helium lines}



%\subsection{UV Line Profiles}

%Figure \ref{fig:RMS_UV} shows the RMS profiles of the Ly$\alpha$ line, the N\,\textsc{v}$\,\lambda\lambda 1238,\,1242$ doublet, the  Si\,\textsc{iv}$\,\lambda\lambda 1393,\,1402$ doublet and the 	C\textsc{iv}$\,\lambda\lambda 1548,\,1550$ doublet. Starting with  Ly$\alpha$, its RMS profile is blended with line absorption in its blue flank, as well as with the N\,\textsc{v}$\,\lambda\lambda 1238,\,1242$ doublet in his red wing. Still, its central part is good distinguishable. Similar to the Balmer lines, the most parts, as well as the highest flux is shifted towards positive velocities. It shows a steep blue wing and a slightly broader red wing, which a small broadening at about half height, resulting in a FWHM of about $\sim 4600 \pm 350 \mathrm{km\,s^{-1}}$. \\
%The RMS profile of the neighboring N\,\textsc{v}$\,\lambda\lambda 1238,\,1242$ doublet is blended with absorption as well as with the Ly$\alpha$ RMS profile. This makes its RMS profile difficult to distinguish. Still it was attempted to estimate its FWHM by taking the width of the red wing of the central peak, resulting in an FWHM of about $3400 \pm 1000 \mathrm{km\,s^{-1}}$.\\ The RMS profile Si\,\textsc{iv}$\,\lambda\lambda 1393,\,1402$ doublet shows a very broad and scattered profile and is blended with the semi-forbidden line doublet O\textsc{IV}]$\,\lambda\lambda\,1397,\,1400$. This leads to a very broad FWHM of about $\sim 10000 \pm 1000 \mathrm{km\,s^{-1}}$. Due to the noise profile and because of a possible interference with the O\textsc{IV}]$\,\lambda\lambda\,1397,\,1400$ doublet, the measured FWHM value will not taken into account for the subsequent analysis.\\
%Finally, the RMS profile of the C\textsc{iv}$\,\lambda\lambda 1548,\,1550$ doublet shows a very asymmetric profile with a broad plateau like blue wing, with flux staying below half height of the maximum until about $-3000\mathrm{km\,s^{-1}}$. The centre of the RMS profile is shows a double-peak-like shape with a narrower blue peak and a broader red peak. They are located at about $\sim -1500\mathrm{km\,s^{-1}}$ and $250\mathrm{km\,s^{-1}}$, with higher flux in the red peak. Between the peaks the flux drops below half of the maximum, due to blending with absorption, which is why the left wing of the blue peak and the right wing of the red peak has been used to measure the FWHM with $\sim 8400 \pm 1000$. 


\begin{figure}[htbp]
	\centering
	\includegraphics[width=\textwidth]{pictures/Chapter4/line_profiles/RMS_overlay_Helium.pdf}
	\caption{Comparison of the normalized RMS line profiles of the Helium lines}
	\label{fig:RMS_Helium}
\end{figure}

%\begin{figure}[htbp]
%	\centering
%	\includegraphics[width=0.75\textwidth]{pictures/Chapter4/line_profiles/UV_Lines.pdf}
%	\caption{Comparison of the normalized RMS line profiles of the UV lines}
%	\label{fig:RMS_UV}
%\end{figure}






\section{Black Hole Mass}

The mass of the SMBH can be estimated by deriving the virial mass for each measured emission line and adjusting it with a scale factor f \parencite{peterson2004}. A scale factor of $f=1.8$ is assumed, since FWHM of the RMS profiles is used to parameterize the velocity dispersion $\Delta V$, following \cite{probst2025emissionlinecontinuumreverberationmapping} (see Section \ref{subsec:BHM}). O\,\textsc{i}$\,\lambda8446$, N\,\textsc{v}$\,\lambda\lambda 1238,\,1242$ and Si\,\textsc{iv}$\,\lambda\lambda 1238,\,1242$ are excluded in the mass estimation, since they could not be measured reliably.\\
Taking equation \ref{eqn:charc_radius}, \ref{eqn:M_vir} and \ref{eqn:M_BH} the SMBH mass can be estimated based on the time lag and FWHM of for each emission line using: 
\begin{equation}
	\label{eqn:BHM}
	M_{\mathrm{BH}} = 1.8 \cdot \frac{c \cdot \tau_{\mathrm{centroid}}\,\mathrm{FWHM}^2}{G}\,.
\end{equation}
The resulting SMBH mass results, as well as the corresponding time lag and FWHM values of the broad emission lines are listed in Table \ref{tab:BH_mass}.
\begin{table}[htp]
	\centering
	\small
	\caption{Estimated characteristic BLR radii, FWHM and SMBH masses.}
	\label{tab:BH_mass}
	\begin{tabular}{l c c c}
		\hline
		\hline
		\textbf{Line} & $R_{\mathrm{BLR}}$ [ld] & FWHM (rms)[km/s] & $M_{\text{BH}} [10^7 M_\odot]$ \\
		\hline
		\hline
		Ly$\alpha$ & $1.5 \ensuremath{_{-0.2}^{+0.4}}$ & $4600 \pm 500$ &$1.2 \ensuremath{_{-0.2}^{+0.3}}$ \\
		H$\alpha$ & $3.8 \ensuremath{_{-0.6}^{+1.1}}$ & $3100 \pm 250$ &$1.3 \ensuremath{_{-0.2}^{+0.4}}$ \\
		H$\beta$ & $2.8 \ensuremath{_{-0.3}^{+1.0}}$ & $3400 \pm 200$&$1.2 \ensuremath{_{-0.2}^{+0.5}}$ \\
		H$\gamma$ & $2.2 \ensuremath{_{-0.4}^{+0.2}}$ & $3900 \pm 300$ &$1.2 \ensuremath{_{-0.3}^{+0.2}}$ \\
		H$\delta$ & $2.2 \ensuremath{_{-0.4}^{+0.4}}$ & $5900 \pm 1000$  & $1.9 \ensuremath{_{-0.4}^{+0.4}}$ \\
		He\,\textsc{i}$\,\lambda5876$ & $3.8 \ensuremath{_{-0.4}^{+0.7}}$ & $ 4000 \pm 300$&$2.1 \ensuremath{_{-0.3}^{+0.4}}$ \\
		He\,\textsc{ii}$\,\lambda1640$ & $0.8 \ensuremath{_{-0.2}^{+0.6}}$ & $6900 \pm 1000$ & $1.3 \ensuremath{_{-0.3}^{+1.0}}$ \\
		He\,\textsc{ii}$\,\lambda4686$ & $1.2 \ensuremath{_{-0.3}^{+0.1}}$ & $6000 \pm 300$ &$1.6 \ensuremath{_{-0.4}^{+0.2}}$ \\
		C\,\textsc{iv}$\,\lambda\lambda 1548,\,1550$ & $1.3 \ensuremath{_{-0.1}^{+0.5}}$ & $8400 \pm 1000$ &$3.2 \ensuremath{_{-0.1}^{+1.3}}$ \\
		\hline
		\hline
	\end{tabular}
\end{table}\\



To obtain a final estimation for the SMBH mass, a inverse-variance weighted mean is calculated. Since the uncertainties are asymmetric the higher gets adopted, $\sigma_i=\mathrm{max}(\sigma_i^-,\sigma_i^+)$.
The weighted mean is then given by
\begin{equation}
	\bar{M}_{\mathrm{BH}}=\frac{\sum_i w_i M_{\mathrm{BH},i}}{\sum_i w_i}\,,\qquad
	w_i=\frac{1}{\sigma_i^2}\,,
\end{equation}
with an uncertainty
\begin{equation}
	\sigma_{\bar{M}}=\left(\sum_i w_i\right)^{-1/2}\,.
\end{equation}
This gives a weighted-mean SMBH mass of $\bar{M}_{\mathrm{BH}}\approx (1.40 \pm 0.12)\times 10^{7}\,M_\odot$.\\\\
It has to be noted, that the scale factor of $f=1.8$ does not account the low-inclination $i \sim 11^\circ$ of the elliptic accretion disk, modeled in \cite{Ochmann_2025}. Adopting the scaling relation $f \sim \sin^{-2}(i)$ introduced by \cite{krolik2001systematic}, a value of $f \sim 27.5$ would be needed to account for the low inclination. \cite{krolik2001systematic} used the line dispersion to parameterize the velocity dispersion, which is why the relation $\sigma_{\mathrm{line}} \approx \mathrm{FWHM}/2$ \parencite{peterson2004} has to be accounted again. Due to the square relation of the velocity dispersion to the black hole mass seen in Equation \ref{eqn:BHM}, the scale factor finally gets corrected to a value $f \sim 6.8$. Subsequently the found BH mass has to be multiplied by a additional factor of $3.77$. Subsequently, the weighted-mean SMBH mass can be estimated to be $\bar{M}_{\mathrm{BH}} \approx (5.28 \pm 0.12)\times 10^{7}\,M_\odot$.



%\begin{table}[htp]
%	\centering
%	\small
%	\caption{Estimated time lags, FWHM and SMBH masses.}
%	\label{tab:BH_vmass}
%	\begin{tabular}{l c}
%		\hline
%		\hline
%		\textbf{Line} & $\tau_{\text{cent}}$ [d] \\
%		\hline
%		\hline
%		H$\alpha$ & $3.3 \ensuremath{_{-0.6}^{+1.1}}$ \\
%		H$\beta$ & $2.3 \ensuremath{_{-0.3}^{+1.0}}$ \\
%		H$\gamma$ & $1.7 \ensuremath{_{-0.4}^{+0.2}}$ \\
%		H$\delta$ & $1.7 \ensuremath{_{-0.4}^{+0.4}}$ \\
%		Ly$\alpha$ & $1.0 \ensuremath{_{-0.2}^{+0.4}}$\\
%		\hline
%		He\,\textsc{i}$\,\lambda5876$ & $3.3 \ensuremath{_{-0.4}^{+0.7}}$ \\
%		He\,\textsc{ii}$\,\lambda1640$ & $0.3 \ensuremath{_{-0.2}^{+0.6}}$ \\
%		He\,\textsc{ii}$\,\lambda4686$ & $0.7 \ensuremath{_{-0.3}^{+0.1}}$ \\
%		\hline
%		N\,\textsc{v}$\,\lambda\lambda 1238,\,1242$ & $1.0 \ensuremath{_{-0.4}^{+0.2}}$  \\
%		Si\,\textsc{iv}$\,\lambda\lambda 1393,\,1402$  & $1.2 \ensuremath{_{-0.4}^{+0.1}}$ \\
%		C\,\textsc{iv}$\,\lambda\lambda 1548,\,1550$ & $0.8 \ensuremath{_{-0.1}^{+0.5}}$ \\
%		\hline
%		\hline
%	\end{tabular}
%\end{table}

\section{Bowen Fluorescence of O\,\textsc{I}$\,\lambda 8446$}
\label{sec:BF_Analysis}

Emission lines in the optical range, like the Balmer and helium lines, are commonly used in many reverberation-mapping campaigns, as they are easily accessible for nearby AGN \parencite{peterson1991steps, kollatschny1997balmer}. This is not the case for the low-ionization line O\,\textsc{i}$\,\lambda 8446$, which makes the emission line of particular interest in this analysis, since it exhibits noticeable variability (see. Table \ref{tab:varstatistics}) and therefore can be monitored in a RM analysis for the first time. O\,\textsc{i}$\,\lambda 8446$ is also know as a Bowen fluorescence line, that can be pumped by Ly$\beta$ emission \parencite{grandi1980}. As described in Section \ref{sec:bowen_fluorescence}, Ly$\beta$ excites O\,\textsc{i} through a near-resonant transition at $\lambda\,1025\,\AA$. This is followed by decay via O\,\textsc{i}\,$\lambda\,11287$ and subsequently by O\,\textsc{i}\,$\lambda\,8446$, producing the observed O\,\textsc{i}\,$\lambda\,8446$ emission (see Figure \ref{fig:bowen_cascate}). This process was identified as the dominant excitation mechanism of O\,\textsc{i}$\,\lambda 8446$ in many other AGN (e.g. \cite{rudy1989detection, rodriguez2002oi, landt2008near}), showing that the variation in O\,\textsc{i}\,$\lambda\,8446$ is caused by L$\beta$ Bowen fluorescence. This also can be assumed for NGC\,4593, since it is possible to rule-out recombination as a significant excitation mechanism of O\,\textsc{i}$\,\lambda 8446$ through analyzing line ratios. If recombination would drive the O\,\textsc{i}$\,\lambda 8446$ emission, a relative strength between O\,\textsc{i}$\,\lambda 8446$ and the quintet line O\,\textsc{i}$\,\lambda 7774$ of $\lambda7774/\lambda8446 \approx 1.1-1.7$ should be detectable \cite{grandi1980, landt2008near}.  While O\,\textsc{i}$\,\lambda 7774$ is not detectable in the HST dataset, it is possible to determine a line ratio $\lambda7774/\lambda8446 = 0.2$ based on the Muse spectrum of NGC\,4593 presented by \cite{Ochmann_2025}, whic was taken in 2019. Therefore it can be assumed that the O\,\textsc{i}$\,\lambda 8446$ variation in NGC\,4593 is driven through Bowen fluorescence, which enables to examine the Bowen fluorescence relation of O\,\textsc{i}$\,\lambda 8446$ for the first time in a RM analysis. \\
While Ly$\beta$ is not covered in the spectral range of the HST dataset, it is possible to use Ly$\alpha$ as a proxy for Ly$\beta$, since it is expected that both lines arise under similar physical conditions. Therefore, the time lags between the UVW2 and emission-line lightcurves, as well as the CCFs and time lags between O\,\textsc{i}$\,\lambda 8446$ and Ly$\alpha$, between H$\alpha$ and Ly$\alpha$, and between O\,\textsc{i}$\,\lambda 8446$ and H$\alpha$, are calculated and shown in Figure \ref{fig:ccfs_Bowen}. All lightcurve show a strong correlation between $\sim 0.75 - 0.8$, with the exception of the O\,\textsc{i}$\,\lambda 8446$ and UVW2 lightcurves, which show a maximum correlation of only $0.6$. The time lags and their uncertainties are calculated as before. O\,\textsc{i}$\,\lambda 8446$ lags behind Ly$\alpha$ by $2.5 \ensuremath{_{-0.3}^{+1.7}}$ days and behind UVW2 by $4.7 \ensuremath{_{-1.4}^{+2.4}}$ days, while H$\alpha$ lags behind Ly$\alpha$ by $1.8 \ensuremath{_{-0.5}^{+0.5}}$ days and behind UVW2 by $3.3 \ensuremath{_{-0.6}^{+1.1}}$ days. Therefore the direct relation of O\,\textsc{i}$\,\lambda 8446$ to UVW2 shows a slightly higher lag than H$\alpha$ to UVW2 within uncertainties. Assuming Bowen fluorescence as the dominant excitation mechanism the lag of O\,\textsc{i}$\,\lambda 8446$ should be estimated following the Bowen fluorescence path. With the Ly$\alpha$ light curve lagging behind the UVW2 light curve by $1.0 \ensuremath{_{-0.2}^{+0.4}}$ days, this suggests a lag along the Bowen-fluorescence path from UVW2 to Ly$\alpha$ and from Ly$\alpha$ to O\,\textsc{i}$\,\lambda 8446$ sums to $\sim 3.5$ days. Therefore, O\,\textsc{i}$\,\lambda 8446$ shows a lag much closer and consistent with the lag of H$\alpha$ within uncertainties.  Looking at the O\,\textsc{i}$\,\lambda 8446$ and H$\alpha$ lightcurves, they also show a strong correlation, with O\,\textsc{i}$\,\lambda 8446$ lagging behind H$\alpha$ by about $0.3 \ensuremath{_{-0.1}^{+1.6}}$ days, which is consistent with the lag found for the Bowen fluorescence path. 


\begin{figure}[htbp]
	\centering
	\includegraphics[width=0.98\textwidth]{pictures/Chapter4/lighcurves_and_ccfs_and_time_lag_tables/OI_ccfs_and_reference_lightcurves_paper HAlpha.pdf}
	\caption{Comparison of normalized lightcurves and CCFs for Bowen Fluorescence.}
	\label{fig:ccfs_Bowen}
\end{figure}

