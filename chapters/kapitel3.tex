\chapter{Reverberation Mapping Analysis of NGC4593}
\label{cap: Results}

\section{Line Identification}

Having obtained the AVG- and RMS spectrum of NGC4593, the next step is the identification of the emission lines. Figures  \ref{fig:AVG_RMS_SPECTRUM} and \ref{fig:UV_uncalibrated_AVG_RMS} show the optical to near-infrared range between $3900 \AA$ and $9000 \AA$ and  the UV range between $1100 \AA$ and $1700 \AA$, respectively. 

\begin{figure}[!htbp]
	\centering
	\makebox[\textwidth][c]{%
		\includegraphics[width=1.2\textwidth]{pictures/Chapter4/avg_rms_spec/avg_rms_spec.pdf}}
	\caption{Optical-to-NIR AVG and RMS spectrum with identified emission lines.}
	\label{fig:AVG_RMS_SPECTRUM}
\end{figure}

\newpage

Looking at Figure \ref{fig:AVG_RMS_SPECTRUM}, the most prominent broad emission lines in the AVG spectrum are the Balmer-Lines, with H$\alpha$ being the strongest followed by H$\beta$ and H$\gamma$. Their variations are clearly visible in the RMS spectrum and the line profiles of H$\alpha$ and H$\beta$ in particular show strong similarities, which will be discussed in more detail later in this chapter. In addition to the Balmer emission lines, several He-emission lines such as He\,II $\lambda 4685$, He\,I $\lambda 5875$ and He\,I $\lambda 7065$ show  variability. Another significant broad emission line complex appears in the NIR part of the spectrum, including the O\,I $\lambda 8446$ and Ca\,II lines. The O\,I $\lambda 8446$ line is of particular interest in this thesis, as it shows variability that has never been measured using reverberation mapping before. \\
Apart from the broad emission lines, the AVG spectrum also exhibits several strong forbidden emission lines with the [O\,III] $\lambda 5007$ and the [O\,III] $\lambda 4958$ as the most prominent ones. As mentioned in the previous chapter, the first one was used for the intercalibration of the spectra and, as expected, shows no variability in the RMS spectrum, similar to the other forbidden lines. Finally, the AVG spectrum shows additionally two Fe II emission line groups, that shows no variation in the RMS spectrum.

\begin{figure}[!htbp]
	\centering
	\makebox[\textwidth][c]{%
		\includegraphics[width=1.2\textwidth]{pictures/Chapter4/avg_rms_spec/UV_uncalibrated_AVG_RMS.pdf}}
	\caption{UV spectrum AVG and RMS spectrum with identified emission lines}
	\label{fig:UV_uncalibrated_AVG_RMS}
\end{figure}


The UV spectrum shown in Figure \ref{fig:UV_uncalibrated_AVG_RMS} contains both broad and semi-forbidden emission lines. In addition to the prominent Si and C broad emission lines, it shows strong and variable Ly$\alpha$ emission, which is of particular significance for this thesis. \\
The Bowen fluorescence of O\,I $\lambda 8446$, which is investigated in this analysis, is typical driven by the emission of the Ly$\beta$ line \parencite{grandi1980}, which lies outside the spectral range of the HST Campaign. However, as can be seen in figure \ref{fig:UV_uncalibrated_AVG_RMS}, Ly$\alpha$  still lies in the spectral range of the campaign, as the bluest broad line of the taken spectra. As Ly$\alpha$ and Ly$\beta$ are assumed to originate in the same physical region, Ly$_\alpha$ can be used to examine the corelation of O\,I $\lambda 8446$ and Ly$\alpha$ for Bowen fluorescence.
 
 




\section{Emission Line and Continua Measurement}

After identifying the emission lines, the next step of the analysis is to calculate the fluxes of those lines. This is done with the help of a python-based tool called GECHO created by M. Probst. It enables to import full campaigns, determine the AVG and RMS spectra, extracting lightcurves and do further measurments, like e.g. ICCF methode to measure lags and line-wifth measurments \parencite{gaskell_peterson1986, peterson2004}, which will be further discussed in Section \ref{sec:time_lag_bh_mass}\\
With help of GECHO the flux density gets integrated over the extent of each emission line for every observed spectrum of the campaign. Here it is important to define the integration limits so that all of the emission line flux is measured in this way. To ensure this, a parallel view of the AVG- and RMS-spectrum was used to define those integration limits, and to ensured, that no component of any other line is contributing to the line flux of the measured line. But before the line flux can be calculated, the surrounding continuum has to be subtracted, which can be done by interpolating a linear underlying continuum between sections on the blue and red side with no line contribution. The chosen integration limits and pseudo-continua can be found in Table \ref{tab:emission_lines}. With the fluxes of each line now derived from all observed spectra, it is possible to extract the lightcurves of the measured emission line, which will be further covered in the next section.\\
Besides the emission line lightcurves, continua lightcurves from different wavelength ranges are required for the further analysis. The extraction process is similar to the emission line lightcurves, except that they get calculated by the mean flux density of a sufficiently large region without line emission or absorption. Therefore, no pseudo-continuum subtraction is necessary. The chosen integration limits for the continua can be found in Table \ref{tab:continua}.\\

\begin{table}[h!]
	\centering
	\small
	\caption{Integration Limits and Pseudo-Continua of the measured emission lines}
	\label{tab:emission_lines}
	\begin{tabular}{lcc}
		\hline
		\hline
		\textbf{Line} & \textbf{Integration Limits $[\AA]$} & \textbf{Pseudo-Continua $[\AA]$}  \\
		\hline
		Ly$\alpha$ & $1207-1238$ & $1151-1161, 1462-1468$\\
		\hline
		H$\alpha$ & $6520-6634$ & $6107-6129, 6861-6900$ \\
		H$\beta$ & $4828-4924$ & $4762-4774, 5085-5112$ \\
		H$\gamma$ & $4317-4391$ & $4197-4220, 4435-4450$ \\
		
		HeI$\,\lambda5875$ & $5840-5941$ & $5645-5653, 6044-6057$ \\
		HeII$\,\lambda4685$ & $4610-4744$ & $4435-4450, 4762-4774$ \\
		OI$\,\lambda 8446$ & $8380-8498$ & $8005-8031, 8850-8955$ \\
		\hline
		OIII$\,\lambda 5007$ & $4982-5033$ & $4762-4774, 5085-5112$ \\
		\hline
		\hline
	\end{tabular}
\end{table}


\begin{table}[h!]
	\centering
	\small
	\caption{Integration Limits of the measured continua}
	\label{tab:continua}
	\begin{tabular}{lc}
		\hline
		\hline
		\textbf{Line} & \textbf{Integration Limits $[\AA]$}  \\
		\hline
		Cont. 1150 & $1151-1161$\\
		\hline
		Cont. 4010 & $4026-4033$\\
		Cont. 4440 & $4435-4450$\\
		Cont. 5100 & $5085-5112$\\
		Cont. 6110 & $6107-6129$\\
		Cont. 6880 & $6861-6900$\\
		Cont. 8015 & $8005-8031$\\
		Cont. 8900 & $8864-8955$\\
		\hline
		\hline
	\end{tabular}
\end{table}

\subsection{Variability Statistics}

To quantify the variability of the emission lines and continua variability statistics, defined by  \cite{rodriguez1997steps} gets adopted. This definition includes the extrema of the emission line flux densities as well as the extrema of the integrated continuum fluxes, $F_\mathrm{min}$ and $F_\mathrm{max}$, the maximum-to-minimum flux ratio $R_\mathrm{max}$, the mean flux $\langle F \rangle$, the standard deviation $\sigma_F$, and finally the fractional variability, which is defined as: 
\begin{equation}
	F_\mathrm{var}=\frac{\sqrt{\sigma_F^2-\Delta^2}}{\left<F\right>}
\end{equation}
Here $\Delta^2$ is the mean square value of the uncertainties of the fluxes defined as:
\begin{equation}
	\Delta^2 = \frac{1}{N} \sum_{i=1}^{N}\Delta_i^2
\end{equation}
The results of these parameters can be found in Table \ref{tab:varstatistics}. \\







\begin{table}[!htb] 
	\centering 
	\caption{Variability statistics of the measured continua and broad lines with minimum (2) and maximum flux density or integrated flux (3), peak-to-peak ratio (4), mean (5), standard deviation (6) and fractional variation (7).} 
	\begin{tabular}{lrrrrrr} 
		\hline 
		\hline 
		Continuum/Line &  \multicolumn{1}{c}{$F_{\text{min}}$} &  \multicolumn{1}{c}{$F_{\text{max}}$} &  \multicolumn{1}{c}{$R_{\text{max}}$} &  \multicolumn{1}{c}{$\langle F \rangle$} &  \multicolumn{1}{c}{$\sigma_F$} &  \multicolumn{1}{c}{$F_{\text{var}}$} \\ 
		\multicolumn{1}{c}{(1)} & \multicolumn{1}{c}{(2)} & \multicolumn{1}{c}{(3)} & \multicolumn{1}{c}{(4)} & \multicolumn{1}{c}{(5)} & \multicolumn{1}{c}{(6)} & \multicolumn{1}{c}{(7)} \\ 
		\hline
		Cont. 1150  & $0.52$ & $1.35$ & $2.58$ & $0.86$ & $0.25$ & $0.28$ \\
		Cont. 4010  & $2.68$ & $4.21$ & $1.57$ & $3.49$ & $0.47$ & $0.14$ \\
		Cont. 4440  & $2.42$ & $3.73$ & $1.54$ & $3.14$ & $0.39$ & $0.12$ \\
		Cont. 5600  & $1.36$ & $2.15$ & $1.59$ & $1.82$ & $0.25$ & $0.14$ \\
		Cont. 6110  & $1.49$ & $2.27$ & $1.53$ & $1.9$ & $0.23$ & $0.12$ \\
		Cont. 6880  & $1.33$ & $2.01$ & $1.5$ & $1.72$ & $0.2$ & $0.11$ \\
		Cont. 8015  & $1.18$ & $1.69$ & $1.43$ & $1.48$ & $0.15$ & $0.1$ \\
		Cont. 8900  & $1.14$ & $1.52$ & $1.33$ & $1.38$ & $0.11$ & $0.08$ \\
		\hline 
		Ly$\alpha$  & $66.87$ & $94.88$ & $1.42$ & $82.21$ & $8.03$ & $0.1$ \\
		H$\alpha$  & $112.34$ & $129.72$ & $1.15$ & $122.03$ & $4.36$ & $0.04$ \\
		H$\beta$  & $32.7$ & $39.12$ & $1.2$ & $36.49$ & $1.77$ & $0.05$ \\
		H$\gamma$  & $14.45$ & $17.85$ & $1.24$ & $16.5$ & $0.94$ & $0.06$ \\
		HeII$\,\lambda$4685  & $5.53$ & $9.81$ & $1.77$ & $7.73$ & $1.37$ & $0.18$ \\
		HeI$\,\lambda5875$  & $6.81$ & $9.54$ & $1.4$ & $8.49$ & $0.62$ & $0.07$ \\
		OI$\,\lambda 8446$ & $7.47$ & $9.13$ & $1.22$ & $8.32$ & $0.37$ & $0.04$ \\
		\hline 
		\hline 
		\label{tab:varstatistics} 
	\end{tabular} 
	
\end{table}


\section{Lightcurves}

The variability of the lightcurves can be seen in the visualization of the lightcurves in Figure \ref{fig:continua_lightcurves} and \ref{fig:emission_line_lightcurves}. In addition to the measured lightcurves of the HST campaign, the UVOT UVW2 lightcurve taken with SWIFT by \cite{mchardy2018x} is displayed for comparison which shows a higher sample-size that the lightcurves of the HST campaign. Its central wavelength is located at about 1930\,\AA\, \parencite{mchardy2018x} and was used as a reference lightcurve in \cite{cackett2018accretion} as well, which makes is interesting as a reference lightcurve for this analysis too. \\
Looking at the  continuum lightcurves in Figure \ref{fig:continua_lightcurves} they show a broadly similar overall shape. All lightcurves begin at the highest flux level in most cases, except for UVW2, whose peaks have comparable heights throughout, and the 1150 continuum, which reaches its highest flux at the second maximum. The initial maximum is followed by a pronounced flux minimum, with a decrease of about $25$--$46\%$ relative to the overall maximum flux of each curve. Subsequently, the light curves rise again toward another peak, which is more distinct in the UV and blue continua and is followed by another drop in flux. While all continua except Cont. 4010 exhibit an additional another local maximum, the flux then decline steadily toward a global minimum near the end of the campaign.
\begin{figure}[!ht]
	\centering
	\includegraphics[width=\textwidth]{pictures/Chapter4/lightcurves/NGC4593_Continua.pdf}
	\caption{Comparison of the continua lightcurves. The first panel shows the UVW2 continuum lightcurve obtained from \cite{mchardy2018x}, while the other panels show the measured continua defined in Table \ref{tab:continua} }
	\label{fig:continua_lightcurves}
\end{figure}
\begin{figure}[!ht]
	\centering
	\includegraphics[width=1\textwidth]{pictures/Chapter4/lightcurves/NGC4593_Lines_UVW2.pdf}
	\caption{Comparison of the emission-line lightcurves to the UVW2 reference lightcurve in the first panel. The UVW2 lightcurve was obtained from \cite{mchardy2018x}}
	\label{fig:emission_line_lightcurves}
\end{figure}
\\
For the RM analysis the real distance between the SMBH and the BLR is of most interests. Assuming that the continuum radiation originate from the accretion disk it is common to use the bluest available continuum band, which is
expected to originate closest to the SMBH, which would be the continuum around 1150 \AA. But because the higher sample rate, the UVW2 continuum was selected as the main reference lightcurve for the further analysis. \\
Looking now at the emission-line lightcurves in Figure \ref{fig:emission_line_lightcurves} correlation between them and the UVW2 continuum can already be spotted. The three major features of the UVW2 lightcurve, the first strong minimum, the two maxima in the center and the strong drop at the end, can be also identified in the emission line lightcurves with respective shifts in time, which will be further investigated in Section \ref{sec:time_lag_bh_mass}.





%\begin{figure}[!ht]
%	\centering
%	\includegraphics[width=1\textwidth]{pictures/Chapter4/lightcurves/NGC4593_Lines_Cont1150_not_optical_calibrated.pdf}
%	\caption{AVG RMS Spektrum}
%	\label{fig:emission_line_lightcurves_hst}
%\end{figure}







\section{Line Profiles}

In addition to the variability, it is possible to investigate the kinematics of the BLR by extracting the line profiles of broad emission lines and investigating their shape and line width. Following the same procedure as for the lightcurve extraction, an underlying linear continuum is interpolated using a chosen blue and red pseudo-continuum for each line in both the AVG and RMS spectra. The selected boundaries of the pseudo-continua are listed in Table \ref{tab:line_profiles_pseudo}. This interpolated continuum is subtracted to define a zero-flux baseline for each line profile and to allow a better comparison. Subsequently, the line profile is converted to velocity space using\\ 
\begin{equation}
	v_i =  c \cdot \frac{(\lambda_i - \lambda_\mathrm{central})}{\lambda_\mathrm{central}}.
\end{equation}\\
Here, $\lambda_i$ denotes the wavelength values, $\lambda_\mathrm{central}$ the central wavelength of the emission line, and $c$ the speed of light. The flux is normalized to the maximum of the emission lines. Figure \ref{fig:line_profiles} shows an overview of the extracted and normalized line profiles of H$\alpha$, H$\beta$, H$\gamma$, O\,\textsc{i}$\,\lambda8446$, He\,\textsc{i}$\,\lambda5875$, and He\,\textsc{ii}$\,\lambda4685$ in both the AVG and RMS spectra. It should be noted that the AVG profiles are influenced by narrow-line components, which results in slightly different line widths and shapes. However, since the narrow components are difficult to distinguish and vanish in the RMS profiles, no attempt was made to subtract them from the AVG profiles, as the RMS line widths are used for the further analysis. \\
The Balmer lines, as well as the helium lines, show well-defined profiles in both AVG and RMS, making them suitable for further kinematic analysis. The O\,\textsc{i}$\,\lambda8446$ profile is blended with the Ca\,\textsc{ii} $\lambda8498$\, $\lambda8542$\, $\lambda8662$ triplet in the AVG spectrum. Moreover, the RMS profile has low signal-to-noise, so this line is not considered further in the kinematic analysis. 
\begin{figure}[!ht]
	\centering
	\includegraphics[width=\textwidth]{pictures/Chapter4/line_profiles/Normalized_Line_Profiles.pdf}
	\caption{A plot of the AVG and RMS line profiles of the broad emission lines H$\alpha$, H$\beta$, H$\gamma$, OI$\,\lambda 8446$, HeI$\,\lambda5875$ and  HeII$\,\lambda4685$.}
	\label{fig:line_profiles}
\end{figure}
\begin{table}[h!]
	\centering
	\small
	\caption{Bounderies of the blue and red pseudo-continua used for the interpolation of underlying continua for line profile extraction.}
	\label{tab:line_profiles_pseudo}
	\begin{tabular}{lcc}
		\hline
		\hline
		\textbf{Line} & \textbf{\textbf{Pseudo-Continua AVG $[\AA]$}} & \textbf{Pseudo-Continua RMS $[\AA]$}  \\
		\hline
		H$\alpha$ & $6194-6216, 6861-6900$ & $6279-6301, 6742-6781$ \\
		H$\beta$ & $4762-4774, 5085-5112$ & $4762-4774, 4967-4984$ \\
		H$\gamma$ & $4197-4220, 4435-4450$ & $4197-4220, 4417-4429$ \\
		OI$\,\lambda 8446$ & $7999-8025, 8775-8798$ & $8222-8238, 8748-8767$ \\
		HeI$\,\lambda5875$ & $5679-5697, 6044-6057$ & $5736-5753, 6027-6045$ \\
		HeII$\,\lambda4685$ & $4198-4225, 4762-4774$ & $4543-4554, 4766-4778$ \\
		
		\hline
		\hline
	\end{tabular}
\end{table}


\subsection{Balmer-Lines}

A comparison of the Balmer-line profiles H$\alpha$, H$\beta$, and H$\gamma$ is shown in Figure \ref{fig:AVG_RMS_Balmer}. With the exception of the AVG profile of H$\gamma$, which is affected by the narrow [O III] $\lambda4363$ line overlapping the H$\gamma$ profile, the lines exhibit a similar asymmetric shape their AVG profiles. All three lines show a steep blue wing. H$\alpha$ and H$\beta$ exhibit nearly identical gradients, while H$\gamma$ is broader up to $\approx -1000\,\mathrm{km\,s^{-1}}$ before matching the slope of the other profiles. The red wings of H$\alpha$ and H$\beta$ are again similar, showing a bump at around $\approx 1000\,\mathrm{km\,s^{-1}}$ and $\approx 2500\,\mathrm{km\,s^{-1}}$, respectively, resulting in the asymmetric profile. \\
A similar behavior is seen in the RMS profiles shown in the second row of Figure \ref{fig:AVG_RMS_Balmer}. All three profiles show an asymmetric double-peaked shape, with higher flux variation in the blue peak and lower flux variation in the red peak. Again, the profiles of H$\alpha$ and H$\beta$ show a very similar shape, while the blue wing of H$\gamma$ is more extended and becomes noisy at higher velocities. Since the narrow components vanish in the RMS profiles, the red wing of H$\gamma$ becomes visible and follows the shape of H$\alpha$ and H$\beta$. For all three profiles, the double-peaked structure is red-shifted, with the blue peak located at $\approx 200\,\mathrm{km\,s^{-1}}$ and the red peak at $\approx 2000\,\mathrm{km\,s^{-1}}$. \\
The similarity of these features suggests that the Balmer lines likely originate from the same emitting region and share a common kinematic structure.




\begin{figure}[!ht]
	\centering
	\includegraphics[width=\textwidth]{pictures/Chapter4/line_profiles/AVG_and_RMS_overlay_Balmer.pdf}
	\caption{Comparison of the normalized AVG and RMS line profiles of the Balmer lines H$\alpha$ vs H$\beta$ and H$\alpha$ vs H$\gamma$ in velocity space.}
	\label{fig:AVG_RMS_Balmer}
\end{figure}

\newpage

\subsection{Helium-Lines}

Figure \ref{fig:AVG_RMS_Helium}, shows a comparison of the HeI$\,\lambda5875$ and HeII$\,\lambda$4685 emission line profile. The blue wing of the HeII$\,\lambda$4685 AVG profile is partly overlapped by the Fe II regime between $4489 \, \AA$  and $4629 \, \AA$ at velocities above about $-1000\,\mathrm{km\,s^{-1}}$. While the visible part of its blue wing follows the same shape as the HeI$\,\lambda5875$ blue wing shape, its red wing is only half as broad as the red wing of the HeI$\,\lambda5875$ red wing. Apart from that both lines show a smaller peak at around $1000\,\mathrm{km\,s^{-1}}$, which gives both line an asymmetric shape like the balmer line profiles. \\


\begin{figure}[!ht]
	\centering
	\includegraphics[width=\textwidth]{pictures/Chapter4/line_profiles/AVG_and_RMS_overlay_Helium.pdf}
	\caption{Comparison of the AVG and RMS line profiles of the Helium lines HeI$\,\lambda5875$ vs HeII$\,\lambda$4685.}
	\label{fig:AVG_RMS_Helium}
\end{figure}

\subsection{FWHM}


\begin{table}[h!]
	\centering
	\small
	\caption{Measured FWHM and line dispersion $\sigma_\mathrm{line}$ of the  RMS line profiles.}
	\label{tab:line_width_FWHM}
	\begin{tabular}{lcc}
		\hline
		\hline
		\textbf{Line} & \textbf{FWHM [km/s]} & \textbf{$\sigma_\mathrm{line}$ [km/s]}   \\
		\hline
		H$\alpha$ &  $3111 \pm 250$ & $1180 \pm 250$\\
		H$\beta$ &  $3437 \pm 200$ & $1210 \pm 200$\\
		H$\gamma$ &  $3852 \pm 300$ & $1243 \pm 300$\\
		\hline
		HeI$\,\lambda5875$ &  $3793 \pm 300$ & $1216 \pm 300$ \\
		HeII$\,\lambda4685$ &  $5778 \pm 200$ &  $1998 \pm 200$\\
		
		\hline
		\hline
	\end{tabular}
\end{table}



\section{Time Lag and BH Masses}
\label{sec:time_lag_bh_mass}

\begin{figure}[!ht]
	\centering
	\includegraphics[width=0.9\textwidth]{pictures/Chapter4/ccfs/CCFs_H_He_LyA_O_UVW2.pdf}
	\caption{AVG RMS Spektrum}
	\label{fig:ccfs}
\end{figure}

\begin{figure}[!ht]
	\centering
	\includegraphics[width=0.7\textwidth]{pictures/Chapter4/lighcurves_and_ccfs_and_time_lag_tables/UV_to_HAlpha_ccfs_and_reference_lightcurves.pdf}
	\caption{AVG RMS Spektrum}
	\label{fig:ccfs_lightcurves}
\end{figure}

