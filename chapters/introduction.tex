\chapter{Introduction}





Active galactic nuclei (AGN) are among the most luminous objects in the universe, emitting radiation across the entire electromagnetic spectrum \parencite{netzer2013agn}. Unlike inactive galaxies like the Milky Way, the supermassive black-hole (SMBH) in an active galaxy continues to accrete matter from its central region. This process generates thermal radiation \parencite{peterson1997introduction} as well as non-thermal emission such as photoionization and subsequent recombination \parencite{netzer2013agn}. While the central AGN region is not directly resolvable, analyzing its spectral features allows constraints on the kinematics, geometry, and physical conditions of the central region. Owing to the intrinsic variability of AGNs, time lags can be measured between the light curves of different spectral features in the spectrum, particularly between the continuum emission from the accretion disk and the responding broad emission lines of the broad-line region. By combining these time delays with the observed widths of the emission lines, the mass of the SMBH can be estimated. This technique is known as reverberation mapping (RM) \parencite{peterson1993}.\\
A representative example of this class of objects is the AGN of the active barred spiral galaxy NGC 4593, which shows strong variability from the X-ray to the optical bands \parencite{mchardy2017ngc4593}.
To perform a reverberation mapping analysis an observation campaign over several days or weeks is needed, to measure the variability of NGC 4593. For this purpose, a monitoring campaign with the Hubble Space Telescope (HST) was used, which was conducted by E. M. Cackett \parencite{cackett2018accretion}.\\
Based on this campaign, the goal if this thesis will be a classical reverberation mapping analysis of NGC 4593, with a particular focus on spectral features like the optical broad emission lines such as the Balmer emission lines. An additional focus will be on the variability and time lag of the  O\textsc{i}$\,\lambda8446$ emission line, which may be strengthened by a Bowen fluorescence process \parencite{Bowen_1947, netzer1976}. This is of particular interesting, because the measurement of O\textsc{i}$\,\lambda8446$ variability as well as its lag in NGC 4593 from a RM campaign has not be done before.