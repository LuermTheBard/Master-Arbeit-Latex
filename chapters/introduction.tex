\chapter{Introduction}



Active galactic nuclei (AGNs) are among the most luminous objects in the universe, emitting radiation across the entire electromagnetic spectrum \parencite{netzer2013agn} and exhibiting strong variability on timescales ranging from hours to days \parencite{Ulrich1997}. In contrast to inactive galaxies such as the Milky Way, the supermassive black hole (SMBH) in an active galaxy accretes matter from the central region. This process generates thermal emission from the accretion flow \parencite{peterson1997introduction} and drives line emission through photoionization and subsequent recombination in the surrounding gas \parencite{netzer2013agn}. These ionized line-emitting regions are commonly divided into the broad-line region (BLR) and the narrow-line region (NLR) \parencite{peterson1997introduction}. The BLR is located at close distances of light-days to a few light-years from the central SMBH \parencite{goadBroadLine} and rotates at high velocities of several thousand $\mathrm{km\,s^{-1}}$, which leads to strong Doppler broadening of the emission lines \parencite{peterson1997introduction}. In contrast, the NLR extends out to several hundred parsecs from the central region \parencite{peterson1997introduction} with velocities of only a few hundred $\mathrm{km\,s^{-1}}$, resulting in comparatively narrow line profiles.\\ An interesting example of this class of objects is the barred spiral galaxy NGC\,4593. It shows pronounced variability from the X-ray to the optical bands \parencite{mchardy2017ngc4593, cackett2018accretion} and exhibits several prominent broad emission lines, including Balmer and helium lines, as well as low-ionization lines like O\textsc{i}$\,\lambda8446$ \parencite{kollatschny1997balmer, cackett2018accretion, Ochmann_2025}. Several studies on NGC\,4593 have analyzed the structure and kinematics of its broad-line region (e.g. \cite{kollatschny1997balmer, denney2006ngc4593, cackett2018accretion}) using reverberation mapping (RM) of the broad emission lines. This method uses measurable time delays between variations in the ionizing continuum and the response of the line-emitting gas to infer the characteristic size and geometry of the broad-line region \parencite{peterson1993}.\\ In a recent study, \cite{cackett2018accretion} carried out an observation campaign using the Hubble Space Telescope (HST) with the Space Telescope Imaging Spectrograph (STIS), between 12 July and 6 August 2016, on NGC\,4593 covering wavelength ranges of approximately  $1100\,\AA$ -- $1700\,\AA$ and $3900\,\AA$ -- $9000\,\AA$. This campaign focused on the reverberation of the accretion disk of NGC\,4593 by analyzing the UV and optical continua \parencite{cackett2018accretion}. However, a RM analysis of the BLR has not been attempted with this dataset. As a HST dataset, it is ideally suited for a RM analysis, thanks to its daily cadence and broad wavelength coverage. It allows a more detailed structural analysis of the BLR of NGC\,4593 compared to earlier campaigns \parencite{kollatschny1997balmer, denney2006ngc4593, Williams_2018} by including broad emission lines to the analysis, which have not been monitored for NGC\,4593 yet (e.g. UV emission lines). \\
Furthermore, NGC\,4593 shows variability in the low-ionization O\,\textsc{i}$\,\lambda8446$ line and the Ca\,\textsc{ii}$\,\lambda8498\,\lambda8542\,\lambda8662$ triplet \parencite{Ochmann_2025}, which have not been included in RM campaigns so far. The O\,\textsc{i}$\,\lambda8446$ line is particularly interesting because its strength can be enhanced by Bowen fluorescence \parencite{Bowen_1947, grandi1980}. Since it is covered by the \cite{cackett2018accretion} campaign, its variability and time lag to can be measured and its Bowen fluorescence process investigated in a RM analysis for the first time.