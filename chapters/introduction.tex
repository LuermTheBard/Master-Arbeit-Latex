\chapter{Introduction}

The barred spiral Galaxy NGC 4593 belongs to the active galaxies with an active galactic nucleus (AGN). Its among some of the most luminous objects in the universe emitting radiation across the entire electromagnetic spectrum. Most of the radiation originates from the AGN itself, outshining its host-galaxy by multiple orders of magnitude. In contrast to inactive galaxies, such as the milky-way, the supermassive black-hole (SMBH) in an active galaxies is still absorbing matter from the central region, creating thermal radiation as well as other non-thermal processes like photo-ionization and followed recombination. While the central AGN region is not directly resolvable, measuring and analyzing the spectral features in its emissions makes is possible to study its structure. If an AGN is variable, which is the case for NGC4593, it is possible to measure time lags between the light-curves of different spectral features in the spectra. With these time lags and the measurable widths of the emission lines  in the spectra, it is possible to determine the radii of the different emitting regions around the SMBH. This kind of analysis is called reverberation mapping (RM) and can be further used to estimate the mass of the SMBH. \\
To execute a reverberation mapping analysis an observation campaign over multiple days or weeks is needed, to measure the variability of NGC4593. For that, an observation campaign from the Hubble Space Telescope (HST) was used, which was conducted by E. M. Cackett \parencite{cackett2018accretion}.\\
Based on this campaign, this thesis will cover a classical reverberation mapping analysis of NGC4593, with a particular focus on spectral features like the optical broad emission lines such as the Balmer emission lines. An additional focus will be on the variability and time lag of the  O\textsc{i}$\,\lambda8446$ emission line and a possible Bowen-fluorescence process driven by the Lyman emission lines.