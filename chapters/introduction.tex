\chapter{Introduction}



Active galactic nuclei (AGNs) are among the most luminous objects in the universe, emitting radiation across the entire electromagnetic spectrum \parencite{netzer2013agn}. In contrast to inactive galaxies such as the Milky Way, the supermassive black hole (SMBH) in an active galaxy continues to accrete matter from its central region. This process produces thermal emission from the accretion flow \parencite{peterson1997introduction} and drives line emission through photoionization and subsequent recombination in the surrounding gas \parencite{netzer2013agn}. AGNs exhibits strong variability across the electromagnetic spectrum, with timescales ranging from hours to days \parencite{Ulrich1997}.\\ An interesting example of this class of objects is the barred spiral galaxy NGC\,4593. It shows pronounced variability from the X-ray to the optical bands \parencite{mchardy2017ngc4593, cackett2018accretion} and exhibits several prominent broad emission lines, including Balmer and helium lines, as well as low-ionization lines like O\textsc{i}$\,\lambda8446$ \parencite{kollatschny1997balmer, cackett2018accretion, Ochmann_2025}. Several studies have analysed the structure and kinematics of NGC\,4593 (e.g. \parencite{kollatschny1997balmer, denney2006ngc4593, cackett2018accretion}) using reverberation mapping (RM) of the broad emission lines. This method uses measurable time delays between variations in the ionizing continuum and the response of the line-emitting gas to infer the characteristic size and geometry of the broad-line region \parencite{peterson1993}.\\ In a recent study, \cite{cackett2018accretion} carried out an observation campaign using the Hubble Space Telescope (HST) with the Space Telescope Imaging Spectrograph (STIS), between 12 July and 6 August 2016 on NGC\,4593, covering wavelength ranges of approximately  $1100\,\AA$ -- $1700\,\AA$ and $3900\,\AA$ -- $9000\,\AA$. This campaign focused on the reverberation of the accretion disk of NGC\,4593 by analysing the UV and optical continua \parencite{cackett2018accretion}. This enables to conduct a classical reverberation mapping analysis of emission lines ranging from the UV up to the NIR band, which has not be don for NGC\,4593 yet and derives an estimate of the SMBH mass. \\
Furthermore, NGC\,4593 shows prominent emission in the low-ionization O\textsc{i}$\,\lambda8446$ line and the Ca\,\textsc{ii}$\,\lambda8498\,\lambda8542\,\lambda8662$ triplet \parencite{Ochmann_2025}, which have rarely been included in RM campaigns so far. The O\textsc{i}$\,\lambda8446$ line is particularly interesting because its strength can be enhanced by Bowen fluorescence\parencite{Bowen_1947, grandi1980}. Since it is covered by the \cite{cackett2018accretion} campaign, its variability and time lag to can be measured from a HST/STIS RM campaign for the first time.