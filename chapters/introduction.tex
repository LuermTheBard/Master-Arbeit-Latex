\chapter{Introduction}



Active galactic nuclei (AGNs) are among the most luminous objects in the universe, emitting radiation across the entire electromagnetic spectrum \parencite{netzer2013agn}. Unlike inactive galaxies such as the Milky Way, the supermassive black hole (SMBH) in an active galaxy continues to accrete matter from its central region. This process generates thermal radiation \parencite{peterson1997introduction} as well as non-thermal emission such as photoionization and subsequent recombination \parencite{netzer2013agn}. AGN show strong variability across the entire electromagnetic spectrum, with timescales ranging from hours to days \parencite{Ulrich1997}.\\ A particularly interesting example is the barred spiral galaxy NGC\,4593. NGC\,4593 shows strong variability from the X-ray to the optical bands \parencite{mchardy2017ngc4593, cackett2018accretion} with several strong broad emission lines, including Balmer lines, Lyman lines and helium lines, among others \parencite{kollatschny1997balmer, cackett2018accretion, Ochmann_2025}. Several studies have analysed the structure and kinematics of NGC\,4593 (e.g. \parencite{kollatschny1997balmer, denney2006ngc4593, cackett2018accretion}) using reverberation mapping (RM) of the broad emission lines. This method uses measurable time lags between the response of the emitting regions to variations in the ionizing continuum to probe the structure and size of the broad-line region \parencite{peterson1993}.\\ In a recent work, \cite{cackett2018accretion} conducted an observation campaign using the Hubble Space Telescope (HST) with the Space Telescope Imaging Spectrograph (STIS), between 12 July and 6 August 2016 on NGC\,4593, covering wavelength ranges from about $1100\,\AA$ to $1700\,\AA$ and from $3900\,\AA$ to $9000\,\AA$. This work focused on the reverberation of the accretion disk of NGC\,4593 by analyzing the UV and optical continua \parencite{cackett2018accretion}. This dataset allows me to perform a classical reverberation mapping analysis of the broad emission-lines based on these data, and to determine the mass of the SMBH.\\
Furthermore, NGC\,4593 shows strong emission in the O\textsc{i}$\,\lambda8446$ emission line and the Ca\,\textsc{ii}$\,\lambda8498\,\lambda8542\,\lambda8662$ triplet \parencite{Ochmann_2025}. These low-ionization lines have not yet been included in RM campaigns yet. The O\textsc{i}$\,\lambda8446$ emission line is particularly interesting, as it can get enhanced through Bowen fluorescence \parencite{Bowen_1947, grandi1980}. Its presence allows the variability and time lag to be measured from a HST/STIS RM campaign for the first time.