\chapter{Discussion}

The nearby Seyfert I galaxy NGC\,4593 has been monitored in several earlier RM studies (e.g. \cite{kollatschny1997balmer,denney2006ngc4593, cackett2018accretion}). In a recent work, \cite{cackett2018accretion} obtained an observation campaign of NGC\,4593 with HST including the optical to NIR range between $\sim 3900 \AA$--$9000 \AA$ as well as the range of $\sim 1100\AA$--$1700\AA$ of the UV region. Only a few other AGN have been monitored by HST, and the high cadence of nearly daily observations made this campaign of particular interests for a RM campaign. It enabled us to conduct RM analysis of the BLR of NGC\,4593, including broad emission lines that had not been monitored before (e.g. UV lines) and examine the BLR structure of NGC\,4593. In addition the Bowen fluorescents emission line O\,\textsc{i}\,$\lambda\,8446$ has been included. Given the range of the campaign, it enabled to examine its Bowen Fluorescence relation to Ly$\beta$, which has never been monitored in a RM campaign of HST/STIS before. It was possible to use the UV continuum around $1150\,\AA$ to use as a proxy for the ionized continuum rather than a optical continuum used in earlier works (e.g. \cite{kollatschny1997balmer, denney2006ngc4593}). As the higher ionized continua, the UV continuum is assumed to origin closer from the SMBH  \textbf{cite}.  The $1150\,\AA$ continuum light curve has been compared to the more densely sampled \textit{Swift} UVOT UVW2 light curve presented by \cite{mchardy2018x} and showed a strong similarity. Owing to its higher cadence, the UVW2 light curve is adopted as the reference light curve for the RM analysis presented here, which covered the UV emission lines Ly$\alpha$, N\,\textsc{v}$\,\lambda\lambda 1238,\,1242$, Si\,\textsc{iv}$\,\lambda\lambda 1393,\,1402$, C\textsc{iv}$\,\lambda\lambda 1548,\,1550$ and He\,\textsc{ii}$\,\lambda1640$, the optical emission lines H$\alpha$ - H$\delta$, He\,\textsc{ii}$\,\lambda4686$ and He\,\textsc{i}$\,\lambda5876$ and the NIR emission line O\,\textsc{i}\,$\lambda\,8446$.  

\section{Structure of the BLR}

The extracted light curves of the emission lines showed a overall similar shape with strong correlation the the UVW2 light curve with CCF maximal between $0.7$--$0.9$. Only the O\,\textsc{i}\,$\lambda\,8446$ light curve showed lower maximal correlation to UVW2 of $0.6$, as its variability is majorly driven through Bowen fluorescence \cite{ochmann2026first}, which will be further discussed in Section \ref{sec:diskussion_BF}.\\
Earlier works established that the BLR of AGN show radial ionization stratification, resulting that higher ionized ions are found closer to the ionization source than less ionized ions \parencite{collin1988broad, xi2025supermassive}.
The obtained time lags implied that this stratification also applies the BLR of NGC\,4593. Earlier studies of NGC\,4593 already showed different response times for H$\alpha$ and H$\beta$ to the ionizing continua \parencite{kollatschny1997balmer,denney2006ngc4593, Williams_2018}, which are in good agreement of the lags found in this analysis with $3.2^{+1.0}_{-0.6}\,$ and $2.3^{+1.0}_{-0.3}\,$days. The higher-order Balmer lines H$\gamma$ and H$\delta$ show slightly shorter delays than H$\beta$, with $1.7^{+0.2}_{-0.4}\,$days and $1.7^{+0.4}_{-0.4}\,$days. Stratification can be as well seen in the lags of the Helium lines. While the lower ionized He\,\textsc{i}$\,\lambda5876$ lags  $3.3^{+0.7}_{-0.4}\,$days behind UVW2, the H$\alpha$ lag within uncertainties, He\,\textsc{ii}$\,\lambda1640$ and He\,\textsc{ii}$\,\lambda4686$ exhibits the fastes responsive time of the monitored emission lines of $0.3^{+0.6}_{-0.2}\,$days and $0.7^{+0.1}_{-0.3}\,$days, respectively. The UV lines all show a consistent lag of about $\sim 1\,$day with $1.0 \ensuremath{_{-0.2}^{+0.4}}$, $1.0 \ensuremath{_{-0.4}^{+0.2}}$, $1.2 \ensuremath{_{-0.4}^{+0.1}}$ and $0.8 \ensuremath{_{-0.1}^{+0.5}}$ days for Ly$\alpha$, N\,\textsc{v}$\,\lambda\lambda 1238,\,1242$, Si\,\textsc{iv}$\,\lambda\lambda 1393,\,1402$ and C\textsc{iv}$\,\lambda\lambda 1548,\,1550$, respectively. This concludes that He\,\textsc{ii} originates the closest form the ionizing continuum, followed by the high ionized UV emission lines and the Balmer and He\,\textsc{i} lines with the longest distance. Because UVW2 lags the extracted HST/STIS $1150\,\AA$ continuum by about $\sim 0.5\,$days this lag has to be added, to optain the final BLR radii for every emission line. \\
The RMS line profiles of the monitored lines all show a asymmetric shape. The Balmer line profiles showed a similar asymmetric, double-peaked structure and are shifted toward positive velocities. This shift indicates that most of the line variability originates from the receding part of the emitting region. H$\alpha$ and H$\beta$ show a comparatively steeper blue wing, whereas H$\gamma$ and H$\delta$ exhibit a more extended blue wing, implying that they may probe somewhat different kinematic conditions than H$\alpha$ and H$\beta$, which also can be seen in the line width. The measured FWHM of H$\alpha$ with $3111 \pm 250\,\mathrm{km\,s^{-1}}$ and H$\beta$ with $3437 \pm 200\,\mathrm{km\,s^{-1}}$ agree well with previously reported line widths (e.g. \cite{kollatschny1997balmer, denney2006ngc4593}). The broader blue wing of H$\gamma$ and H$\delta$ resulted in higher values $3852 \pm 400,\mathrm{km,s^{-1}}$ and $5905 \pm 1000,\mathrm{km,s^{-1}}$. \\
The RMS profiles of the Helium lines exhibit a broader blue wing and a steep red wing. While 




\section{Helium Lines}

In addition to the Balmer lines, the helium lines He\,\textsc{ii}$\,\lambda4686$, He\,\textsc{ii}$\,\lambda1640$, and He\,\textsc{i}$\,\lambda5876$ have been included in this RM analysis. Like the balmer lines light curves, all helium-line light curves track the overall variability pattern of UVW2. A larger relative delay is apparent for He\,\textsc{i}$\,\lambda5876$ compared to the He\,\textsc{ii} lines. Their correlation with the UVW2 light curves peak at about $\sim 0.7$ for He\,\textsc{ii}$\,\lambda1640$ and He\,\textsc{i}$\,\lambda5876$ at about $\sim 0.85$ for He\,\textsc{ii}$\,\lambda4686$. He\,\textsc{i}$\,\lambda5876$ lags  $3.3^{+0.7}_{-0.4}\,$days behind UVW2, consistent with the H$\alpha$ lag within uncertainties. In contrast, the He\,\textsc{ii} lines respond significantly faster, with lags of $0.7^{+0.1}_{-0.3}\,$days for He\,\textsc{ii}$\,\lambda4686$ and $0.3^{+0.6}_{-0.2}\,$days for He\,\textsc{ii}$\,\lambda1640$. This implies that the He\,\textsc{ii} lines exhibits a much smaller BLR radius of about $\sim 1\,$day than He\,\textsc{i} with about $3.8\,$days, by accounting for the lag between the UVW2 and the $1150\,\AA$ continuum.\\
Like the RMS profile of the Balmer lines, the maximum of the He\,\textsc{i}$\,\lambda5876$ RMS profile is shifted to positive velocities and exhibits a FWHM of $3952 \pm 400\,\mathrm{km\,s^{-1}}$. The He\,\textsc{ii} profiles are significantly broader with FWHM values of $5971 \pm 400\,\mathrm{km\,s^{-1}}$ for He\,\textsc{ii}$\,\lambda4686$ and $6891 \pm 600\,\mathrm{km\,s^{-1}}$ for He\,\textsc{ii}$\,\lambda1640$. 



\section{UV Lines}

The wavelength coverage of the \cite{cackett2018accretion} campaign allowed to include the UV emission lines Ly$\alpha$, N\,\textsc{v}$\,\lambda\lambda 1238,\,1242$, Si\,\textsc{iv}$\,\lambda\lambda 1393,\,1402$ and C\textsc{iv}$\,\lambda\lambda 1548,\,1550$ in a RM analysis of NGC\,4593 for the first time. With exception of the previously discussed UV He\,\textsc{i} line, the correlation of the  UV emission line light curves with the UVW2 light curves peak at about $\sim 0.8$ -- $0.9$, comparable to the Balmer and He,\textsc{ii} lines. The lags relative to UVW2 have been founded with $1.0 \ensuremath{_{-0.2}^{+0.4}}$, $1.0 \ensuremath{_{-0.4}^{+0.2}}$, $1.2 \ensuremath{_{-0.4}^{+0.1}}$ and $0.8 \ensuremath{_{-0.1}^{+0.5}}$ days for Ly$\alpha$, N\,\textsc{v}$\,\lambda\lambda 1238,\,1242$, Si\,\textsc{iv}$\,\lambda\lambda 1393,\,1402$ and C\textsc{iv}$\,\lambda\lambda 1548,\,1550$, respectively. Within uncertainties, these lags are mutually consistent, suggesting that the UV lines mainly originate at similar BLR radii from the central SMBH. Accounting for the lag between UVW2 and the $1150\,\AA$ continuum, this corresponds to a BLR radius of approximately $\sim1.5\,$days and is therefor located between the emitting region of the He\,\textsc{i} lines and the emitting region of the Balmer and He\,\textsc{ii} lines. \\
The FHWM of the UV emission lines have been found with $4556 \pm 350\,\mathrm{km\,s^{-1}}$ for Ly$\alpha$, $3383 \pm 1000\,\mathrm{km\,s^{-1}}$ for N\,\textsc{v}$\,\lambda\lambda 1238,\,1242$, $10005 \pm 1000\,\mathrm{km\,s^{-1}}$ for Si\,\textsc{iv}$\,\lambda\lambda 1393,\,1402$ and $8428 \pm 500\,\mathrm{km\,s^{-1}}$ for C\textsc{iv}$\,\lambda\lambda 1548,\,1550$. The FWHM of N\,\textsc{v}$\,\lambda\lambda1238,\,1242$ is estimated from the red wing only, since the blue side is blended with Ly$\alpha$. Also the FWHM of the Si\,\textsc{iv}$\,\lambda\lambda 1393,\,1402$ RMS profile is not suitable for further interpretations, due to the noisy profile and possible blending by O\,\textsc{iv}] emission. 



\section{Black Hole Mass}

Based on the reverberation mapping results, virial products were derived for the selected emission lines, and the SMBH mass was estimated using a scale factor of $f=6.9$. The adopted scale factor accounts for the low inclination angle of $i \sim 11^\circ$ reported by \cite{ochmann2024transient}. Its estimation is based on the relation $f \sim \sin^{-2}(i)$ \parencite{krolik2001systematic}, which captures the inclination dependence of the observed line-of-sight velocity dispersion. Since the velocity width is parameterized via FWHM in this work rather than the line dispersion, an additional conversion has been applied using $\sigma_{\mathrm{line}} \approx \mathrm{FWHM}/2$ \parencite{peterson2004}. With that a inverse-variance–weighted mean of the BH mass has been found with $\bar{M}_{\mathrm{BH}} \approx (3.35 \pm 0.62)\times 10^{7}\,M_\odot$. This value is conclusive with determined masses of earlier works (e.g. \cite{kollatschny1997balmer, denney2006ngc4593}) in order of magnitude.

\section{Variation and Bowen Fluorescence of O\,\textsc{I}$\,\lambda8446$}
\label{sec:diskussion_BF}

The low-ionization line O\,\textsc{i}\,$\lambda\,8446$ was included in the RM analysis, as it exhibits noticeable variability in its RMS spectrum and has not yet been monitored in an HST reverberation-mapping campaign. O\,\textsc{i}\,$\lambda\,8446$ is known to be a Bowen fluorescence line pumped by Ly$\beta$ emission \parencite{grandi1980}. The wavelength coverage of the campaign, spanning parts of the UV and NIR region, allowed an examination of this process. Although Ly$\beta$ is not covered by the observations, Ly$\alpha$ has been used as a proxy to investigate its expected Bowen fluorescence connection to O\,\textsc{i}\,$\lambda\,8446$. As described in Section \ref{sec:bowen_fluorescence}, Ly$\beta$ excites O\,\textsc{i} through a near-resonant transition at $\lambda\,1025\,\AA$. This is followed by decay via O\,\textsc{i}\,$\lambda\,11287$ and subsequently by O\,\textsc{i}\,$\lambda\,8446$, producing the observed O\,\textsc{i}\,$\lambda\,8446$ emission (see Figure \ref{fig:bowen_cascate}). This cascade is expected to occur with a photon ratio of 1:1 \parencite{grandi1980}. \cite{ochmann2026first} reported a photon ratio of $\sim 0.8$ between O\,\textsc{i}\,$\lambda\,8446$ and O\,\textsc{i}\,$\lambda\,11287$ (taken from the near-infrared SpeX spectra obtained by \cite{landt2008near}). This result indicates that Ly$\beta$ pumping is the main excitation mechanism for O\,\textsc{i}\,$\lambda\,8446$ in NGC\,4593. In addition, the comparison shown in Figure \ref{fig:ccfs_Bowen} demonstrates that O\,\textsc{i}\,$\lambda\,8446$ exhibits a significantly stronger correlation with Ly$\alpha$ than with UVW2, further supporting this interpretation. Therefore, the emitting region of O\,\textsc{i} is likely located at a distance corresponding to a lag of $\sim 3.5$ days from the ionizing continuum, consistent with the Bowen fluorescence scenario, rather than the $5.0^{+2.1}_{-1.7}$ days implied by the direct lag relative to UVW2 under the assumption of recombination as the main excitation mechanism. With that, the Bowen fluorescence lag of O\,\textsc{i} corresponds to the radius of the emitting region of H$\alpha$ with $3.2 \ensuremath{_{-0.7}^{+1.0}}\,$ days. Since H$\alpha$ and O\,\textsc{i}\,$\lambda\,8446$ also exhibits a strong correlation with a minimal time shift of $\sim 0.3$ days, it can be inferred that O\,\textsc{i}\,$\lambda\,8446$ is emitted at approximately the same radius as H$\alpha$ within uncertainties.


 




