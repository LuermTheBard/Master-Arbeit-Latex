\chapter{Discussion}
The observing campaign presented by \cite{cackett2018accretion} covers an extensive wavelength range for NGC\,4593, spanning $\sim1100$–$1700\,\AA$ in the UV and $\sim3900$–$9000\,\AA$ in the optical to NIR. Reverberation-mapping studies of NGC\,4593 have previously focused on optical broad emission lines (e.g. \cite{kollatschny1997balmer, denney2006ngc4593}). Consequently, the wavelength coverage of \cite{cackett2018accretion} allowed UV broad emission lines to be included in the conducted RM analysis and BH estimation of NGC\,4593 for the first time.

\section{Light Curves}



\section{Line Profiles}


\section{Time Lag}






\textbf{Ha und Hb region mit früheren papern vergleichen. Erste behandlung von Helium und UV linien}








\section{Black Hole Mass}

Based on the results of the reverberation campaign, virial products for the selected emission lines were derived and the black hole mass was estimated using a scale factor of $f=6.9$. The scale factor accounts for the low inclination ($i \sim 11^\circ$) of the modeled elliptical accretion disc in NGC\,4593 \parencite{ochmann2024transient}. It was estimated using the relation $f \sim \sin^{-2}(i)$ \parencite{krolik2001systematic} and adjusted based on $\sigma_{\mathrm{line}} \approx \mathrm{FWHM}/2$ \parencite{peterson2004}, since the velocity dispersion was parameterized using the FWHM rather than the line dispersion. After the mass estimates were obtained for each selected emission line, an inverse-variance–weighted mean was calculated: $\bar{M}_{\mathrm{BH}} \approx (3.35 \pm 0.62)\times 10^{7},M\odot$. Earlier works using similar approaches reported $M \approx 1.4 \times 10^7,M_\odot$ \cite{kollatschny1997balmer} and $M = (9.8 \pm 2.1) \times 10^6,M_\odot$ \cite{denney2006ngc4593}. These estimates were primarily based on H$\alpha$ \parencite{kollatschny1997balmer} and H$\beta$ \parencite{denney2006ngc4593}, respectively. Therefore, the weighted-mean mass based on the Balmer, helium, and UV lines is consistent in order of magnitude with previous estimates, but yields a slightly higher value. It should be noted that these studies adopted different approximations for the scale factor, which can contribute to the differences in the reported values.

\section{Variation and Bowen Fluorescence of O\,\textsc{I}$\,\lambda8446$}

The low-ionization line O\,\textsc{i}\,$\lambda\,8446$ was included in the RM analysis, as it exhibits noticeable variability in its RMS spectrum and has not yet been monitored in an HST reverberation-mapping campaign. O\,\textsc{i}\,$\lambda\,8446$ is known to be a Bowen fluorescence line pumped by Ly$\beta$ emission \parencite{grandi1980}. The wavelength coverage of the campaign, spanning parts of the UV and NIR region, allowed an examination of this process. Although Ly$\beta$ is not covered by the observations, Ly$\alpha$ has been used as a proxy to investigate its expected Bowen fluorescence connection to O\,\textsc{i}\,$\lambda\,8446$. As described in Section \ref{sec:bowen_fluorescence}, Ly$\beta$ excites O\,\textsc{i} through a near-resonant transition at $\lambda\,1025\,\AA$. This is followed by decay via O\,\textsc{i}\,$\lambda\,11287$ and subsequently by O\,\textsc{i}\,$\lambda\,8446$, producing the observed O\,\textsc{i}\,$\lambda\,8446$ emission (see Figure \ref{fig:bowen_cascate}). This cascade is expected to occur with a photon ratio of 1:1 \parencite{grandi1980}. \cite{ochmann2026first} reported a photon ratio of $\sim 0.8$ between O\,\textsc{i}\,$\lambda\,8446$ and O\,\textsc{i}\,$\lambda\,11287$ (taken from thenear-infrared SpeX spectra obtained by \cite{landt2008near}). This result indicates that Ly$\beta$ pumping is the main excitation mechanism for O\,\textsc{i}\,$\lambda\,8446$ in NGC\,4593. In addition, the comparison shown in Figure \ref{fig:ccfs_Bowen} demonstrates that O,\textsc{i}\,$\lambda\,8446$ exhibits a significantly stronger correlation with Ly$\alpha$ than with UVW2, further supporting this interpretation. Therefore, the emitting region of O\,\textsc{i} is likely located at a distance corresponding to a lag of $\sim 3.5$ days from the ionizing continuum, consistent with the Bowen fluorescence scenario, rather than the $5.0^{+2.1}_{-1.7}$ days implied by the direct lag relative to UVW2 under the assumption of recombination as the main excitation mechanism. The Bowen fluorescence lag of O\,\textsc{i} of corresponds to the distance of H$\alpha$ region from the ionizing continuum with $3.2 \ensuremath{_{-0.7}^{+1.0}}$. Since H$\alpha$ and O\,\textsc{i}\,$\lambda\,8446$ also exhibits a strong correlation with a minimal time shift of $\sim 0.3$ days, it can be inferred that O\,\textsc{i}\,$\lambda\,8446$ is emitted at approximately the same distance as H$\alpha$ within uncertainties.


 




