\chapter{Discussion}

\section{Continua}
The observation campaign of \cite{cackett2018accretion} covers a extensive wavelength range of NGC\,4593, between $\sim 1100\AA$--$1700\AA$ and $\sim 3900 \AA$--$9000 \AA$. While several studies already conducted reverberation mapping analysis of the broad emission lines in the optical region of NGC\,4593 (e.g \cite{kollatschny1997balmer, denney2006ngc4593}), the wavelength range of the \cite{cackett2018accretion} campaign allowed to include the UV range in a RM analysis of this galaxy for the first time. In addition, the UV continuum around $1150\,\AA$ has been adopted as a proxy for the ionizing continuum rather than an optical continuum, which has been done in earlier studies (e.g. \cite{kollatschny1997balmer, denney2006ngc4593}). The extracted $1150\,\AA$ continuum light curve and its variability parameters have been compared to the corresponding results reported by \cite{cackett2018accretion} and show good agreement. The shape of the $1150\,\AA$ continuum light curve closely matches the more densely sampled \textit{Swift} UVOT UVW2 light curve presented by \cite{mchardy2018x}. Owing to its higher sample rate, the UVW2 light curve is adopted as the reference light curve for the RM analysis presented here. The extracted emission light curves showed a similar but shifted shape. 

\section{Balmer Lines}
In addition to the commonly studied H$\alpha$ and H$\beta$ lines, the higher order Blamer lines H$\gamma$ and H$\delta$ has been included in this analysis. Their light curves showed strong correlation with the UVW2 light curves, with maxmum valus between $\sim 0.8$--$0.9$ The centroids of the CCF yielded a time lag of $3.2^{+1.0}_{-0.6}\,$ and $2.3^{+1.0}_{-0.3}\,$days for H$\alpha$ and H$\beta$ relative to UVW2 respectively, while the higher-order lines H$\gamma$ and H$\delta$ show a slightly shorter time lag with $1.7 \ensuremath{_{-0.4}^{+0.2}}$ and $1.7 \ensuremath{_{-0.4}^{+0.4}}\,$days behind UVW2 within uncertainties. With UVW2 lag about $\sim 0.5$ days behind the $1150\,\AA$ continuum, this infers a characteristic radius of the emitting region of $\sim 3.7\,$days for H$\alpha$, $\sim 2.8\,$days for H$\beta$ and $\sim 2.2\,$days for H$\gamma$ and H$\delta$. The found radii for H$\alpha$ and H$\beta$ are in good agreement with earlier studies (eg. \cite{kollatschny1997balmer,denney2006ngc4593,Williams_2018}).\\
The RMS line profiles exhibit similar asymmetric, double-peaked shape shifted to positive velocities. This shift indicates that most of the  line variability originates from the receding part of the emitting region.\\
In a more detailed compartment, H$\alpha$ and H$\beta$ as well as H$\gamma$ and H$\delta$ showed more similarities in shape to each other respectively, with a steep blue flank for H$\alpha$ and H$\beta$ and a more extended blue flank H$\gamma$ and H$\delta$. This and the small diffing time lags indicates, that H$\gamma$ and H$\delta$ originate from a closer emitting region under a slightly differed kinematic properties, that H$\alpha$ and H$\beta$. The found FWHM of the H$\alpha$ and H$\beta$ profiles of $3111 \pm 250\,\mathrm{km\,s^{-1}}$ and $3437 \pm 200\,\mathrm{km\,s^{-1}}$ are in good agreement with earlier reports of the line width (eg. \cite{kollatschny1997balmer, denney2006ngc4593}). Within uncertainties H$\gamma$ and H$\delta$ show slightly higher FWHM values  with $3852 \pm 400\,\mathrm{km\,s^{-1}}$ and $5905 \pm 1000\,\mathrm{km\,s^{-1}}$ compared to H$\alpha$ and H$\beta$, which also strengthens the assumption that they originate from a slightly different region. 


\section{Helium Lines}




\section{UV Lines}

The results of the first RM analysis of the UV emission lines Ly$\alpha$, N\,\textsc{v}$\,\lambda\lambda 1238,\,1242$, Si\,\textsc{iv}$\,\lambda\lambda 1393,\,1402$ and C\textsc{iv}$\,\lambda\lambda 1548,\,1550$ in NGC\,4593 are presented in this section. With exception of the previously discussed UV He\,\textsc{i} line, the light curves showed maximal correlation values of between $\sim 0.8$ -- $0.9$ like the Balmer and He\,\textsc{ii} lines. The derived centroid of the CCF yields lags of $1.0 \ensuremath{_{-0.2}^{+0.4}}$, $1.0 \ensuremath{_{-0.4}^{+0.2}}$, $1.2 \ensuremath{_{-0.4}^{+0.1}}$ and $0.8 \ensuremath{_{-0.1}^{+0.5}}$ days relative to the UVW2 light curves for Ly$\alpha$, N\,\textsc{v}$\,\lambda\lambda 1238,\,1242$, Si\,\textsc{iv}$\,\lambda\lambda 1393,\,1402$ and C\textsc{iv}$\,\lambda\lambda 1548,\,1550$ respectively. Therefore it can be followed within uncertainties, that they all originate within the same distance from the SMBH that. Together with the time lag between the UVW2 and the $1150\,\AA$ continuum light curve this correspondence to a characteristic radius of the emitting region of about $\sim 1.5\,$days. Therefor it lies between the He\,\textsc{i} region and the Balmer and He\,\textsc{ii} regions. \\
The analysis of the line profiles yielded a FWHM of $4556 \pm 350\,\mathrm{km\,s^{-1}}$ for Ly$\alpha$, $3383 \pm 1000\,\mathrm{km\,s^{-1}}$ for N\,\textsc{v}$\,\lambda\lambda 1238,\,1242$, $10005 \pm 1000\,\mathrm{km\,s^{-1}}$ for Si\,\textsc{iv}$\,\lambda\lambda 1393,\,1402$ and $8428 \pm 500\,\mathrm{km\,s^{-1}}$ for C\textsc{iv}$\,\lambda\lambda 1548,\,1550$. It has to be noted that the FWHM of the  N\,\textsc{v}$\,\lambda\lambda 1238,\,1242$ profile has been estimated based on its red flank, as it is blended with the Ly$\alpha$ profile.
The FWHM of the Si\,\textsc{iv}$\,\lambda\lambda 1393,\,1402$ RMS profile has been excluded from the further analysis, due to its noise and possible overlap with O\,\textsc{iv}] emission. \\









\section{Black Hole Mass}

Based on the results of the reverberation campaign, virial products for the selected emission lines were derived and the black hole mass was estimated using a scale factor of $f=6.9$. The scale factor accounts for the low inclination ($i \sim 11^\circ$) of the modeled elliptical accretion disc in NGC\,4593 \parencite{ochmann2024transient}. It was estimated using the relation $f \sim \sin^{-2}(i)$ \parencite{krolik2001systematic} and adjusted based on $\sigma_{\mathrm{line}} \approx \mathrm{FWHM}/2$ \parencite{peterson2004}, since the velocity dispersion was parameterized using the FWHM rather than the line dispersion. After the mass estimates were obtained for each selected emission line, an inverse-variance–weighted mean was calculated: $\bar{M}_{\mathrm{BH}} \approx (3.35 \pm 0.62)\times 10^{7},M\odot$. Earlier works using similar approaches reported $M \approx 1.4 \times 10^7,M_\odot$ \parencite{kollatschny1997balmer} and $M = (9.8 \pm 2.1) \times 10^6,M_\odot$ \parencite{denney2006ngc4593}. These estimates were primarily based on H$\alpha$ \parencite{kollatschny1997balmer} and H$\beta$ \parencite{denney2006ngc4593}, respectively. Therefore, the weighted-mean mass based on the Balmer, helium, and UV lines is consistent in order of magnitude with previous estimates. It should be noted that these studies adopted different approximations for the scale factor, which can contribute to the differences in the reported values.

\section{Variation and Bowen Fluorescence of O\,\textsc{I}$\,\lambda8446$}
\label{sec:diskussion_BF}

The low-ionization line O\,\textsc{i}\,$\lambda\,8446$ was included in the RM analysis, as it exhibits noticeable variability in its RMS spectrum and has not yet been monitored in an HST reverberation-mapping campaign. O\,\textsc{i}\,$\lambda\,8446$ is known to be a Bowen fluorescence line pumped by Ly$\beta$ emission \parencite{grandi1980}. The wavelength coverage of the campaign, spanning parts of the UV and NIR region, allowed an examination of this process. Although Ly$\beta$ is not covered by the observations, Ly$\alpha$ has been used as a proxy to investigate its expected Bowen fluorescence connection to O\,\textsc{i}\,$\lambda\,8446$. As described in Section \ref{sec:bowen_fluorescence}, Ly$\beta$ excites O\,\textsc{i} through a near-resonant transition at $\lambda\,1025\,\AA$. This is followed by decay via O\,\textsc{i}\,$\lambda\,11287$ and subsequently by O\,\textsc{i}\,$\lambda\,8446$, producing the observed O\,\textsc{i}\,$\lambda\,8446$ emission (see Figure \ref{fig:bowen_cascate}). This cascade is expected to occur with a photon ratio of 1:1 \parencite{grandi1980}. \cite{ochmann2026first} reported a photon ratio of $\sim 0.8$ between O\,\textsc{i}\,$\lambda\,8446$ and O\,\textsc{i}\,$\lambda\,11287$ (taken from thenear-infrared SpeX spectra obtained by \cite{landt2008near}). This result indicates that Ly$\beta$ pumping is the main excitation mechanism for O\,\textsc{i}\,$\lambda\,8446$ in NGC\,4593. In addition, the comparison shown in Figure \ref{fig:ccfs_Bowen} demonstrates that O,\textsc{i}\,$\lambda\,8446$ exhibits a significantly stronger correlation with Ly$\alpha$ than with UVW2, further supporting this interpretation. Therefore, the emitting region of O\,\textsc{i} is likely located at a distance corresponding to a lag of $\sim 3.5$ days from the ionizing continuum, consistent with the Bowen fluorescence scenario, rather than the $5.0^{+2.1}_{-1.7}$ days implied by the direct lag relative to UVW2 under the assumption of recombination as the main excitation mechanism. The Bowen fluorescence lag of O\,\textsc{i} of corresponds to the distance of H$\alpha$ region from the ionizing continuum with $3.2 \ensuremath{_{-0.7}^{+1.0}}$. Since H$\alpha$ and O\,\textsc{i}\,$\lambda\,8446$ also exhibits a strong correlation with a minimal time shift of $\sim 0.3$ days, it can be inferred that O\,\textsc{i}\,$\lambda\,8446$ is emitted at approximately the same distance as H$\alpha$ within uncertainties.


 




