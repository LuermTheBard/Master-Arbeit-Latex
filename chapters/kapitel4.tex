\chapter{Discussion}

\section{Continua}
The observation campaign of \cite{cackett2018accretion} covers a extensive wavelength range of NGC\,4593, between $\sim 1100\AA$--$1700\AA$ and $\sim 3900 \AA$--$9000 \AA$. While several studies already conducted reverberation mapping analysis of the broad emission lines in the optical region of NGC\,4593 (e.g \cite{kollatschny1997balmer, denney2006ngc4593}), the wavelength range of the \cite{cackett2018accretion} campaign allowed to include the UV range in a RM analysis of this galaxy for the first time. In addition, the UV continuum around $1150\,\AA$ has been adopted as a proxy for the ionizing continuum rather than an optical continuum, which has been done in earlier studies (e.g. \cite{kollatschny1997balmer, denney2006ngc4593}). The extracted $1150\,\AA$ continuum light curve and its variability parameters have been compared to the corresponding results reported by \cite{cackett2018accretion} and show good agreement. The shape of the $1150\,\AA$ continuum light curve closely matches the more densely sampled \textit{Swift} UVOT UVW2 light curve presented by \cite{mchardy2018x}. Owing to its higher sample rate, the UVW2 light curve is adopted as the reference light curve for the RM analysis presented here. The extracted emission light curves also showed a similar but shifted shape and exhibited strong correlation with CCF maxima of $\sim 0.7$--$0.9$, with the exception for the NIR emission line O\,\textsc{i}$\,\lambda 8446$ that only shows a maxima of $\sim 0.6$ (see Section \ref{sec:diskussion_BF}). 

\section{Balmer Lines}

The Balmer lines light curves all exhibits a strong correlation with the UVW2 lightcurves resulting in CCF maxima with $\sim 0.8$--$0.9$. The RM analysis yielded that the first order Balmer lines H$\alpha$ and H$\beta$ lag with $3.2^{+1.0}_{-0.6}\,$ and $2.3^{+1.0}_{-0.3}\,$days, while the higher-order lines H$\gamma$ and H$\delta$ show a slightly shorter lag with $1.7 \ensuremath{_{-0.4}^{+0.2}}$ and $1.7 \ensuremath{_{-0.4}^{+0.4}}\,$days behind UVW2 within uncertainties. With UVW2 lag about $\sim 0.5$ days behind the $1150\,\AA$ continuum, this infers a characteristic radius of the emitting region of $\sim 3.7\,$days for H$\alpha$, $\sim 2.8\,$days for H$\beta$ and $\sim 2.2\,$days for H$\gamma$ and H$\delta$. The found radii for H$\alpha$ and H$\beta$ are in good agreement with earlier studies (eg. \cite{kollatschny1997balmer,denney2006ngc4593,Williams_2018}).\\
The RMS line profiles exhibit similar asymmetric, double-peaked shape shifted to positive velocities. Especially H$\alpha$ and H$\beta$ as well as H$\gamma$ and H$\delta$ showed strong similarities in shape, with a steep blue flank for H$\alpha$ and H$\beta$ and a more extended blue flank H$\gamma$ and H$\delta$. This and the small diffing time lags indicates, that H$\gamma$ and H$\delta$ originate from a closer emitting region under a slightly differed kinematic properties, that H$\alpha$ and H$\beta$. As H$\gamma$ and H$\delta$ show higher FWHM values compared to H$\alpha$ and H$\beta$ within uncertainties, this strengthens this assumption. The found FWHM of H$\alpha$ and H$\beta$ the profiles in good agreement with earlier reports of the line width (eg. \cite{kollatschny1997balmer, denney2006ngc4593}). 




\section{Helium Lines}




\section{UV Lines}


The UV lines showed overall a significant shorter time lag to the continuum variations, with centroid lags of order $\sim 1$ day for Ly$\alpha$, 	N\,\textsc{v}$\,\lambda\lambda 1238,\,1242$, Si\,\textsc{iv}$\,\lambda\lambda 1393,\,1402$ and C\textsc{iv}$\,\lambda\lambda 1548,\,1550$ relative to UVW2. These short lags indicate that the high-ionization UV-emitting gas is located closer to the ionizing source than the Balmer and He\,\textsc{i} regions.













\section{Black Hole Mass}

Based on the results of the reverberation campaign, virial products for the selected emission lines were derived and the black hole mass was estimated using a scale factor of $f=6.9$. The scale factor accounts for the low inclination ($i \sim 11^\circ$) of the modeled elliptical accretion disc in NGC\,4593 \parencite{ochmann2024transient}. It was estimated using the relation $f \sim \sin^{-2}(i)$ \parencite{krolik2001systematic} and adjusted based on $\sigma_{\mathrm{line}} \approx \mathrm{FWHM}/2$ \parencite{peterson2004}, since the velocity dispersion was parameterized using the FWHM rather than the line dispersion. After the mass estimates were obtained for each selected emission line, an inverse-variance–weighted mean was calculated: $\bar{M}_{\mathrm{BH}} \approx (3.35 \pm 0.62)\times 10^{7},M\odot$. Earlier works using similar approaches reported $M \approx 1.4 \times 10^7,M_\odot$ \parencite{kollatschny1997balmer} and $M = (9.8 \pm 2.1) \times 10^6,M_\odot$ \parencite{denney2006ngc4593}. These estimates were primarily based on H$\alpha$ \parencite{kollatschny1997balmer} and H$\beta$ \parencite{denney2006ngc4593}, respectively. Therefore, the weighted-mean mass based on the Balmer, helium, and UV lines is consistent in order of magnitude with previous estimates. It should be noted that these studies adopted different approximations for the scale factor, which can contribute to the differences in the reported values.

\section{Variation and Bowen Fluorescence of O\,\textsc{I}$\,\lambda8446$}
\label{sec:diskussion_BF}

The low-ionization line O\,\textsc{i}\,$\lambda\,8446$ was included in the RM analysis, as it exhibits noticeable variability in its RMS spectrum and has not yet been monitored in an HST reverberation-mapping campaign. O\,\textsc{i}\,$\lambda\,8446$ is known to be a Bowen fluorescence line pumped by Ly$\beta$ emission \parencite{grandi1980}. The wavelength coverage of the campaign, spanning parts of the UV and NIR region, allowed an examination of this process. Although Ly$\beta$ is not covered by the observations, Ly$\alpha$ has been used as a proxy to investigate its expected Bowen fluorescence connection to O\,\textsc{i}\,$\lambda\,8446$. As described in Section \ref{sec:bowen_fluorescence}, Ly$\beta$ excites O\,\textsc{i} through a near-resonant transition at $\lambda\,1025\,\AA$. This is followed by decay via O\,\textsc{i}\,$\lambda\,11287$ and subsequently by O\,\textsc{i}\,$\lambda\,8446$, producing the observed O\,\textsc{i}\,$\lambda\,8446$ emission (see Figure \ref{fig:bowen_cascate}). This cascade is expected to occur with a photon ratio of 1:1 \parencite{grandi1980}. \cite{ochmann2026first} reported a photon ratio of $\sim 0.8$ between O\,\textsc{i}\,$\lambda\,8446$ and O\,\textsc{i}\,$\lambda\,11287$ (taken from thenear-infrared SpeX spectra obtained by \cite{landt2008near}). This result indicates that Ly$\beta$ pumping is the main excitation mechanism for O\,\textsc{i}\,$\lambda\,8446$ in NGC\,4593. In addition, the comparison shown in Figure \ref{fig:ccfs_Bowen} demonstrates that O,\textsc{i}\,$\lambda\,8446$ exhibits a significantly stronger correlation with Ly$\alpha$ than with UVW2, further supporting this interpretation. Therefore, the emitting region of O\,\textsc{i} is likely located at a distance corresponding to a lag of $\sim 3.5$ days from the ionizing continuum, consistent with the Bowen fluorescence scenario, rather than the $5.0^{+2.1}_{-1.7}$ days implied by the direct lag relative to UVW2 under the assumption of recombination as the main excitation mechanism. The Bowen fluorescence lag of O\,\textsc{i} of corresponds to the distance of H$\alpha$ region from the ionizing continuum with $3.2 \ensuremath{_{-0.7}^{+1.0}}$. Since H$\alpha$ and O\,\textsc{i}\,$\lambda\,8446$ also exhibits a strong correlation with a minimal time shift of $\sim 0.3$ days, it can be inferred that O\,\textsc{i}\,$\lambda\,8446$ is emitted at approximately the same distance as H$\alpha$ within uncertainties.


 




