\chapter{Discussion}

\section{Reverberation Mapping UV-NIR Broad Emission Lines}

The observation campaign of \cite{cackett2018accretion} covers a extensive wavelength range of NGC\,4593, between $\sim 1100\AA$--$1700\AA$ and $\sim 3900 \AA$--$9000 \AA$. While several studies already conducted reverberation mapping analysis of the broad emission lines in the optical region of NGC\,4593 (e.g \cite{kollatschny1997balmer, denney2006ngc4593}), the wavelength range of the \cite{cackett2018accretion} campaign allowed to include broad emission lines from UV range, as well as the NIR low-ionization line O\,\textsc{i}\,$\lambda\,8446$ into a RM analysis of this galaxy for the first time. 







\textbf{Continua Variation vergleichen und auf intercalibration eingehen}




\textbf{Reference to earlier studies.}

\textbf{Besonders auf UV eingehen}


\section{Variation and Bowen Fluorescence of O\,\textsc{I}$\,\lambda8446$}





\section{Black Hole Mass}

The mass of the SMBH was estimated by deriving the virial mass for each selected emission line and multiplied it by a correction factor $f$, following \cite{peterson2004}. As the FWHM of the lines was used to parametrize its velocity dispersion a correction factor $f=1.8$ has bin adopted \parencite{probst2025emissionlinecontinuumreverberationmapping}. This yields in a uncertainty-weighed-mean SMBH mass of $\bar{M}_{\mathrm{BH}}\approx (0.89 \pm 0.16)\times 10^{7}\,M_\odot$.

\textbf{Reference to earlier studies.}

\textbf{first time UV lines are measured for NGC4593 !}



