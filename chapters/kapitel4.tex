\chapter{Discussion}

\section{Emission Lines and Continua}

The observation campaign of \cite{cackett2018accretion} covers a extensive wavelength range of NGC\,4593, between $\sim 1100\AA$--$1700\AA$ and $\sim 3900 \AA$--$9000 \AA$. Several broad emission lines could been identified in the optical to NIR range, as well as in the UV range. While several studies already conducted reverberation mapping analysis of the broad emission lines of NGC\,4593 (e.g \cite{kollatschny1997balmer, denney2006ngc4593}), they mainly covered optical lines (e.g. Balmer lines). The range of this campaigns enabled to include UV emission lines into a RM analysis of NGC\,4593 for the first time.

\section{Line Profiles}

\section{Time Lags}


\section{Bowen Fluorescence of O\,\textsc{I}$\,\lambda8446$}

\section{Black Hole Mass}



\textbf{Reference to earlier studies.}

\textbf{first time UV lines are measured for NGC4593 !}

\textbf{Bowen Fluorescence}

