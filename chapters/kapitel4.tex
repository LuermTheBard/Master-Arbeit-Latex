\chapter{Discussion}

The nearby Seyfert I galaxy NGC\,4593 has been monitored in several earlier RM studies (e.g. \cite{kollatschny1997balmer,denney2006ngc4593, cackett2018accretion}). In a recent work, \cite{cackett2018accretion} obtained an observation campaign of NGC\,4593 with HST including the optical to NIR range between $\sim 3900 \AA$--$9000 \AA$ as well as the range of $\sim 1100\AA$--$1700\AA$ of the UV region. Only a few other AGN have been monitored by HST, and the high cadence of nearly daily observations made this campaign of particular interests for a RM campaign. It enabled us to conduct RM analysis of the BLR of NGC\,4593, including broad emission lines that had not been monitored before (e.g. UV lines) and examine the BLR stratification of NGC\,4593. Also it enabled to use the UV continuum around $1150\,\AA$ to use as a proxy for the ionized continuum rather than a optical continuum used in earlier works (e.g. \cite{kollatschny1997balmer, denney2006ngc4593}). As the higher ionized continua, the UV continuum is assumed to origin closer from the SMBH.

\section{Structure of the BLR}






\section{Continua}
The observation campaign of \cite{cackett2018accretion} covers a extensive wavelength range of NGC\,4593, between $\sim 1100\AA$--$1700\AA$ and $\sim 3900 \AA$--$9000 \AA$. While several studies already conducted reverberation mapping analysis of the broad emission lines in the optical region of NGC\,4593 (e.g \cite{kollatschny1997balmer, denney2006ngc4593}), the wavelength range of the \cite{cackett2018accretion} campaign allowed to include the UV range in a RM analysis of this galaxy for the first time. Therefore, the UV continuum around $1150\,\AA$ has been adopted as a proxy for the ionizing continuum rather than an optical continuum, which has been done in earlier studies (e.g. \cite{kollatschny1997balmer, denney2006ngc4593}). The shape of the $1150\,\AA$ continuum light curve closely matches the more densely sampled \textit{Swift} UVOT UVW2 light curve presented by \cite{mchardy2018x}. Owing to its higher sample rate, the UVW2 light curve is adopted as the reference light curve for the RM analysis presented here.

\section{Balmer Lines}
RM studies of NGC 4593 have traditionally monitored H$\alpha$ and H$\beta$ (e.g. \cite{kollatschny1997balmer,denney2006ngc4593,Williams_2018}). Given the strong emission of H$\gamma$ and H$\delta$ in the campaign of \cite{cackett2018accretion}, these higher-order Balmer lines are included as well. All Balmer line light curves show a strong correlation with the UVW2 continuum light curve, reaching a peak CCF of roughly $0.8$--$0.9$. Time lags of $3.2^{+1.0}_{-0.6}\,$ and $2.3^{+1.0}_{-0.3}\,$days for H$\alpha$ and H$\beta$ relative to UVW2 have been found, which are in good agreement with earlier studies (eg. \cite{kollatschny1997balmer,denney2006ngc4593,Williams_2018}). H$\gamma$ and H$\delta$ show slightly shorter delays, with lags of $1.7^{+0.2}_{-0.4}\,$days and $1.7^{+0.4}_{-0.4}\,$days behind UVW2. Since UVW2 itself lags the $1150\,\AA$ continuum by about $\sim 0.5$ days, this translates to BLR radii of $\sim 3.7\,$days for H$\alpha$, $\sim 2.8\,$days for H$\beta$, and $\sim 2.2\,$days for H$\gamma$ and H$\delta$.\\
The RMS line profiles of all Balmer lines show a similar asymmetric, double-peaked structure and are shifted toward positive velocities. This shift indicates that most of the line variability originates from the receding part of the emitting region.\\
Looking in more detail, H$\beta$ shows a strong similarity with H$\alpha$, as well as H$\delta$ to H$\gamma$. H$\alpha$ and H$\beta$ show a comparatively steeper blue wing, whereas H$\gamma$ and H$\delta$ exhibit a more extended blue wing. Together with the shorter time lags, this supports the idea that H$\gamma$ and H$\delta$ originate from a slightly smaller radius and may probe somewhat different kinematic conditions than H$\alpha$ and H$\beta$, which also can be seen in the line width. The measured FWHM of H$\alpha$ with $3111 \pm 250\,\mathrm{km\,s^{-1}}$ and H$\beta$ with $3437 \pm 200\,\mathrm{km\,s^{-1}}$ agree well with previously reported line widths (e.g. \cite{kollatschny1997balmer, denney2006ngc4593}). Within uncertainties, H$\gamma$ and H$\delta$ show larger FWHM values of $3852 \pm 400,\mathrm{km,s^{-1}}$ and $5905 \pm 1000,\mathrm{km,s^{-1}}$, respectively, compared to H$\alpha$ and H$\beta$. 


\section{Helium Lines}

In addition to the Balmer lines, the helium lines He\,\textsc{ii}$\,\lambda4686$, He\,\textsc{ii}$\,\lambda1640$, and He\,\textsc{i}$\,\lambda5876$ have been included in this RM analysis. Like the balmer lines light curves, all helium-line light curves track the overall variability pattern of UVW2. A larger relative delay is apparent for He\,\textsc{i}$\,\lambda5876$ compared to the He\,\textsc{ii} lines. Their correlation with the UVW2 light curves peak at about $\sim 0.7$ for He\,\textsc{ii}$\,\lambda1640$ and He\,\textsc{i}$\,\lambda5876$ at about $\sim 0.85$ for He\,\textsc{ii}$\,\lambda4686$. He\,\textsc{i}$\,\lambda5876$ lags  $3.3^{+0.7}_{-0.4}\,$days behind UVW2, consistent with the H$\alpha$ lag within uncertainties. In contrast, the He\,\textsc{ii} lines respond significantly faster, with lags of $0.7^{+0.1}_{-0.3}\,$days for He\,\textsc{ii}$\,\lambda4686$ and $0.3^{+0.6}_{-0.2}\,$days for He\,\textsc{ii}$\,\lambda1640$. This implies that the He\,\textsc{ii} lines exhibits a much smaller BLR radius of about $\sim 1\,$day than He\,\textsc{i} with about $3.8\,$days, by accounting for the lag between the UVW2 and the $1150\,\AA$ continuum.\\
Like the RMS profile of the Balmer lines, the maximum of the He\,\textsc{i}$\,\lambda5876$ RMS profile is shifted to positive velocities and exhibits a FWHM of $3952 \pm 400\,\mathrm{km\,s^{-1}}$. The He\,\textsc{ii} profiles are significantly broader with FWHM values of $5971 \pm 400\,\mathrm{km\,s^{-1}}$ for He\,\textsc{ii}$\,\lambda4686$ and $6891 \pm 600\,\mathrm{km\,s^{-1}}$ for He\,\textsc{ii}$\,\lambda1640$. 



\section{UV Lines}

The wavelength coverage of the \cite{cackett2018accretion} campaign allowed to include the UV emission lines Ly$\alpha$, N\,\textsc{v}$\,\lambda\lambda 1238,\,1242$, Si\,\textsc{iv}$\,\lambda\lambda 1393,\,1402$ and C\textsc{iv}$\,\lambda\lambda 1548,\,1550$ in a RM analysis of NGC\,4593 for the first time. With exception of the previously discussed UV He\,\textsc{i} line, the correlation of the  UV emission line light curves with the UVW2 light curves peak at about $\sim 0.8$ -- $0.9$, comparable to the Balmer and He,\textsc{ii} lines. The lags relative to UVW2 have been founded with $1.0 \ensuremath{_{-0.2}^{+0.4}}$, $1.0 \ensuremath{_{-0.4}^{+0.2}}$, $1.2 \ensuremath{_{-0.4}^{+0.1}}$ and $0.8 \ensuremath{_{-0.1}^{+0.5}}$ days for Ly$\alpha$, N\,\textsc{v}$\,\lambda\lambda 1238,\,1242$, Si\,\textsc{iv}$\,\lambda\lambda 1393,\,1402$ and C\textsc{iv}$\,\lambda\lambda 1548,\,1550$, respectively. Within uncertainties, these lags are mutually consistent, suggesting that the UV lines mainly originate at similar BLR radii from the central SMBH. Accounting for the lag between UVW2 and the $1150\,\AA$ continuum, this corresponds to a BLR radius of approximately $\sim1.5\,$days and is therefor located between the emitting region of the He\,\textsc{i} lines and the emitting region of the Balmer and He\,\textsc{ii} lines. \\
The FHWM of the UV emission lines have been found with $4556 \pm 350\,\mathrm{km\,s^{-1}}$ for Ly$\alpha$, $3383 \pm 1000\,\mathrm{km\,s^{-1}}$ for N\,\textsc{v}$\,\lambda\lambda 1238,\,1242$, $10005 \pm 1000\,\mathrm{km\,s^{-1}}$ for Si\,\textsc{iv}$\,\lambda\lambda 1393,\,1402$ and $8428 \pm 500\,\mathrm{km\,s^{-1}}$ for C\textsc{iv}$\,\lambda\lambda 1548,\,1550$. The FWHM of N\,\textsc{v}$\,\lambda\lambda1238,\,1242$ is estimated from the red wing only, since the blue side is blended with Ly$\alpha$. Also the FWHM of the Si\,\textsc{iv}$\,\lambda\lambda 1393,\,1402$ RMS profile is not suitable for further interpretations, due to the noisy profile and possible blending by O\,\textsc{iv}] emission. 



\section{Black Hole Mass}

Based on the reverberation mapping results, virial products were derived for the selected emission lines, and the SMBH mass was estimated using a scale factor of $f=6.9$. The adopted scale factor accounts for the low inclination angle of $i \sim 11^\circ$ reported by \cite{ochmann2024transient}. Its estimation is based on the relation $f \sim \sin^{-2}(i)$ \parencite{krolik2001systematic}, which captures the inclination dependence of the observed line-of-sight velocity dispersion. Since the velocity width is parameterized via FWHM in this work rather than the line dispersion, an additional conversion has been applied using $\sigma_{\mathrm{line}} \approx \mathrm{FWHM}/2$ \parencite{peterson2004}. With that a inverse-variance–weighted mean of the BH mass has been found with $\bar{M}_{\mathrm{BH}} \approx (3.35 \pm 0.62)\times 10^{7}\,M_\odot$. Earlier works using similar approaches reported masses of $M \approx 1.4 \times 10^7\,M_\odot$ \parencite{kollatschny1997balmer} and $M = (9.8 \pm 2.1) \times 10^6\,M_\odot$ \parencite{denney2006ngc4593}. The mass obtained here is therefore higher by a factor of $\sim2$--$3$, while remaining within the same order of magnitude. These difference may occur through the estimation over several emission lines in this work and also from different assumptions in the adopted scale factor used for the scale factor in the mass estimation.

\section{Variation and Bowen Fluorescence of O\,\textsc{I}$\,\lambda8446$}
\label{sec:diskussion_BF}

The low-ionization line O\,\textsc{i}\,$\lambda\,8446$ was included in the RM analysis, as it exhibits noticeable variability in its RMS spectrum and has not yet been monitored in an HST reverberation-mapping campaign. O\,\textsc{i}\,$\lambda\,8446$ is known to be a Bowen fluorescence line pumped by Ly$\beta$ emission \parencite{grandi1980}. The wavelength coverage of the campaign, spanning parts of the UV and NIR region, allowed an examination of this process. Although Ly$\beta$ is not covered by the observations, Ly$\alpha$ has been used as a proxy to investigate its expected Bowen fluorescence connection to O\,\textsc{i}\,$\lambda\,8446$. As described in Section \ref{sec:bowen_fluorescence}, Ly$\beta$ excites O\,\textsc{i} through a near-resonant transition at $\lambda\,1025\,\AA$. This is followed by decay via O\,\textsc{i}\,$\lambda\,11287$ and subsequently by O\,\textsc{i}\,$\lambda\,8446$, producing the observed O\,\textsc{i}\,$\lambda\,8446$ emission (see Figure \ref{fig:bowen_cascate}). This cascade is expected to occur with a photon ratio of 1:1 \parencite{grandi1980}. \cite{ochmann2026first} reported a photon ratio of $\sim 0.8$ between O\,\textsc{i}\,$\lambda\,8446$ and O\,\textsc{i}\,$\lambda\,11287$ (taken from the near-infrared SpeX spectra obtained by \cite{landt2008near}). This result indicates that Ly$\beta$ pumping is the main excitation mechanism for O\,\textsc{i}\,$\lambda\,8446$ in NGC\,4593. In addition, the comparison shown in Figure \ref{fig:ccfs_Bowen} demonstrates that O\,\textsc{i}\,$\lambda\,8446$ exhibits a significantly stronger correlation with Ly$\alpha$ than with UVW2, further supporting this interpretation. Therefore, the emitting region of O\,\textsc{i} is likely located at a distance corresponding to a lag of $\sim 3.5$ days from the ionizing continuum, consistent with the Bowen fluorescence scenario, rather than the $5.0^{+2.1}_{-1.7}$ days implied by the direct lag relative to UVW2 under the assumption of recombination as the main excitation mechanism. With that, the Bowen fluorescence lag of O\,\textsc{i} corresponds to the radius of the emitting region of H$\alpha$ with $3.2 \ensuremath{_{-0.7}^{+1.0}}\,$ days. Since H$\alpha$ and O\,\textsc{i}\,$\lambda\,8446$ also exhibits a strong correlation with a minimal time shift of $\sim 0.3$ days, it can be inferred that O\,\textsc{i}\,$\lambda\,8446$ is emitted at approximately the same radius as H$\alpha$ within uncertainties.


 




