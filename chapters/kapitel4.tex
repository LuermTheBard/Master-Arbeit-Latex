\chapter{Discussion}



The observation campaign of \cite{cackett2018accretion} covers a extensive wavelength range of NGC\,4593, between $\sim 1100\AA$--$1700\AA$ and $\sim 3900 \AA$--$9000 \AA$. While several studies already conducted reverberation mapping analysis of the broad emission lines in the optical region of NGC\,4593 (e.g \cite{kollatschny1997balmer, denney2006ngc4593}), the wavelength range of the \cite{cackett2018accretion} campaign allowed to include broad emission lines from UV range in a RM analysis and BH mass estimation of this galaxy for the first time. The low-ionization line O\,\textsc{i}\,$\lambda\,8446$ was also included in the RM analysis, as it exhibiting noticeable variation in its RMS spectrum, which has not been monitored in a HST RM campaign yet. O\,\textsc{i}\,$\lambda\,8446$ is know to be a Bowen fluorescence line, pumped by Ly$\beta$ emission \parencite{grandi1980}. As the campaign covers the Ly$\alpha$ line, it has been used as a proxy to examine its expected Bowen fluorescence relation to the O\,\textsc{i}\,$\lambda\,8446$ line.


\section{Reverberation Mapping}

The wavelength range for the continua light curves had been adapted from \cite{cackett2018accretion} to accommodated on the intercalibrated RMS. The pattern of the light curves and their fractional variation are in agreement with the derived values of the equivalent ranges in \cite{cackett2018accretion}. As the light curve of the continuum around $1150\,\AA$ to the \textit{Swift} UVOT UVW2 light curve \parencite{mchardy2018x} showed a very similar shape

\section{Variation and Bowen Fluorescence of O\,\textsc{I}$\,\lambda8446$}

 




\section{Black Hole Mass}

Based on the result of the reverberation campaign, it was possible to derive the virial mass of the selected emission lines and estimate the BH mass using the scale factor $f=6.9$. The scaling factor accounts for the low-inclination $i\sim 11$ of the modelled elliptic accretion disc fo NGC\,4593 \parencite{ochmann2024transient} and was estimated with the scaling relation $f\sim \sin^{-2}(i)$ \parencite{krolik2001systematic} and adjusted base on the relation $\sigma_{\mathrm{line}} \approx \mathrm{FWHM}/2$ \parencite{peterson2004} as the velocity dispersion has been parametrized by the FWHM rather than the line dispersion. After the mass estimation for each of the selected emission line, a inverse-variance weighted mean has been calculated: $\bar{M}_{\mathrm{BH}} \approx (3.35 \pm 0.62)\times 10^{7}\,M_\odot$. Earlier works that conducted a similar method for the mass estimation reported values of $M \approx 1.4 \times 10^7\,M_\odot$ \cite{kollatschny1997balmer} and $M = (9.8 \pm 2.1) \times 10^6\,M_\odot$ \cite{denney2006ngc4593}, focused mainly on H$\alpha$ \cite{kollatschny1997balmer} or H$\beta$ \cite{denney2006ngc4593} respectively. Therefore, the weighted mean of the mass, based on the Balmer, Helium and UV lines agree in order of magnitude, but yields a slightly higher value. It has to be noted that the named studies used different approximation for the scaling factor, which can explain the diffing values. 
