\chapter{Discussion}






\section{Reverberation Mapping}

The observation campaign of \cite{cackett2018accretion} covers a extensive wavelength range of NGC\,4593, between $\sim 1100\AA$--$1700\AA$ and $\sim 3900 \AA$--$9000 \AA$. While several studies already conducted reverberation mapping analysis of the broad emission lines in the optical region of NGC\,4593 (e.g \cite{kollatschny1997balmer, denney2006ngc4593}), the wavelength range of the \cite{cackett2018accretion} campaign allowed to include broad emission lines from UV range in a RM analysis of this galaxy for the first time. \\
Therefore continua light curve from the UV, optical and NIR range has been extracted. The wavelength range for the continua light curves has been selected based on the ranges adopted from \cite{cackett2018accretion}. The pattern of the light curves and their fractional variation are in agreement with the derived values of the equivalent ranges in \cite{cackett2018accretion}. The shape of the continuum light curve around $1150\,\AA$ showed also a strong similarity to the higher sampled \textit{Swift} UVOT UVW2 light curve \parencite{mchardy2018x}, which has been used as the reference light curve for the ionizing continuum for the subsequent analysis. \\


\section{Line Profiles}



\section{Black Hole Mass}

Based on the result of the reverberation campaign, it was possible to derive the virial mass of the selected emission lines and estimate the BH mass using the scale factor $f=6.9$. The scaling factor accounts for the low-inclination $i\sim 11$ of the modelled elliptic accretion disc fo NGC\,4593 \parencite{ochmann2024transient} and was estimated with the scaling relation $f\sim \sin^{-2}(i)$ \parencite{krolik2001systematic} and adjusted base on the relation $\sigma_{\mathrm{line}} \approx \mathrm{FWHM}/2$ \parencite{peterson2004} as the velocity dispersion has been parametrized by the FWHM rather than the line dispersion. After the mass estimation for each of the selected emission line, a inverse-variance weighted mean has been calculated: $\bar{M}_{\mathrm{BH}} \approx (3.35 \pm 0.62)\times 10^{7}\,M_\odot$. Earlier works that conducted a similar method for the mass estimation reported values of $M \approx 1.4 \times 10^7\,M_\odot$ \cite{kollatschny1997balmer} and $M = (9.8 \pm 2.1) \times 10^6\,M_\odot$ \cite{denney2006ngc4593}, focused mainly on H$\alpha$ \cite{kollatschny1997balmer} or H$\beta$ \cite{denney2006ngc4593} respectively. Therefore, the weighted mean of the mass, based on the Balmer, Helium and UV lines agree in order of magnitude, but yields a slightly higher value. It has to be noted that the named studies used different approximation for the scaling factor, which also can result in the differences of the values. 

\section{Variation and Bowen Fluorescence of O\,\textsc{I}$\,\lambda8446$}
The low-ionization line O\,\textsc{i}\,$\lambda\,8446$ was also included in the RM analysis, as it exhibiting noticeable variation in its RMS spectrum, which has not been monitored in a HST RM campaign yet. O\,\textsc{i}\,$\lambda\,8446$ is also know to be a Bowen fluorescence line, pumped by Ly$\beta$ emission \parencite{grandi1980}. The range of the campaign covering parts of the UV range and the NIR range, allowed to examine this process. While Ly$\beta$ is still not covered, the Ly$\alpha$ line has been used as a proxy to examine its expected Bowen fluorescence relation to the O\,\textsc{i}\,$\lambda\,8446$ line. As described in Section \ref{sec:bowen_fluorescence} Ly$\beta$ excites OI through a near-resonant transition of $\lambda\,1025\,\AA$, followed by decay with the transition O\,\textsc{i}\,$\lambda\,11287$ and subsequent with the transition O\,\textsc{i}\,$\lambda\,8446$ resulting in the O\,\textsc{i}\,$\lambda\,8446$ emission line (see Figure \ref{fig:bowen_cascate}), which should happen with a photon ratio of 1:1 \parencite{grandi1980}. \cite{ochmann2026first} showed a photon ratio between the O\,\textsc{i}\,$\lambda\,8446$ line and the O\,\textsc{i}\,$\lambda11287$, taken from the near-infrared SpeX spectrum observed by \cite{landt2008near}, of $\sim 0.8$, following that Ly$\beta$ pumping is the major excitation mechanism for the O\,\textsc{i}\,$\lambda\,8446$ line in NGC\,4593. In addition, the comparison in Figure \ref{fig:ccfs_Bowen} showed that O\,\textsc{i}\,$\lambda\,8446$ exhibits a much stronger correlation with Ly$\alpha$ than with UVW2, which indicated the same conclusion. Therefore, the emitting region of OI should be located in a distance of $\sim 3.5$ days from the ionizing continuum, following the Bowen fluorescence path rather than $5.0 \ensuremath{_{-1.7}^{+2.1}}$ days implied by the direct lag of OI to UVW2 in the case of recombination. The Bowen fluorescence lag of O\,\textsc{i} of corresponds to the distance of H$\alpha$ region from the ionizing continuum with $3.2 \ensuremath{_{-0.7}^{+1.0}}$. Because H$\alpha$ and O\,\textsc{i}\,$\lambda\,8446$ also exhibits a strong correlation with a minimal time shift of $\sim 0.3$ days, it could be followed, that O\,\textsc{i}\,$\lambda\,8446$ is emitted at the same distance as H$\alpha$ within uncertainties.


 




