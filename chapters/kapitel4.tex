\chapter{Discussion}

The nearby Seyfert I galaxy NGC\,4593 has been monitored in several earlier RM studies (e.g. \cite{kollatschny1997balmer,denney2006ngc4593, cackett2018accretion}). In a recent study, \cite{cackett2018accretion} conducted an observing campaign of NGC\,4593 with HST (PI: Cackett, E.M, Prop ID 14121), covering the optical to near-infrared range between $\sim 3900\,\AA$–$9000\,\AA$ as well as the UV range between $\sim 1100\,\AA$–$1700\,\AA$. Only a few other AGN have been monitored by HST, and the high cadence of nearly daily observations makes this campaign of particular interests for a RM campaign. This enabled a RM analysis of the BLR of NGC\,4593 based on a HST/STIS campaign for the first time, including broad emission lines that had not previously been monitored (e.g. UV lines), and allowed an examination of the BLR structure of the galaxy. In addition the Bowen fluorescent emission line O\,\textsc{i}\,$\lambda\,8446$ was included. Given the wavelength coverage of the campaign, it was possible to examine its Bowen fluorescence relation to Ly$\beta$, which has not previously been monitored in a RM campaign. Also, the UV continuum around $1150\,\AA$ was used as a proxy for the ionizing continuum instead of an optical continuum, as adopted in earlier works (e.g. \cite{kollatschny1997balmer, denney2006ngc4593}), since it is assumed to originate closer to the SMBH \parencite{collin1988broad}. The more densely sampled \textit{Swift} UVOT UVW2 light curve presented by \cite{mchardy2018x} has was compared to the light curve of the $1150\,\AA$ continuum and showed a strong similarity. Owing to its higher cadence, the UVW2 light curve is adopted as the reference light curve for the RM analysis presented here. This analysis covered the UV emission lines Ly$\alpha$, N\,\textsc{v}$\,\lambda\lambda 1238,\,1242$, Si\,\textsc{iv}$\,\lambda\lambda 1393,\,1402$, C\textsc{iv}$\,\lambda\lambda 1548,\,1550$ and He\,\textsc{ii}$\,\lambda1640$, the optical emission lines H$\alpha$ - H$\delta$, He\,\textsc{ii}$\,\lambda4686$ and He\,\textsc{i}$\,\lambda5876$ and the NIR emission line O\,\textsc{i}\,$\lambda\,8446$, which allowed a extended structural analysis of the BLR of NGC\,4593 compared to earlier campaigns \parencite{kollatschny1997balmer, denney2006ngc4593, Williams_2018}.  

\section{Structure of the BLR}
\subsection{Time Lags}
The extracted light curves of the emission lines showed an overall similar shape, with strong correlation with the UVW2 light curve and CCF peak values between $0.7$--$0.9$. Only the O\,\textsc{i}\,$\lambda\,8446$ light curve showed a lower peak correlation with UVW2 of $0.6$, as its variability is primary driven through Bowen fluorescence, which will be further discussed in Section \ref{sec:diskussion_BF}.\\
H$\alpha$ and H$\beta$ showed different time lags of $3.3^{+1.1}_{-0.6}\,$ and $2.3^{+1.0}_{-0.3}\,$days which are in good agreement with the results of earlier works \parencite{kollatschny1997balmer, denney2006ngc4593, Williams_2018}. The higher-order Balmer lines H$\gamma$ and H$\delta$ show a significantly shorter time lags with $1.7^{+0.2}_{-0.4}\,$days and $1.7^{+0.4}_{-0.4}\,$days, which follows the trend seen in other Seyfert galaxies that higher ordered Balmer lines show shorter time lags compared to the lower ordered Balmer lines (e.g. \cite{gaskell1986line, kollatschny2001short}). It was also possible to infer the time lag of  Ly$\alpha$, which shows a even shorter time lag of $1.0 \ensuremath{_{-0.2}^{+0.4}}\,$days. \\ The helium lines show differences in time lag as well, depending on their ionization.
He\,\textsc{i}$\,\lambda5876$ lags $3.3^{+0.7}_{-0.4}\,$days behind UVW2, while the higher-ionized He\,\textsc{ii}$\,\lambda1640$ and He\,\textsc{ii}$\,\lambda4686$ exhibit a much faster response times, with lags of $0.3^{+0.6}_{-0.2}\,$days and $0.7^{+0.1}_{-0.3}\,$days, respectively. Therefore the emitting region of He\,\textsc{i} is located at the same distance as the H$\alpha$ weighted BLR, while the He\,\textsc{ii} lines originate much shorter distance to the ionizing continuum.
The high-ionized UV lines all show similar lags within uncertainties, with measured values of $1.0 \ensuremath{_{-0.4}^{+0.2}}$, $1.2 \ensuremath{_{-0.4}^{+0.1}}$ and $0.8 \ensuremath{_{-0.1}^{+0.5}}$ days for N\,\textsc{v}$\,\lambda\lambda 1238,\,1242$, Si\,\textsc{iv}$\,\lambda\lambda 1393,\,1402$ and C\textsc{iv}$\,\lambda\lambda 1548,\,1550$, respectively. Therefore, the broad emission lines show a tendency that higher ionized broad emission lines exhibit shorter time lags than lower ionized broad emission lines, which implies an ionization stratification structure of the BLR that has been reported in other Seyfert galaxies as well (e.g \cite{collin1988broad, gaskell1986line, kollatschny2001short}).\\
Since UVW2 lags behind the extracted HST/STIS $1150\,\AA$ continuum by approximately $0.5\,$days, this offset must be added to obtain the final BLR radii for each emission line.

\subsection{Line Profiles}
To draw conclusions about the kinematics of the BLR, the RMS line profiles of the broad emission lines have been obtained. \\
The Balmer line profiles show a similar asymmetric, double-peaked structure and are shifted toward positive velocities. This shift is also noticeable in the Ly$\alpha$ profile and He\,\textsc{i}$\,\lambda5876$ profile. This indicates that most of the line variability of these lines originates from the receding part of the emitting region.\\
Opposing to that, the maxima of the higher-ionization He\,\textsc{ii}$\,\lambda1640$ and He\,\textsc{ii}$\,\lambda4686$ profiles are located around their rest-frame wavelengths. A similar behavior is observed in the RMS profiles of the high-ionized UV lines N\,\textsc{v}$\,\lambda\lambda 1238,\,1242$, Si\,\textsc{iv}$\,\lambda\lambda 1393,\,1402$ and C\textsc{iv}$\,\lambda\lambda 1548,\,1550$. High-ionized broad emission lines often show a blue-shift compared to lower-ionized broad emission lines \parencite{gaskell1982redshift}, which could be an explanation for the shift in the RMS profile. \\
H$\alpha$ and H$\beta$ show comparatively steeper blue wings, whereas H$\gamma$ and H$\delta$ exhibit more extended blue wings, implying that they may probe somewhat different kinematic conditions than H$\alpha$ and H$\beta$, which is also reflected in their line widths. The measured FWHM values of H$\alpha$ with $3100 \pm 250\,\mathrm{km\,s^{-1}}$ agree well with previously reported line widths, while  H$\beta$ with $3400 \pm 200\,\mathrm{km\,s^{-1}}$ shows a slightly smaller value(e.g. \cite{kollatschny1997balmer, denney2006ngc4593}). The broader blue wings of H$\gamma$ and H$\delta$ result in higher FWHM values of $3900 \pm 400\,\mathrm{km\,s^{-1}}$ and $5900 \pm 1000\,\mathrm{km\,s^{-1}}$, respectively. A simillar line width can also be seen for Ly$\alpha$ with FHWM of $4600 \pm 500\,\mathrm{km\,s^{-1}}$, although it has to be noted that its line profile exhibits strong absorption.\\
The RMS profiles of the helium lines exhibit an extended blue wing and a steep red wing with the He\,\textsc{ii} lines showing broader line widths. This results in FWHM values of $4000 \pm 400\,\mathrm{km\,s^{-1}}$ for He\,\textsc{i}$\,\lambda5876$, and broader values of $6000 \pm 400\,\mathrm{km\,s^{-1}}$ for He\,\textsc{ii}$\,\lambda4686$ and $6900 \pm 1000\,\mathrm{km\,s^{-1}}$ for He\,\textsc{ii}$\,\lambda1640$, respectively.
The line profiles of the high-ionized UV lines partly exhibit absorption and high noise level, which results in significantly higher uncertainties than for the other emission lines. The FWHM values of the UV emission lines were determined to be $3400 \pm 1000\,\mathrm{km\,s^{-1}}$ for N\,\textsc{v}$\,\lambda\lambda 1238,\,1242$ and $8400 \pm 500\,\mathrm{km\,s^{-1}}$ for C\textsc{iv}$\,\lambda\lambda 1548,\,1550$, respectively. The FWHM of the Si\,\textsc{iv}$\,\lambda\lambda 1393,\,1402$ RMS profile could not be reliably estimated due to possible blending with O\,\textsc{iv}] emission, which cannot be separated from its profile. The estimated FWHM of N\,\textsc{v}$\,\lambda\lambda 1238,\,1242$ must also be treated with caution, as it was approximated using only the red wing, since the blue wing is blended with Ly$\alpha$. Excluding N\,\textsc{v}$\,\lambda\lambda 1238,\,1242$, a tendency in the line widths is apparent, with the higher-ionization broad emission lines tending to exhibit larger FWHM values than the lower-ionization broad emission lines. 



\section{Black Hole Mass}

Based on the reverberation mapping results, virial products were derived for the selected emission lines, and the SMBH mass was estimated using a scale factor of $f=6.9$. The adopted scale factor accounts for the low inclination angle of $i \sim 11^\circ$ reported by \cite{ochmann2024transient}. Its estimation is based on the relation $f \sim \sin^{-2}(i)$ \parencite{krolik2001systematic}, which captures the inclination dependence of the observed line-of-sight velocity dispersion. Since the velocity width is parameterized via the FWHM in this work rather than the line dispersion, an additional conversion was applied using $\sigma_{\mathrm{line}} \approx \mathrm{FWHM}/2$ \parencite{peterson2004}. Using this approach, an inverse-variance–weighted mean BH mass of $\bar{M}_{\mathrm{BH}} \approx (5.28 \pm 0.12)\times 10^{7}\,M_\odot$ was obtained from the determined values of each suitable emission line. Earlier works reported values in the same order of magnitude, e.g. $M \approx 1.4 \times 10^7\,M_\odot$ \parencite{kollatschny1997balmer} and $M = (9.8 \pm 2.1) \times 10^6\,M_\odot$ \parencite{denney2006ngc4593}. Using a similar approach, \cite{kollatschny1997balmer} monitored H$\alpha$ and estimated the SMBH mass using a scale factor of $1.5$, whereas \cite{denney2006ngc4593} monitored H$\beta$ and used a scale factor of $5.5$. Rescaling these values to the scale factor of $6.9$ used here yields $M \approx 6.44 \times 10^7\,M_\odot$ \parencite{kollatschny1997balmer} and $M = (1.23 \pm 0.26) \times 10^7\,M_\odot$ \parencite{denney2006ngc4593}. Therefore, the SMBH mass found in this analysis is consistent in order of magnitude with these earlier estimates.

\section{Variation and Bowen Fluorescence of O\,\textsc{I}$\,\lambda8446$}
\label{sec:diskussion_BF}

The low-ionization line O\,\textsc{i}\,$\lambda\,8446$ was included in the RM analysis, as it exhibits noticeable variability in its RMS spectrum and is known to be a Bowen fluorescence line pumped by Ly$\beta$ emission \parencite{grandi1980}. Being monitored in an HST/STIS campaign, it allowed the examination of its relation to Ly$\beta$ for the first time in an RM analysis. Although Ly$\beta$ is not covered by the observations, Ly$\alpha$ was used as a proxy to investigate the expected Bowen fluorescence connection to O\,\textsc{i}\,$\lambda\,8446$. As described in Section \ref{sec:bowen_fluorescence}, Ly$\beta$ excites O\,\textsc{i} through a near-resonant transition at $\lambda\,1025\,\AA$. This is followed by decay via O\,\textsc{i}\,$\lambda\,11287$ and subsequently by O\,\textsc{i}\,$\lambda\,8446$, producing the observed O\,\textsc{i}\,$\lambda\,8446$ emission (see Figure \ref{fig:bowen_cascate}). This cascade is expected to occur with a photon ratio of 1:1 \parencite{grandi1980}. \cite{ochmann2026first} reported a photon ratio of $\sim 0.8$ between O\,\textsc{i}\,$\lambda\,8446$ and O\,\textsc{i}\,$\lambda\,11287$, which was taken from the near-infrared SpeX spectra obtained by \cite{landt2008near}. This result shows that Ly$\beta$ pumping is the dominant excitation mechanism in O\,\textsc{i}\,$\lambda\,8446$ in NGC\,4593. \\
In addition, the comparison shown in Figure \ref{fig:ccfs_Bowen} demonstrates that O\,\textsc{i}\,$\lambda\,8446$ exhibits a significantly stronger correlation with Ly$\alpha$ than with UVW2, further supporting this interpretation. Therefore, the emitting region of O\,\textsc{i} is likely located at a distance corresponding to a lag of $\sim 3.5$ days from the ionizing continuum, consistent with the Bowen fluorescence scenario, rather than the $4.7^{+2.4}_{-1.4}$ days implied by the direct lag relative to UVW2 under the assumption of recombination as the main excitation mechanism. With that, the Bowen fluorescence lag of O\,\textsc{i} corresponds to the radius of the emitting region of H$\alpha$ with $3.3 \ensuremath{_{-0.6}^{+1.1}}\,$ days. Since H$\alpha$ and O\,\textsc{i}\,$\lambda\,8446$ also exhibits a strong correlation with a minimal time shift of $\sim 0.3$ days, it can be inferred that O\,\textsc{i}\,$\lambda\,8446$ is emitted at approximately the same radius as H$\alpha$ within uncertainties.


 
\section{Summary}

This thesis presents the results of the first reverberation-mapping analysis of the broad-line region (BLR) of the Seyfert I galaxy NGC\,4593 based on an HST/STIS campaign (PI: Cackett, E.M, Prop ID 14121). This dataset is ideally suited for reverberation mapping thanks to its daily cadence and its broad wavelength coverage spanning parts of the UV and the optical-to-NIR regimes. This enabled a more detailed investigation of the BLR structure than in previous RM studies of NGC\,4593 by including additional broad emission lines that had not been covered before. The measured time lags and FWHM values are listed in Tables \ref{tab:lags_UVW2} and \ref{tab:line_width_FWHM}, respectively. The results indicate a systematic trend: higher-ionization broad emission lines exhibit shorter time lags and broader line widths than lower-ionization lines, indicating an ionization stratified BLR \parencite{collin1988broad}.\\
Based on the virial products of the reverberation-mapped lines, a SMBH mass of $\bar{M}_{\mathrm{BH}} = (3.35 \pm 0.62)\times 10^{7}\,M_\odot$ was inferred, which agrees in order of magnitude with earlier estimates (e.g. \cite{kollatschny1997balmer, denney2006ngc4593}).\\
In addition, variability of the low-ionization line O\,\textsc{i}\,$\lambda\,8446$ was detected for the first time in a reverberation-mapping analysis. It was show that this variability is majorly driven through Ly$\beta$ fluorescence, making O\,\textsc{i}\,$\lambda\,8446$ the first Bowen-fluorescence line investigated with RM. Following the Bowen-fluorescence process, the emitting region was approximately found at the same distance as the emitting region of H$\alpha$ towards the ionizing continuum.



% The higher order Balmer line exhibit shorter time lags and slightly differences in their RMS line profiles and line widths compared to H$\alpha$ and H$\beta$. Ly$\alpha$ lags even shorter than the Blamer lines, showing a similar line width to the higher order Balmer lines within uncertainties. He\,\textsc{i}$\,\lambda5876$ show a approximately similar time lag as H$\alpha$ while the higher ionized He\,\textsc{ii} exhibit much shorter time lags with significantly broader line widths than He\,\textsc{i}$\,\lambda5876$. Finally, the high-ionized UV emission lines all show similar short time lags compared to the lower-ionized Balmer lines.






