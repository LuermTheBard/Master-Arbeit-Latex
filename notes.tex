\chapter{Ablauf Notizen}


\section{RM of NGC4593}
\begin{itemize}
	\item Intercalibration der Spectra an der Linie \(\text{O [III] } \lambda 5006\)
	\item Linienbestimmung im Spektrum
	\item Ausmessung der Interessanten Linien:
	\begin{itemize}
		\item \(\text{H}\alpha\)
		\item \(\text{H}\beta\)
		\item \(\text{H}\gamma\)
		\item \(\text{H}\delta\)
		\item \(\text{He I } \lambda 4471\) 
		\item \(\text{He I } \lambda 5015\) 
		\item \(\text{He I } \lambda 5875\) 
		\item \(\text{He I } \lambda 7065\) 
		\item \(\text{He II } \lambda 4685\)
		\item \(\text{O I } \lambda 8446\)  
	\end{itemize}
	\item Ausmessung der Continua:
	\begin{itemize}
		\item $\mathrm{Cont1150} \quad (1140\,\text{-}\,1160\,\text{\AA})$
		\item $\mathrm{Cont4010} \quad (4026\,\text{-}\,4033\,\text{\AA})$
		\item $\mathrm{Cont4200} \quad (4197\,\text{-}\,4220\,\text{\AA})$
		\item $\mathrm{Cont4440} \quad (4435\,\text{-}\,4450\,\text{\AA})$
		\item $\mathrm{Cont4765} \quad (4762\,\text{-}\,4774\,\text{\AA})$
		\item $\mathrm{Cont5100} \quad (5085\,\text{-}\,5112\,\text{\AA})$
		\item $\mathrm{Cont5600} \quad (5645\,\text{-}\,5653\,\text{\AA})$
		\item $\mathrm{Cont6045} \quad (6044\,\text{-}\,6057\,\text{\AA})$
		\item $\mathrm{Cont6110} \quad (6107\,\text{-}\,6129\,\text{\AA})$
		\item $\mathrm{Cont6880} \quad (6861\,\text{-}\,6900\,\text{\AA})$
		\item $\mathrm{Cont7390} \quad (7382\,\text{-}\,7405\,\text{\AA})$
		\item $\mathrm{Cont8015} \quad (8005\,\text{-}\,8031\,\text{\AA})$
		\item $\mathrm{Cont8900} \quad (8864\,\text{-}\,8955\,\text{\AA})$
	\end{itemize}
	\item Erstellung von Lichtkurven der Linien und Continua
	\item Gewählte Linien zur Darstellung: 
	\begin{itemize}
		\item \(\text{H}\alpha\)
		\item \(\text{H}\beta\)
		\item \(\text{H}\gamma\) 
		\item \(\text{He I } \lambda 5875\) 
		\item \(\text{He I } \lambda 7065\) 
		\item \(\text{He II } \lambda 4685\)
		\item \(\text{O I } \lambda 8446\)  
	\end{itemize}
	\item Gewählte Continua zur Darstellung: 
	\begin{itemize}
		\item $\mathrm{Cont1150} \quad (1140\,\text{-}\,1160\,\text{\AA})$
		\item $\mathrm{Cont4010} \quad (4026\,\text{-}\,4033\,\text{\AA})$
		\item $\mathrm{Cont4440} \quad (4435\,\text{-}\,4450\,\text{\AA})$
		\item $\mathrm{Cont5100} \quad (5085\,\text{-}\,5112\,\text{\AA})$
		\item $\mathrm{Cont6110} \quad (6107\,\text{-}\,6129\,\text{\AA})$
		\item $\mathrm{Cont6880} \quad (6861\,\text{-}\,6900\,\text{\AA})$
		\item $\mathrm{Cont8015} \quad (8005\,\text{-}\,8031\,\text{\AA})$
		\item $\mathrm{Cont8900} \quad (8864\,\text{-}\,8955\,\text{\AA})$ 
	
	\end{itemize}
	\item Auswahl von Cont 1150 und Cont 5100 für die Erstellung der CCFs.
	\item Bestimmung der Linienprofile von AVG and RMS durch Subtraktion der Pseudocontinua.
	\item Ausmessung des FWHM von AVG und RMS 
	\item Bestimmung der Centroid Verteilung der CCFs zur Bestimung des Time Lags
	\item Bestimmung der BH Masse
	\item Substraktion der pseude conts der intercalibrierten Spektren von $\text{H}\alpha$ und $\text{H}\beta$ und Bestimmung des AVG/RMS
	\item Abzug der narrow line komponenten aus dem AVG (noch nicht fertig)
	
	\item Bowen Fluoreszenz überprüfen
	\item ist wahrscheinlich, aber kann aufgrund der Auflösung nicht nachgewiesen werden
	\item allerdings korreliert OI deutlich mehr mit LyAlpha bzw. H Beta, als mit dem UV Spektrum
	
	\item Bisherige Publikationen nehmen an, das OI nicht variiert und nur durch Photoionisation entsteht. Aber hier variiert OI deutlich
	
	
\end{itemize}
